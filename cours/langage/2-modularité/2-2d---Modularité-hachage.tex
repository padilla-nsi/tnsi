\documentclass[a4paper,17pt]{extarticle}


    \usepackage[sfdefault, condensed]{roboto} % police d'écriture plus moderne
\usepackage[french]{babel} % francisation
\usepackage[parfill]{parskip} %suppression indentation

\usepackage{fancyhdr}
\usepackage{multicol}

% figure non flotantes
\usepackage{float}
\let\origfigure\figure
\let\endorigfigure\endfigure
\renewenvironment{figure}[1][2] {
    \expandafter\origfigure\expandafter[H]
} {
    \endorigfigure
}

% mois/année
\usepackage{datetime}
\newdateformat{monthyeardate}{%
  \monthname[\THEMONTH] \THEYEAR}

% couleurs perso
\usepackage[table]{xcolor}
\definecolor{deepblue}{rgb}{0.3,0.3,0.8}
\definecolor{darkblue}{rgb}{0,0,0.3}
\definecolor{deepred}{rgb}{0.6,0,0}
\definecolor{iremred}{RGB}{204,35,50}
\definecolor{deepgreen}{rgb}{0,0.6,0}
\definecolor{backcolor}{rgb}{0.98,0.95,0.95}
\definecolor{grisClair}{rgb}{0.95,0.95,0.95}
\definecolor{orangeamu}{RGB}{250,178,11}
\definecolor{noiramu}{RGB}{35,31,32}
\definecolor{bleuamu}{RGB}{20,118,198}
\definecolor{bleuamudark}{RGB}{15,90,150}
\definecolor{cyanamu}{RGB}{77,198,244}


\usepackage{/home/bouscadilla/Documents/Code/nbconvert/template/latex/pdf_solution/xeboiboites}
%
% exemple
\newbreakabletheorem[
    small box style={fill=deepblue!90,draw=deepblue!15, rounded corners,line width=1pt},%
    big box style={fill=deepblue!5,draw=deepblue!15,thick,rounded corners,line width=1pt},%
    headfont={\color{white}\bfseries}
        ]{exemple}{Exemple}{}%{counterCo}
%
% remarque
\newbreakabletheorem[
    small box style={draw=ansi-green-intense!100,line width=2pt,fill=ansi-green-intense!0,rounded corners,decoration=penciline, decorate},%
	big box style={color=ansi-green-intense!90,fill=ansi-green-intense!10,thick,decoration={penciline},decorate},
    broken edges={draw=ansi-green-intense!90,thick,fill=orange!20!black!5, decoration={random steps, segment length=.5cm,amplitude=1.3mm},decorate},%
    other edges={decoration=penciline,decorate,thick},%
    headfont={\color{ansi-green-intense}\large\scshape\bfseries}
    ]{remarque}{Remarque}{}%{counterCa}
%
% formule (sans titre)
\newboxedequation[%
    big box style={fill=cyanamu!10,draw=cyanamu!100,thick,decoration=penciline,decorate}]%
    {form}
%
% Réponse
\newbreakabletheorem[
    small box style={fill=bleuamu!100, draw=bleuamu!60, line width=1pt,rounded corners,decorate},%
    big box style={fill=bleuamu!10,draw=bleuamu!30,thick,rounded corners,decorate},
    headfont={\color{white}\large\scshape\bfseries}
        ]{reponse}{Correction}{}
%

%
% À retenir
%\newbreakabletheorem[
%    small box style={fill=deepred!100, draw=deepred!80, line width=1pt,rounded corners,decorate},%
%    big box style={fill=deepred!10,draw=deepred!50,thick,rounded corners,decorate},
%    headfont={\color{white}\large\scshape\bfseries}
%        ]{retenir}{À retenir}{}
%
\newboxedequation[%
    big box style={fill=deepred!10,draw=deepred!0,thick,decoration=penciline,decorate}]%
    {retenir}



% astuce
\newspanning[
    image=/home/bouscadilla/Documents/Code/nbconvert/template/latex/pdf_solution/fig-idee,headfont=\bfseries,
    spanning style={very thick,decoration=penciline,decorate}
    ]{astuce}{Astuce}{}
%
% activité
\newbreakabletheorem[
    small box style={draw=orangeamu!100,line width=2pt,fill=orangeamu!100,rounded corners,decoration=penciline, decorate},%
	big box style={color=orangeamu!100,fill=orangeamu!5,thick,decoration={penciline},decorate},
    broken edges={draw=orangeamu!100,thick,fill=orangeamu!100, decoration={random steps, segment length=.5cm,amplitude=1.3mm},decorate},%
    other edges={decoration=penciline,decorate,thick},%
    headfont={\color{white}\large\scshape\bfseries}
    ]{activite}{\adjustimage{height=1cm, valign=m}{/home/bouscadilla/Documents/Code/nbconvert/template/latex/pdf_solution/papier_eleve_investigation.png}%
    Activité}{}%{counterCa}
%   
%   environnement élève
%
\newenvironment{eleve}%
%{\begin{activite}\large\\} % écrire plus gros
{\begin{activite}\color{noiramu}\\[-0.5cm]}
{\end{activite}}

\newenvironment{formule}%
%{\begin{activite}\large\\} % écrire plus gros
{\begin{form}\color{bleuamu}}
{\end{form}}


\usepackage[breakable]{tcolorbox}
    \usepackage{parskip} % Stop auto-indenting (to mimic markdown behaviour)
    
    \usepackage{iftex}
    \ifPDFTeX
    	\usepackage[T1]{fontenc}
    	\usepackage{mathpazo}
    \else
    	\usepackage{fontspec}
    \fi

    % Basic figure setup, for now with no caption control since it's done
    % automatically by Pandoc (which extracts ![](path) syntax from Markdown).
    \usepackage{graphicx}
    % Maintain compatibility with old templates. Remove in nbconvert 6.0
    \let\Oldincludegraphics\includegraphics
    % Ensure that by default, figures have no caption (until we provide a
    % proper Figure object with a Caption API and a way to capture that
    % in the conversion process - todo).
    \usepackage{caption}
    \DeclareCaptionFormat{nocaption}{}
    \captionsetup{format=nocaption,aboveskip=0pt,belowskip=0pt}

    \usepackage[Export]{adjustbox} % Used to constrain images to a maximum size
    \adjustboxset{max size={0.9\linewidth}{0.9\paperheight}}
    \usepackage{float}
    \floatplacement{figure}{H} % forces figures to be placed at the correct location
    \usepackage{xcolor} % Allow colors to be defined
    \usepackage{enumerate} % Needed for markdown enumerations to work
    \usepackage{geometry} % Used to adjust the document margins
    \usepackage{amsmath} % Equations
    \usepackage{amssymb} % Equations
    \usepackage{textcomp} % defines textquotesingle
    % Hack from http://tex.stackexchange.com/a/47451/13684:
    \AtBeginDocument{%
        \def\PYZsq{\textquotesingle}% Upright quotes in Pygmentized code
    }
    \usepackage{upquote} % Upright quotes for verbatim code
    \usepackage{eurosym} % defines \euro
    \usepackage[mathletters]{ucs} % Extended unicode (utf-8) support
    \usepackage{fancyvrb} % verbatim replacement that allows latex

    % The hyperref package gives us a pdf with properly built
    % internal navigation ('pdf bookmarks' for the table of contents,
    % internal cross-reference links, web links for URLs, etc.)
    \usepackage{hyperref}
    % The default LaTeX title has an obnoxious amount of whitespace. By default,
    % titling removes some of it. It also provides customization options.
    \usepackage{titling}
    \usepackage{longtable} % longtable support required by pandoc >1.10
    \usepackage{booktabs}  % table support for pandoc > 1.12.2
    \usepackage[inline]{enumitem} % IRkernel/repr support (it uses the enumerate* environment)
    \usepackage[normalem]{ulem} % ulem is needed to support strikethroughs (\sout)
                                % normalem makes italics be italics, not underlines
    \usepackage{mathrsfs}
    

    
    % Colors for the hyperref package
    \definecolor{urlcolor}{rgb}{0,.145,.698}
    \definecolor{linkcolor}{rgb}{.71,0.21,0.01}
    \definecolor{citecolor}{rgb}{.12,.54,.11}

    % ANSI colors
    \definecolor{ansi-black}{HTML}{3E424D}
    \definecolor{ansi-black-intense}{HTML}{282C36}
    \definecolor{ansi-red}{HTML}{E75C58}
    \definecolor{ansi-red-intense}{HTML}{B22B31}
    \definecolor{ansi-green}{HTML}{00A250}
    \definecolor{ansi-green-intense}{HTML}{007427}
    \definecolor{ansi-yellow}{HTML}{DDB62B}
    \definecolor{ansi-yellow-intense}{HTML}{B27D12}
    \definecolor{ansi-blue}{HTML}{208FFB}
    \definecolor{ansi-blue-intense}{HTML}{0065CA}
    \definecolor{ansi-magenta}{HTML}{D160C4}
    \definecolor{ansi-magenta-intense}{HTML}{A03196}
    \definecolor{ansi-cyan}{HTML}{60C6C8}
    \definecolor{ansi-cyan-intense}{HTML}{258F8F}
    \definecolor{ansi-white}{HTML}{C5C1B4}
    \definecolor{ansi-white-intense}{HTML}{A1A6B2}
    \definecolor{ansi-default-inverse-fg}{HTML}{FFFFFF}
    \definecolor{ansi-default-inverse-bg}{HTML}{000000}

    % commands and environments needed by pandoc snippets
    % extracted from the output of `pandoc -s`
    \providecommand{\tightlist}{%
      \setlength{\itemsep}{0pt}\setlength{\parskip}{0pt}}
    \DefineVerbatimEnvironment{Highlighting}{Verbatim}{commandchars=\\\{\}}
    % Add ',fontsize=\small' for more characters per line
    \newenvironment{Shaded}{}{}
    \newcommand{\KeywordTok}[1]{\textcolor[rgb]{0.00,0.44,0.13}{\textbf{{#1}}}}
    \newcommand{\DataTypeTok}[1]{\textcolor[rgb]{0.56,0.13,0.00}{{#1}}}
    \newcommand{\DecValTok}[1]{\textcolor[rgb]{0.25,0.63,0.44}{{#1}}}
    \newcommand{\BaseNTok}[1]{\textcolor[rgb]{0.25,0.63,0.44}{{#1}}}
    \newcommand{\FloatTok}[1]{\textcolor[rgb]{0.25,0.63,0.44}{{#1}}}
    \newcommand{\CharTok}[1]{\textcolor[rgb]{0.25,0.44,0.63}{{#1}}}
    \newcommand{\StringTok}[1]{\textcolor[rgb]{0.25,0.44,0.63}{{#1}}}
    \newcommand{\CommentTok}[1]{\textcolor[rgb]{0.38,0.63,0.69}{\textit{{#1}}}}
    \newcommand{\OtherTok}[1]{\textcolor[rgb]{0.00,0.44,0.13}{{#1}}}
    \newcommand{\AlertTok}[1]{\textcolor[rgb]{1.00,0.00,0.00}{\textbf{{#1}}}}
    \newcommand{\FunctionTok}[1]{\textcolor[rgb]{0.02,0.16,0.49}{{#1}}}
    \newcommand{\RegionMarkerTok}[1]{{#1}}
    \newcommand{\ErrorTok}[1]{\textcolor[rgb]{1.00,0.00,0.00}{\textbf{{#1}}}}
    \newcommand{\NormalTok}[1]{{#1}}
    
    % Additional commands for more recent versions of Pandoc
    \newcommand{\ConstantTok}[1]{\textcolor[rgb]{0.53,0.00,0.00}{{#1}}}
    \newcommand{\SpecialCharTok}[1]{\textcolor[rgb]{0.25,0.44,0.63}{{#1}}}
    \newcommand{\VerbatimStringTok}[1]{\textcolor[rgb]{0.25,0.44,0.63}{{#1}}}
    \newcommand{\SpecialStringTok}[1]{\textcolor[rgb]{0.73,0.40,0.53}{{#1}}}
    \newcommand{\ImportTok}[1]{{#1}}
    \newcommand{\DocumentationTok}[1]{\textcolor[rgb]{0.73,0.13,0.13}{\textit{{#1}}}}
    \newcommand{\AnnotationTok}[1]{\textcolor[rgb]{0.38,0.63,0.69}{\textbf{\textit{{#1}}}}}
    \newcommand{\CommentVarTok}[1]{\textcolor[rgb]{0.38,0.63,0.69}{\textbf{\textit{{#1}}}}}
    \newcommand{\VariableTok}[1]{\textcolor[rgb]{0.10,0.09,0.49}{{#1}}}
    \newcommand{\ControlFlowTok}[1]{\textcolor[rgb]{0.00,0.44,0.13}{\textbf{{#1}}}}
    \newcommand{\OperatorTok}[1]{\textcolor[rgb]{0.40,0.40,0.40}{{#1}}}
    \newcommand{\BuiltInTok}[1]{{#1}}
    \newcommand{\ExtensionTok}[1]{{#1}}
    \newcommand{\PreprocessorTok}[1]{\textcolor[rgb]{0.74,0.48,0.00}{{#1}}}
    \newcommand{\AttributeTok}[1]{\textcolor[rgb]{0.49,0.56,0.16}{{#1}}}
    \newcommand{\InformationTok}[1]{\textcolor[rgb]{0.38,0.63,0.69}{\textbf{\textit{{#1}}}}}
    \newcommand{\WarningTok}[1]{\textcolor[rgb]{0.38,0.63,0.69}{\textbf{\textit{{#1}}}}}
    
    
    % Define a nice break command that doesn't care if a line doesn't already
    % exist.
    \def\br{\hspace*{\fill} \\* }
    % Math Jax compatibility definitions
    \def\gt{>}
    \def\lt{<}
    \let\Oldtex\TeX
    \let\Oldlatex\LaTeX
    \renewcommand{\TeX}{\textrm{\Oldtex}}
    \renewcommand{\LaTeX}{\textrm{\Oldlatex}}
    % Document parameters
    % Document title
    \title{2-2d---Modularité-hachage}
    
    
    
    
    
% Pygments definitions
\makeatletter
\def\PY@reset{\let\PY@it=\relax \let\PY@bf=\relax%
    \let\PY@ul=\relax \let\PY@tc=\relax%
    \let\PY@bc=\relax \let\PY@ff=\relax}
\def\PY@tok#1{\csname PY@tok@#1\endcsname}
\def\PY@toks#1+{\ifx\relax#1\empty\else%
    \PY@tok{#1}\expandafter\PY@toks\fi}
\def\PY@do#1{\PY@bc{\PY@tc{\PY@ul{%
    \PY@it{\PY@bf{\PY@ff{#1}}}}}}}
\def\PY#1#2{\PY@reset\PY@toks#1+\relax+\PY@do{#2}}

\expandafter\def\csname PY@tok@w\endcsname{\def\PY@tc##1{\textcolor[rgb]{0.73,0.73,0.73}{##1}}}
\expandafter\def\csname PY@tok@c\endcsname{\let\PY@it=\textit\def\PY@tc##1{\textcolor[rgb]{0.25,0.50,0.50}{##1}}}
\expandafter\def\csname PY@tok@cp\endcsname{\def\PY@tc##1{\textcolor[rgb]{0.74,0.48,0.00}{##1}}}
\expandafter\def\csname PY@tok@k\endcsname{\let\PY@bf=\textbf\def\PY@tc##1{\textcolor[rgb]{0.00,0.50,0.00}{##1}}}
\expandafter\def\csname PY@tok@kp\endcsname{\def\PY@tc##1{\textcolor[rgb]{0.00,0.50,0.00}{##1}}}
\expandafter\def\csname PY@tok@kt\endcsname{\def\PY@tc##1{\textcolor[rgb]{0.69,0.00,0.25}{##1}}}
\expandafter\def\csname PY@tok@o\endcsname{\def\PY@tc##1{\textcolor[rgb]{0.40,0.40,0.40}{##1}}}
\expandafter\def\csname PY@tok@ow\endcsname{\let\PY@bf=\textbf\def\PY@tc##1{\textcolor[rgb]{0.67,0.13,1.00}{##1}}}
\expandafter\def\csname PY@tok@nb\endcsname{\def\PY@tc##1{\textcolor[rgb]{0.00,0.50,0.00}{##1}}}
\expandafter\def\csname PY@tok@nf\endcsname{\def\PY@tc##1{\textcolor[rgb]{0.00,0.00,1.00}{##1}}}
\expandafter\def\csname PY@tok@nc\endcsname{\let\PY@bf=\textbf\def\PY@tc##1{\textcolor[rgb]{0.00,0.00,1.00}{##1}}}
\expandafter\def\csname PY@tok@nn\endcsname{\let\PY@bf=\textbf\def\PY@tc##1{\textcolor[rgb]{0.00,0.00,1.00}{##1}}}
\expandafter\def\csname PY@tok@ne\endcsname{\let\PY@bf=\textbf\def\PY@tc##1{\textcolor[rgb]{0.82,0.25,0.23}{##1}}}
\expandafter\def\csname PY@tok@nv\endcsname{\def\PY@tc##1{\textcolor[rgb]{0.10,0.09,0.49}{##1}}}
\expandafter\def\csname PY@tok@no\endcsname{\def\PY@tc##1{\textcolor[rgb]{0.53,0.00,0.00}{##1}}}
\expandafter\def\csname PY@tok@nl\endcsname{\def\PY@tc##1{\textcolor[rgb]{0.63,0.63,0.00}{##1}}}
\expandafter\def\csname PY@tok@ni\endcsname{\let\PY@bf=\textbf\def\PY@tc##1{\textcolor[rgb]{0.60,0.60,0.60}{##1}}}
\expandafter\def\csname PY@tok@na\endcsname{\def\PY@tc##1{\textcolor[rgb]{0.49,0.56,0.16}{##1}}}
\expandafter\def\csname PY@tok@nt\endcsname{\let\PY@bf=\textbf\def\PY@tc##1{\textcolor[rgb]{0.00,0.50,0.00}{##1}}}
\expandafter\def\csname PY@tok@nd\endcsname{\def\PY@tc##1{\textcolor[rgb]{0.67,0.13,1.00}{##1}}}
\expandafter\def\csname PY@tok@s\endcsname{\def\PY@tc##1{\textcolor[rgb]{0.73,0.13,0.13}{##1}}}
\expandafter\def\csname PY@tok@sd\endcsname{\let\PY@it=\textit\def\PY@tc##1{\textcolor[rgb]{0.73,0.13,0.13}{##1}}}
\expandafter\def\csname PY@tok@si\endcsname{\let\PY@bf=\textbf\def\PY@tc##1{\textcolor[rgb]{0.73,0.40,0.53}{##1}}}
\expandafter\def\csname PY@tok@se\endcsname{\let\PY@bf=\textbf\def\PY@tc##1{\textcolor[rgb]{0.73,0.40,0.13}{##1}}}
\expandafter\def\csname PY@tok@sr\endcsname{\def\PY@tc##1{\textcolor[rgb]{0.73,0.40,0.53}{##1}}}
\expandafter\def\csname PY@tok@ss\endcsname{\def\PY@tc##1{\textcolor[rgb]{0.10,0.09,0.49}{##1}}}
\expandafter\def\csname PY@tok@sx\endcsname{\def\PY@tc##1{\textcolor[rgb]{0.00,0.50,0.00}{##1}}}
\expandafter\def\csname PY@tok@m\endcsname{\def\PY@tc##1{\textcolor[rgb]{0.40,0.40,0.40}{##1}}}
\expandafter\def\csname PY@tok@gh\endcsname{\let\PY@bf=\textbf\def\PY@tc##1{\textcolor[rgb]{0.00,0.00,0.50}{##1}}}
\expandafter\def\csname PY@tok@gu\endcsname{\let\PY@bf=\textbf\def\PY@tc##1{\textcolor[rgb]{0.50,0.00,0.50}{##1}}}
\expandafter\def\csname PY@tok@gd\endcsname{\def\PY@tc##1{\textcolor[rgb]{0.63,0.00,0.00}{##1}}}
\expandafter\def\csname PY@tok@gi\endcsname{\def\PY@tc##1{\textcolor[rgb]{0.00,0.63,0.00}{##1}}}
\expandafter\def\csname PY@tok@gr\endcsname{\def\PY@tc##1{\textcolor[rgb]{1.00,0.00,0.00}{##1}}}
\expandafter\def\csname PY@tok@ge\endcsname{\let\PY@it=\textit}
\expandafter\def\csname PY@tok@gs\endcsname{\let\PY@bf=\textbf}
\expandafter\def\csname PY@tok@gp\endcsname{\let\PY@bf=\textbf\def\PY@tc##1{\textcolor[rgb]{0.00,0.00,0.50}{##1}}}
\expandafter\def\csname PY@tok@go\endcsname{\def\PY@tc##1{\textcolor[rgb]{0.53,0.53,0.53}{##1}}}
\expandafter\def\csname PY@tok@gt\endcsname{\def\PY@tc##1{\textcolor[rgb]{0.00,0.27,0.87}{##1}}}
\expandafter\def\csname PY@tok@err\endcsname{\def\PY@bc##1{\setlength{\fboxsep}{0pt}\fcolorbox[rgb]{1.00,0.00,0.00}{1,1,1}{\strut ##1}}}
\expandafter\def\csname PY@tok@kc\endcsname{\let\PY@bf=\textbf\def\PY@tc##1{\textcolor[rgb]{0.00,0.50,0.00}{##1}}}
\expandafter\def\csname PY@tok@kd\endcsname{\let\PY@bf=\textbf\def\PY@tc##1{\textcolor[rgb]{0.00,0.50,0.00}{##1}}}
\expandafter\def\csname PY@tok@kn\endcsname{\let\PY@bf=\textbf\def\PY@tc##1{\textcolor[rgb]{0.00,0.50,0.00}{##1}}}
\expandafter\def\csname PY@tok@kr\endcsname{\let\PY@bf=\textbf\def\PY@tc##1{\textcolor[rgb]{0.00,0.50,0.00}{##1}}}
\expandafter\def\csname PY@tok@bp\endcsname{\def\PY@tc##1{\textcolor[rgb]{0.00,0.50,0.00}{##1}}}
\expandafter\def\csname PY@tok@fm\endcsname{\def\PY@tc##1{\textcolor[rgb]{0.00,0.00,1.00}{##1}}}
\expandafter\def\csname PY@tok@vc\endcsname{\def\PY@tc##1{\textcolor[rgb]{0.10,0.09,0.49}{##1}}}
\expandafter\def\csname PY@tok@vg\endcsname{\def\PY@tc##1{\textcolor[rgb]{0.10,0.09,0.49}{##1}}}
\expandafter\def\csname PY@tok@vi\endcsname{\def\PY@tc##1{\textcolor[rgb]{0.10,0.09,0.49}{##1}}}
\expandafter\def\csname PY@tok@vm\endcsname{\def\PY@tc##1{\textcolor[rgb]{0.10,0.09,0.49}{##1}}}
\expandafter\def\csname PY@tok@sa\endcsname{\def\PY@tc##1{\textcolor[rgb]{0.73,0.13,0.13}{##1}}}
\expandafter\def\csname PY@tok@sb\endcsname{\def\PY@tc##1{\textcolor[rgb]{0.73,0.13,0.13}{##1}}}
\expandafter\def\csname PY@tok@sc\endcsname{\def\PY@tc##1{\textcolor[rgb]{0.73,0.13,0.13}{##1}}}
\expandafter\def\csname PY@tok@dl\endcsname{\def\PY@tc##1{\textcolor[rgb]{0.73,0.13,0.13}{##1}}}
\expandafter\def\csname PY@tok@s2\endcsname{\def\PY@tc##1{\textcolor[rgb]{0.73,0.13,0.13}{##1}}}
\expandafter\def\csname PY@tok@sh\endcsname{\def\PY@tc##1{\textcolor[rgb]{0.73,0.13,0.13}{##1}}}
\expandafter\def\csname PY@tok@s1\endcsname{\def\PY@tc##1{\textcolor[rgb]{0.73,0.13,0.13}{##1}}}
\expandafter\def\csname PY@tok@mb\endcsname{\def\PY@tc##1{\textcolor[rgb]{0.40,0.40,0.40}{##1}}}
\expandafter\def\csname PY@tok@mf\endcsname{\def\PY@tc##1{\textcolor[rgb]{0.40,0.40,0.40}{##1}}}
\expandafter\def\csname PY@tok@mh\endcsname{\def\PY@tc##1{\textcolor[rgb]{0.40,0.40,0.40}{##1}}}
\expandafter\def\csname PY@tok@mi\endcsname{\def\PY@tc##1{\textcolor[rgb]{0.40,0.40,0.40}{##1}}}
\expandafter\def\csname PY@tok@il\endcsname{\def\PY@tc##1{\textcolor[rgb]{0.40,0.40,0.40}{##1}}}
\expandafter\def\csname PY@tok@mo\endcsname{\def\PY@tc##1{\textcolor[rgb]{0.40,0.40,0.40}{##1}}}
\expandafter\def\csname PY@tok@ch\endcsname{\let\PY@it=\textit\def\PY@tc##1{\textcolor[rgb]{0.25,0.50,0.50}{##1}}}
\expandafter\def\csname PY@tok@cm\endcsname{\let\PY@it=\textit\def\PY@tc##1{\textcolor[rgb]{0.25,0.50,0.50}{##1}}}
\expandafter\def\csname PY@tok@cpf\endcsname{\let\PY@it=\textit\def\PY@tc##1{\textcolor[rgb]{0.25,0.50,0.50}{##1}}}
\expandafter\def\csname PY@tok@c1\endcsname{\let\PY@it=\textit\def\PY@tc##1{\textcolor[rgb]{0.25,0.50,0.50}{##1}}}
\expandafter\def\csname PY@tok@cs\endcsname{\let\PY@it=\textit\def\PY@tc##1{\textcolor[rgb]{0.25,0.50,0.50}{##1}}}

\def\PYZbs{\char`\\}
\def\PYZus{\char`\_}
\def\PYZob{\char`\{}
\def\PYZcb{\char`\}}
\def\PYZca{\char`\^}
\def\PYZam{\char`\&}
\def\PYZlt{\char`\<}
\def\PYZgt{\char`\>}
\def\PYZsh{\char`\#}
\def\PYZpc{\char`\%}
\def\PYZdl{\char`\$}
\def\PYZhy{\char`\-}
\def\PYZsq{\char`\'}
\def\PYZdq{\char`\"}
\def\PYZti{\char`\~}
% for compatibility with earlier versions
\def\PYZat{@}
\def\PYZlb{[}
\def\PYZrb{]}
\makeatother


    % For linebreaks inside Verbatim environment from package fancyvrb. 
    \makeatletter
        \newbox\Wrappedcontinuationbox 
        \newbox\Wrappedvisiblespacebox 
        \newcommand*\Wrappedvisiblespace {\textcolor{red}{\textvisiblespace}} 
        \newcommand*\Wrappedcontinuationsymbol {\textcolor{red}{\llap{\tiny$\m@th\hookrightarrow$}}} 
        \newcommand*\Wrappedcontinuationindent {3ex } 
        \newcommand*\Wrappedafterbreak {\kern\Wrappedcontinuationindent\copy\Wrappedcontinuationbox} 
        % Take advantage of the already applied Pygments mark-up to insert 
        % potential linebreaks for TeX processing. 
        %        {, <, #, %, $, ' and ": go to next line. 
        %        _, }, ^, &, >, - and ~: stay at end of broken line. 
        % Use of \textquotesingle for straight quote. 
        \newcommand*\Wrappedbreaksatspecials {% 
            \def\PYGZus{\discretionary{\char`\_}{\Wrappedafterbreak}{\char`\_}}% 
            \def\PYGZob{\discretionary{}{\Wrappedafterbreak\char`\{}{\char`\{}}% 
            \def\PYGZcb{\discretionary{\char`\}}{\Wrappedafterbreak}{\char`\}}}% 
            \def\PYGZca{\discretionary{\char`\^}{\Wrappedafterbreak}{\char`\^}}% 
            \def\PYGZam{\discretionary{\char`\&}{\Wrappedafterbreak}{\char`\&}}% 
            \def\PYGZlt{\discretionary{}{\Wrappedafterbreak\char`\<}{\char`\<}}% 
            \def\PYGZgt{\discretionary{\char`\>}{\Wrappedafterbreak}{\char`\>}}% 
            \def\PYGZsh{\discretionary{}{\Wrappedafterbreak\char`\#}{\char`\#}}% 
            \def\PYGZpc{\discretionary{}{\Wrappedafterbreak\char`\%}{\char`\%}}% 
            \def\PYGZdl{\discretionary{}{\Wrappedafterbreak\char`\$}{\char`\$}}% 
            \def\PYGZhy{\discretionary{\char`\-}{\Wrappedafterbreak}{\char`\-}}% 
            \def\PYGZsq{\discretionary{}{\Wrappedafterbreak\textquotesingle}{\textquotesingle}}% 
            \def\PYGZdq{\discretionary{}{\Wrappedafterbreak\char`\"}{\char`\"}}% 
            \def\PYGZti{\discretionary{\char`\~}{\Wrappedafterbreak}{\char`\~}}% 
        } 
        % Some characters . , ; ? ! / are not pygmentized. 
        % This macro makes them "active" and they will insert potential linebreaks 
        \newcommand*\Wrappedbreaksatpunct {% 
            \lccode`\~`\.\lowercase{\def~}{\discretionary{\hbox{\char`\.}}{\Wrappedafterbreak}{\hbox{\char`\.}}}% 
            \lccode`\~`\,\lowercase{\def~}{\discretionary{\hbox{\char`\,}}{\Wrappedafterbreak}{\hbox{\char`\,}}}% 
            \lccode`\~`\;\lowercase{\def~}{\discretionary{\hbox{\char`\;}}{\Wrappedafterbreak}{\hbox{\char`\;}}}% 
            \lccode`\~`\:\lowercase{\def~}{\discretionary{\hbox{\char`\:}}{\Wrappedafterbreak}{\hbox{\char`\:}}}% 
            \lccode`\~`\?\lowercase{\def~}{\discretionary{\hbox{\char`\?}}{\Wrappedafterbreak}{\hbox{\char`\?}}}% 
            \lccode`\~`\!\lowercase{\def~}{\discretionary{\hbox{\char`\!}}{\Wrappedafterbreak}{\hbox{\char`\!}}}% 
            \lccode`\~`\/\lowercase{\def~}{\discretionary{\hbox{\char`\/}}{\Wrappedafterbreak}{\hbox{\char`\/}}}% 
            \catcode`\.\active
            \catcode`\,\active 
            \catcode`\;\active
            \catcode`\:\active
            \catcode`\?\active
            \catcode`\!\active
            \catcode`\/\active 
            \lccode`\~`\~ 	
        }
    \makeatother

    \let\OriginalVerbatim=\Verbatim
    \makeatletter
    \renewcommand{\Verbatim}[1][1]{%
        %\parskip\z@skip
        \sbox\Wrappedcontinuationbox {\Wrappedcontinuationsymbol}%
        \sbox\Wrappedvisiblespacebox {\FV@SetupFont\Wrappedvisiblespace}%
        \def\FancyVerbFormatLine ##1{\hsize\linewidth
            \vtop{\raggedright\hyphenpenalty\z@\exhyphenpenalty\z@
                \doublehyphendemerits\z@\finalhyphendemerits\z@
                \strut ##1\strut}%
        }%
        % If the linebreak is at a space, the latter will be displayed as visible
        % space at end of first line, and a continuation symbol starts next line.
        % Stretch/shrink are however usually zero for typewriter font.
        \def\FV@Space {%
            \nobreak\hskip\z@ plus\fontdimen3\font minus\fontdimen4\font
            \discretionary{\copy\Wrappedvisiblespacebox}{\Wrappedafterbreak}
            {\kern\fontdimen2\font}%
        }%
        
        % Allow breaks at special characters using \PYG... macros.
        \Wrappedbreaksatspecials
        % Breaks at punctuation characters . , ; ? ! and / need catcode=\active 	
        \OriginalVerbatim[#1,codes*=\Wrappedbreaksatpunct]%
    }
    \makeatother

    % Exact colors from NB
    \definecolor{incolor}{HTML}{303F9F}
    \definecolor{outcolor}{HTML}{D84315}
    \definecolor{cellborder}{HTML}{CFCFCF}
    \definecolor{cellbackground}{HTML}{F7F7F7}
    
    % prompt
    \makeatletter
    \newcommand{\boxspacing}{\kern\kvtcb@left@rule\kern\kvtcb@boxsep}
    \makeatother
    \newcommand{\prompt}[4]{
        \ttfamily\llap{{\color{#2}[#3]:\hspace{3pt}#4}}\vspace{-\baselineskip}
    }
    

    
\setlength\headheight{30pt}
\setcounter{secnumdepth}{0} % Turns off numbering for sections

    % Prevent overflowing lines due to hard-to-break entities
    \sloppy 
    % Setup hyperref package
    \hypersetup{
      breaklinks=true,  % so long urls are correctly broken across lines
      colorlinks=true,
      urlcolor=urlcolor,
      linkcolor=linkcolor,
      citecolor=citecolor,
      }
    % Slightly bigger margins than the latex defaults
    \geometry{a4paper,tmargin=3cm,bmargin=2cm,lmargin=1cm,rmargin=1cm}\fancyhead[L]{Thème à définir}\fancyhead[L]{\adjustimage{height=1cm, valign=m}{/home/bouscadilla/Documents/Code/nbconvert/template/latex/pdf_solution/papier_eleve_ico_langage}\ttfamily\scshape Langage}\fancyhead[C]{\bfseries\MakeUppercase{2-2d---Modularité-hachage}}\fancyhead[C]{\bfseries\MakeUppercase{2 --- Modularité}}\fancyhead[R]{\monthyeardate\today}

    \fancyfoot[C]{\thepage}
    % #TODO ajouter les pages totales

    \pagestyle{fancy}
    


\begin{document}
    
    \title{2 --- Modularité}
% \maketitle

    
    

    
    \hypertarget{introduction-aux-tables-de-hachages}{%
\subsection{Introduction aux tables de
hachages}\label{introduction-aux-tables-de-hachages}}
\begin{formule}
    Dans le programme 2 du chapitre \emph{Modularité}, \texttt{s} est un
tableau définit par \texttt{s={[}{]}} :

\begin{itemize}
\tightlist
\item
  le tableau \texttt{s} est petit et prend peu de place en mémoire
\item
  \textbf{mais} la recherche \texttt{if\ x\ in\ s:} n'est pas immédiate.
  Dans \emph{le pire des cas}, il faut parcourir tout le tableau pour
  être certain que \texttt{x} n'y est pas.
\end{itemize}

Au contraire, dans le programme 3, \texttt{s} est définit par
\texttt{s\ =\ {[}False{]}\ *\ 366}:

\begin{itemize}
\tightlist
\item
  la recherche \texttt{if\ s{[}x{]}:} est immédiate
\item
  \textbf{mais} le tableau \texttt{s} prend beaucoup de place en
  mémoire.
\end{itemize}

        \end{formule}\begin{retenir}
    Le programme 4 ci-dessous te propose une solution qui prend le meilleur
des deux tentatives précédentes :

\begin{itemize}
\tightlist
\item
  peu de place en mémoire (comme le programme 2)
\item
  quasi immédiateté de la recherche (comme le programme 3).
\end{itemize}

\begin{Shaded}
\begin{Highlighting}[]

\KeywordTok{def}\NormalTok{ contient\_doublon(t):}
    \CommentTok{"""le tableau t contient{-}il un doublon ?"""}
\NormalTok{    s }\OperatorTok{=}\NormalTok{ [[] }\ControlFlowTok{for}\NormalTok{ \_ }\KeywordTok{in} \BuiltInTok{range}\NormalTok{(}\DecValTok{23}\NormalTok{)]}
    \ControlFlowTok{for}\NormalTok{ x }\KeywordTok{in}\NormalTok{ t:}
        \ControlFlowTok{if}\NormalTok{ x }\KeywordTok{in}\NormalTok{ s[x }\OperatorTok{\%} \DecValTok{23}\NormalTok{]:}
            \ControlFlowTok{return} \VariableTok{True}
\NormalTok{        s[x }\OperatorTok{\%} \DecValTok{23}\NormalTok{].append(x)}
    \ControlFlowTok{return} \VariableTok{False}
\end{Highlighting}
\end{Shaded}

        \end{retenir}\begin{remarque}
    On crée un tableau \texttt{s} de 23 cases car on sait qu'il n'y aura 23
dates à y enregistrer. L'occupation en mémoire est donc faible.

Ensuite, on attribut à chacune des 365 dates possibles une case
\emph{fixe et bien définie}. Par exemple, la date 42 sera toujours
rangée dans \texttt{s{[}19{]}}.

Comment sait-on que la date \texttt{42} est enregistrée dans
l'emplacement \texttt{19} de \texttt{s} ? Pour obtenir ce rang
(\texttt{19}) associé à la date \texttt{42}, on utilise l'opération
\textbf{modulo 23} (notée \texttt{\%\ 23}) qui renvoie le \emph{reste de
la division euclidienne par 23}. Cette opération renvoie un nombre
compris entre 0..22. Ce qui est parfait car le tableau \texttt{s}
contient 23 emplacements. Le calcul du rang \texttt{42\ \%\ 23} donne
pour résultat \texttt{19} et est immédiat.

        \end{remarque}\begin{remarque}
    Mais il est \textbf{possible} que plusieurs dates soient dans la même
case. Par exemple les dates 65 ou 88 se rangeront aussi dans
\texttt{s{[}19{]}}.

C'est pourquoi chaque case de \texttt{s} ne contient pas une date, mais
un tableau de date que nous appellerons \textbf{paquet}. C'est donc
pourquoi \texttt{s\ =\ {[}{[}{]}\ for\ \_\ in\ range(23){]}}.

Pour rendre la recherche \emph{quasi} immédiate, il faut que chaque
paquet soit quasiment vide. Puisque le tirage des 23 dates est
aléatoire, il faudrait beaucoup beaucoup de malchance pour qu'un grand
nombre d'entre elles se trouvent dans le même paquet.

Par exemple, imaginons que la date tirée soit 42. On sait
\emph{immédiatement} qu'il faut chercher dans le paquet
\texttt{s{[}19{]}}. Maintenant si ce paquet contient beaucoup de dates
(pas de chance !) la recherche prend du temps. Sinon, le paquet est
quasiment vide et la recherche est \emph{quasi immédiate} !

On peut donc conclure que, \emph{en moyenne}, la recherche
\texttt{x\ in\ s{[}x\ \%\ 23{]}} est \emph{quasi immédiate}.

        \end{remarque}\begin{retenir}
    La méthode exposée ci-dessus est une ébauche de la structure de données
fondamentale \textbf{table de hachage}.

Cette structure de données est sous-jacente aux \textbf{ensembles} et
aux \textbf{dictionnaires} de Python. Elle est très polyvalente, permet
de représenter des ensembles de taille arbitraire avec des opérations
d'accès aux éléments extrêmement rapides. Cette structure de données est
considérée comme la plus efficace dans la plus grande variété des cas
courants !

        \end{retenir}
    \hypertarget{amuxe9lioration-du-moduxe8le-de-tables-de-hachages}{%
\subsection{\texorpdfstring{Amélioration du modèle de \textbf{tables de
hachages}}{Amélioration du modèle de tables de hachages}}\label{amuxe9lioration-du-moduxe8le-de-tables-de-hachages}}

    Dans la suite, nous allons améliorer notre modèle de \emph{tables de
hachage} afin de :
\begin{formule}
    Représenter des ensembles de taille \emph{arbitraire} de façon efficace.
Par exemple, ne plus limiter à 23 paquets !

        \end{formule}\begin{formule}
    Rendre vraiment aléatoire la répartition dans les paquets. Par exemple,
faire en sorte que les multiples, les nombres proches ou liés les uns
aux autres ne soient pas systématiquement associés au même paquet.

        \end{formule}\begin{formule}
    Rendre l'ensemble plus polyvalent en y associant autre chose que des
nombres entiers. Par exemple on aimerait aussi associer des chaînes de
caractères comme \texttt{"alice"} ou \texttt{"bob"} à différents
paquets.

        \end{formule}\begin{retenir}
    Voici ce que donne le programme final :

\begin{Shaded}
\begin{Highlighting}[]
\KeywordTok{def}\NormalTok{ cree():}
    \ControlFlowTok{return}\NormalTok{ \{ }\StringTok{\textquotesingle{}taille\textquotesingle{}}\NormalTok{:}\DecValTok{0}\NormalTok{ ,}
             \StringTok{\textquotesingle{}paquets\textquotesingle{}}\NormalTok{: [[] }\ControlFlowTok{for}\NormalTok{ \_ }\KeywordTok{in} \BuiltInTok{range}\NormalTok{(}\DecValTok{32}\NormalTok{)] \}}

\KeywordTok{def}\NormalTok{ contient(s, x):}
\NormalTok{    p }\OperatorTok{=} \BuiltInTok{hash}\NormalTok{(x) }\OperatorTok{\%} \BuiltInTok{len}\NormalTok{(s[}\StringTok{\textquotesingle{}paquets\textquotesingle{}}\NormalTok{])}
    \ControlFlowTok{return}\NormalTok{ x }\KeywordTok{in}\NormalTok{ s[}\StringTok{\textquotesingle{}paquets\textquotesingle{}}\NormalTok{][p]}

\KeywordTok{def}\NormalTok{ ajoute(s,x):}
    \ControlFlowTok{if}\NormalTok{ contient(s, x):}
        \ControlFlowTok{return}
\NormalTok{    s[}\StringTok{\textquotesingle{}taille\textquotesingle{}}\NormalTok{] }\OperatorTok{+=} \DecValTok{1}
    \ControlFlowTok{if}\NormalTok{ s[}\StringTok{\textquotesingle{}taille\textquotesingle{}}\NormalTok{] }\OperatorTok{\textgreater{}} \BuiltInTok{len}\NormalTok{(s[}\StringTok{\textquotesingle{}paquets\textquotesingle{}}\NormalTok{]):}
\NormalTok{        \_etend(s)}
\NormalTok{    \_ajoute\_aux(s[}\StringTok{\textquotesingle{}paquets\textquotesingle{}}\NormalTok{], x)}

\KeywordTok{def}\NormalTok{ \_ajoute\_aux(t, x):}
\NormalTok{    p }\OperatorTok{=} \BuiltInTok{hash}\NormalTok{(x) }\OperatorTok{\%} \BuiltInTok{len}\NormalTok{(t)}
\NormalTok{    t[p].append(x)}

\KeywordTok{def}\NormalTok{ \_etend(s):}
\NormalTok{    tmp }\OperatorTok{=}\NormalTok{ [[] }\ControlFlowTok{for}\NormalTok{ \_ }\KeywordTok{in} \BuiltInTok{range}\NormalTok{( }\DecValTok{2} \OperatorTok{*} \BuiltInTok{len}\NormalTok{(s[}\StringTok{\textquotesingle{}paquets\textquotesingle{}}\NormalTok{]) )]}
    \ControlFlowTok{for}\NormalTok{ x }\KeywordTok{in}\NormalTok{ enumere(s):}
\NormalTok{        \_ajoute\_aux(tmp, x)}
\NormalTok{    s[}\StringTok{\textquotesingle{}paquets\textquotesingle{}}\NormalTok{] }\OperatorTok{=}\NormalTok{ tmp}

\KeywordTok{def}\NormalTok{ enumere(s):}
\NormalTok{    tab }\OperatorTok{=}\NormalTok{ []}
    \ControlFlowTok{for}\NormalTok{ paquet }\KeywordTok{in}\NormalTok{ s[}\StringTok{\textquotesingle{}paquets\textquotesingle{}}\NormalTok{]:}
\NormalTok{        tab.extend(paquet)}
    \ControlFlowTok{return}\NormalTok{ tab}
\end{Highlighting}
\end{Shaded}

        \end{retenir}
    \hypertarget{fonction-de-hachage}{%
\subsubsection{Fonction de hachage}\label{fonction-de-hachage}}

    Commençons par les deux derniers points. Pour obtenir une vraie table de
hachage, on va utiliser une fonction appelée \emph{fonction de hachage},
qui prend en paramètre un élément à stocker (nombre entier, chaîne de
caractère, flottant, objet) et renvoie un nombre entier définissant le
numéro dans lequel insérer l'élément.

En utilisant cette fonction \emph{modulo le nombre de paquets}, on
s'assure que l'association entre un objet et son paquet est aléatoire et
indépendant.

    En Python, l'appel à une fonction de hachage se fait par la fonction
\texttt{hash(obj)} où \texttt{obj} est l'objet à associer.

Pour remplir un tableau de 23 cases avec cette fonction de hachage, on
remplace tout simplement \texttt{x\ \%\ 23} par
\texttt{hash(x)\ \%\ 23}. La fonction \texttt{hash(x)} s'occupe de la
répartition aléatoire et \texttt{\%\ 23} se charge de répartir le nombre
entier obtenu en une nombre appartenant à 0..22.

C'est pourquoi dans le programme on a
\texttt{hash(x)\ \%\ len(s{[}\textquotesingle{}paquets\textquotesingle{}{]})}
ou encore \texttt{hash(x)\ \%\ len(t)} : on obtient le rang \texttt{p}
de la date \texttt{x} en effectuant un \emph{modulo le nombre de
paquets} de la table de hachage.

    \hypertarget{tableau-de-taille-variable}{%
\subsubsection{Tableau de taille
variable}\label{tableau-de-taille-variable}}

    Afin de rendre la table de hachage efficace, il faut s'assurer que
chaque paquet reste le plus petit possible. En effet, si un paquet
contient beaucoup d'éléments, la recherche dans ce paquet n'est plus
efficace. En revanche, si le paquet est vide ou contient un ou deux
éléments, la recherche est \emph{immédiate}.

On va alors se fixer la règle suivante :
\begin{formule}
    \emph{la taille de la table de hachage (le nombre de paquets) ne doit
jamais être inférieure au nombre d'éléments stockés qui y sont stockés}.

        \end{formule}
    Ainsi, c'est la procédure \texttt{ajoute(s,x)} qui s'occupe d'ajouter
l'élément \texttt{x} à la table de hachage \texttt{s} et qui a en charge
d'augmenter la taille de la table.

Si \texttt{x} est déjà présent \texttt{if\ contient(s,\ x)}, on ignore
toutes les instructions suivante et on appelle un \texttt{return} tout
seul qui termine l'appel à la fonction \texttt{ajoute()}.

Si \texttt{x} n'est pas déjà dans \texttt{s}, alors les lignes après le
\texttt{return} s'exécutent. Puisque \texttt{x} va être ajouté, on
incrémente la taille de la table de hachage
\texttt{s{[}\textquotesingle{}taille\textquotesingle{}{]}\ +=\ 1}.

Ensuite la procédure vérifie si le nombre d'éléments dépasse le nombre
de paquets
\texttt{if\ s{[}\textquotesingle{}taille\textquotesingle{}{]}\ \textgreater{}\ len(s{[}\textquotesingle{}paquets\textquotesingle{}{]}):}.
Si c'est le cas, la procédure \texttt{\_etend(s)} va :

\begin{itemize}
\tightlist
\item
  créer un tableau deux fois plus grand que la table de hachage
  \texttt{tmp\ =\ {[}{[}{]}\ for\ \_\ in\ range(\ 2\ *\ len(s{[}\textquotesingle{}paquets\textquotesingle{}{]})\ ){]}}
\item
  répartir tous les éléments de la table de hachage dans ce tableau
  temporaire \texttt{for\ x\ in\ enumere(s):\ \_ajoute\_aux(tmp,\ x)}
\item
  faire pointer la table de hachage vers le tableau temporaire
  \texttt{s{[}\textquotesingle{}paquets\textquotesingle{}{]}\ =\ tmp}.
\end{itemize}

Pour finir, un appel à la procédure \texttt{\_ajoute\_aux()} s'occupe
(enfin) d'ajouter l'élément \texttt{x} dans le paquet auquel il est
associé
\texttt{\_ajoute\_aux(s{[}\textquotesingle{}paquets\textquotesingle{}{]},\ x)}.

    \hypertarget{test-du-module}{%
\subsubsection{Test du module}\label{test-du-module}}

        {\scriptsize
    \begin{tcolorbox}[breakable, size=fbox, boxrule=1pt, pad at break*=1mm,colback=cellbackground, colframe=cellborder]
\prompt{In}{incolor}{11}{\boxspacing}
\begin{Verbatim}[commandchars=\\\{\}]
\PY{c+c1}{\PYZsh{} créer doublon}
\PY{k}{def} \PY{n+nf}{contient\PYZus{}doublon}\PY{p}{(}\PY{n}{t}\PY{p}{)}\PY{p}{:}
    \PY{l+s+sd}{\PYZdq{}\PYZdq{}\PYZdq{}la structure contient\PYZhy{}elle un doublon?\PYZdq{}\PYZdq{}\PYZdq{}}
    \PY{n}{s} \PY{o}{=} \PY{n}{cree}\PY{p}{(}\PY{p}{)}
    \PY{k}{for} \PY{n}{x} \PY{o+ow}{in} \PY{n}{t}\PY{p}{:}
        \PY{k}{if} \PY{n}{contient}\PY{p}{(}\PY{n}{s}\PY{p}{,} \PY{n}{x}\PY{p}{)}\PY{p}{:}
            \PY{k}{return} \PY{k+kc}{True}
        \PY{n}{ajoute}\PY{p}{(}\PY{n}{s}\PY{p}{,} \PY{n}{x}\PY{p}{)}
    \PY{k}{return} \PY{k+kc}{False}

\PY{c+c1}{\PYZsh{} Création du tableau de dates aléatoires}
\PY{k+kn}{from} \PY{n+nn}{random} \PY{k+kn}{import} \PY{n}{randint}

\PY{n}{n} \PY{o}{=} \PY{l+m+mi}{0}
\PY{n}{n\PYZus{}doublons} \PY{o}{=} \PY{l+m+mi}{0}

\PY{k}{while} \PY{n}{n} \PY{o}{\PYZlt{}} \PY{l+m+mi}{1000} \PY{p}{:}
    \PY{n}{t} \PY{o}{=} \PY{p}{[}\PY{k+kc}{None}\PY{p}{]} \PY{o}{*} \PY{l+m+mi}{23}
    \PY{k}{for} \PY{n}{j} \PY{o+ow}{in} \PY{n+nb}{range}\PY{p}{(}\PY{l+m+mi}{23}\PY{p}{)}\PY{p}{:}
        \PY{n}{t}\PY{p}{[}\PY{n}{j}\PY{p}{]} \PY{o}{=} \PY{n}{randint}\PY{p}{(}\PY{l+m+mi}{1}\PY{p}{,}\PY{l+m+mi}{365}\PY{p}{)}
    \PY{k}{if} \PY{n}{contient\PYZus{}doublon}\PY{p}{(}\PY{n}{t}\PY{p}{)}\PY{p}{:}
        \PY{n}{n\PYZus{}doublons} \PY{o}{+}\PY{o}{=} \PY{l+m+mi}{1}    
    \PY{n}{n} \PY{o}{+}\PY{o}{=} \PY{l+m+mi}{1}

\PY{n+nb}{print} \PY{p}{(}\PY{n}{n\PYZus{}doublons}\PY{p}{,}\PY{l+s+s2}{\PYZdq{}}\PY{l+s+s2}{doublons sur}\PY{l+s+s2}{\PYZdq{}}\PY{p}{,}\PY{n}{n}\PY{p}{,}\PY{l+s+s2}{\PYZdq{}}\PY{l+s+s2}{tirages}\PY{l+s+s2}{\PYZdq{}}\PY{p}{)}
\PY{n+nb}{print} \PY{p}{(}\PY{l+s+s2}{\PYZdq{}}\PY{l+s+s2}{fréquence : }\PY{l+s+s2}{\PYZdq{}}\PY{p}{,} \PY{n}{n\PYZus{}doublons}\PY{o}{/}\PY{n}{n}\PY{p}{)}
\end{Verbatim}
\end{tcolorbox}
    }

    \begin{Verbatim}[commandchars=\\\{\}]
483 doublons sur 1000 tirages
fréquence :  0.483
    \end{Verbatim}


    % Add a bibliography block to the postdoc
    
    
    
\end{document}
