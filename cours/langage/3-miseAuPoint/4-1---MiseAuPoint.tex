\documentclass[a4paper,17pt]{extarticle}


    \usepackage[sfdefault, condensed]{roboto} % police d'écriture plus moderne
\usepackage[french]{babel} % francisation
\usepackage[parfill]{parskip} %suppression indentation

\usepackage{fancyhdr}
\usepackage{multicol}

% figure non flotantes
\usepackage{float}
\let\origfigure\figure
\let\endorigfigure\endfigure
\renewenvironment{figure}[1][2] {
    \expandafter\origfigure\expandafter[H]
} {
    \endorigfigure
}

% mois/année
\usepackage{datetime}
\newdateformat{monthyeardate}{%
  \monthname[\THEMONTH] \THEYEAR}

% couleurs perso
\usepackage[table]{xcolor}
\definecolor{deepblue}{rgb}{0.3,0.3,0.8}
\definecolor{darkblue}{rgb}{0,0,0.3}
\definecolor{deepred}{rgb}{0.6,0,0}
\definecolor{iremred}{RGB}{204,35,50}
\definecolor{deepgreen}{rgb}{0,0.6,0}
\definecolor{backcolor}{rgb}{0.98,0.95,0.95}
\definecolor{grisClair}{rgb}{0.95,0.95,0.95}
\definecolor{orangeamu}{RGB}{250,178,11}
\definecolor{noiramu}{RGB}{35,31,32}
\definecolor{bleuamu}{RGB}{20,118,198}
\definecolor{bleuamudark}{RGB}{15,90,150}
\definecolor{cyanamu}{RGB}{77,198,244}


\usepackage{/home/bouscadilla/Documents/Code/nbconvert/template/latex/pdf_solution/xeboiboites}
%
% exemple
\newbreakabletheorem[
    small box style={fill=deepblue!90,draw=deepblue!15, rounded corners,line width=1pt},%
    big box style={fill=deepblue!5,draw=deepblue!15,thick,rounded corners,line width=1pt},%
    headfont={\color{white}\bfseries}
        ]{exemple}{Exemple}{}%{counterCo}
%
% remarque
\newbreakabletheorem[
    small box style={draw=ansi-green-intense!100,line width=2pt,fill=ansi-green-intense!0,rounded corners,decoration=penciline, decorate},%
	big box style={color=ansi-green-intense!90,fill=ansi-green-intense!10,thick,decoration={penciline},decorate},
    broken edges={draw=ansi-green-intense!90,thick,fill=orange!20!black!5, decoration={random steps, segment length=.5cm,amplitude=1.3mm},decorate},%
    other edges={decoration=penciline,decorate,thick},%
    headfont={\color{ansi-green-intense}\large\scshape\bfseries}
    ]{remarque}{Remarque}{}%{counterCa}
%
% formule (sans titre)
\newboxedequation[%
    big box style={fill=cyanamu!10,draw=cyanamu!100,thick,decoration=penciline,decorate}]%
    {form}
%
% Réponse
\newbreakabletheorem[
    small box style={fill=bleuamu!100, draw=bleuamu!60, line width=1pt,rounded corners,decorate},%
    big box style={fill=bleuamu!10,draw=bleuamu!30,thick,rounded corners,decorate},
    headfont={\color{white}\large\scshape\bfseries}
        ]{reponse}{Correction}{}
%

%
% À retenir
%\newbreakabletheorem[
%    small box style={fill=deepred!100, draw=deepred!80, line width=1pt,rounded corners,decorate},%
%    big box style={fill=deepred!10,draw=deepred!50,thick,rounded corners,decorate},
%    headfont={\color{white}\large\scshape\bfseries}
%        ]{retenir}{À retenir}{}
%
\newboxedequation[%
    big box style={fill=deepred!10,draw=deepred!0,thick,decoration=penciline,decorate}]%
    {retenir}



% astuce
\newspanning[
    image=/home/bouscadilla/Documents/Code/nbconvert/template/latex/pdf_solution/fig-idee,headfont=\bfseries,
    spanning style={very thick,decoration=penciline,decorate}
    ]{astuce}{Astuce}{}
%
% activité

\newcounter{counterCa}
\newbreakabletheorem[
    small box style={draw=orangeamu!100,line width=2pt,fill=orangeamu!100,rounded corners,decoration=penciline, decorate},%
	big box style={color=orangeamu!100,fill=orangeamu!5,thick,decoration={penciline},decorate},
    broken edges={draw=orangeamu!100,thick,fill=orangeamu!100, decoration={random steps, segment length=.5cm,amplitude=1.3mm},decorate},%
    other edges={decoration=penciline,decorate,thick},%
    headfont={\color{white}\large\scshape\bfseries}
    ]{activite}{\adjustimage{height=1cm, valign=m}{/home/bouscadilla/Documents/Code/nbconvert/template/latex/pdf_solution/papier_eleve_investigation.png}%
    Activité}{counterCa}
%   
%   environnement élève
%
\newenvironment{eleve}%
%{\begin{activite}\large\\} % écrire plus gros
{\begin{activite}\color{noiramu}\\[-0.5cm]}
{\end{activite}}

\newenvironment{formule}%
%{\begin{activite}\large\\} % écrire plus gros
{\begin{form}\color{bleuamu}}
{\end{form}}


\usepackage[breakable]{tcolorbox}
    \usepackage{parskip} % Stop auto-indenting (to mimic markdown behaviour)
    
    \usepackage{iftex}
    \ifPDFTeX
    	\usepackage[T1]{fontenc}
    	\usepackage{mathpazo}
    \else
    	\usepackage{fontspec}
    \fi

    % Basic figure setup, for now with no caption control since it's done
    % automatically by Pandoc (which extracts ![](path) syntax from Markdown).
    \usepackage{graphicx}
    % Maintain compatibility with old templates. Remove in nbconvert 6.0
    \let\Oldincludegraphics\includegraphics
    % Ensure that by default, figures have no caption (until we provide a
    % proper Figure object with a Caption API and a way to capture that
    % in the conversion process - todo).
    \usepackage{caption}
    \DeclareCaptionFormat{nocaption}{}
    \captionsetup{format=nocaption,aboveskip=0pt,belowskip=0pt}

    \usepackage[Export]{adjustbox} % Used to constrain images to a maximum size
    \adjustboxset{max size={0.9\linewidth}{0.9\paperheight}}
    \usepackage{float}
    \floatplacement{figure}{H} % forces figures to be placed at the correct location
    \usepackage{xcolor} % Allow colors to be defined
    \usepackage{enumerate} % Needed for markdown enumerations to work
    \usepackage{geometry} % Used to adjust the document margins
    \usepackage{amsmath} % Equations
    \usepackage{amssymb} % Equations
    \usepackage{textcomp} % defines textquotesingle
    % Hack from http://tex.stackexchange.com/a/47451/13684:
    \AtBeginDocument{%
        \def\PYZsq{\textquotesingle}% Upright quotes in Pygmentized code
    }
    \usepackage{upquote} % Upright quotes for verbatim code
    \usepackage{eurosym} % defines \euro
    \usepackage[mathletters]{ucs} % Extended unicode (utf-8) support
    \usepackage{fancyvrb} % verbatim replacement that allows latex

    % The hyperref package gives us a pdf with properly built
    % internal navigation ('pdf bookmarks' for the table of contents,
    % internal cross-reference links, web links for URLs, etc.)
    \usepackage{hyperref}
    % The default LaTeX title has an obnoxious amount of whitespace. By default,
    % titling removes some of it. It also provides customization options.
    \usepackage{titling}
    \usepackage{longtable} % longtable support required by pandoc >1.10
    \usepackage{booktabs}  % table support for pandoc > 1.12.2
    \usepackage[inline]{enumitem} % IRkernel/repr support (it uses the enumerate* environment)
    \usepackage[normalem]{ulem} % ulem is needed to support strikethroughs (\sout)
                                % normalem makes italics be italics, not underlines
    \usepackage{mathrsfs}
    

    
    % Colors for the hyperref package
    \definecolor{urlcolor}{rgb}{0,.145,.698}
    \definecolor{linkcolor}{rgb}{.71,0.21,0.01}
    \definecolor{citecolor}{rgb}{.12,.54,.11}

    % ANSI colors
    \definecolor{ansi-black}{HTML}{3E424D}
    \definecolor{ansi-black-intense}{HTML}{282C36}
    \definecolor{ansi-red}{HTML}{E75C58}
    \definecolor{ansi-red-intense}{HTML}{B22B31}
    \definecolor{ansi-green}{HTML}{00A250}
    \definecolor{ansi-green-intense}{HTML}{007427}
    \definecolor{ansi-yellow}{HTML}{DDB62B}
    \definecolor{ansi-yellow-intense}{HTML}{B27D12}
    \definecolor{ansi-blue}{HTML}{208FFB}
    \definecolor{ansi-blue-intense}{HTML}{0065CA}
    \definecolor{ansi-magenta}{HTML}{D160C4}
    \definecolor{ansi-magenta-intense}{HTML}{A03196}
    \definecolor{ansi-cyan}{HTML}{60C6C8}
    \definecolor{ansi-cyan-intense}{HTML}{258F8F}
    \definecolor{ansi-white}{HTML}{C5C1B4}
    \definecolor{ansi-white-intense}{HTML}{A1A6B2}
    \definecolor{ansi-default-inverse-fg}{HTML}{FFFFFF}
    \definecolor{ansi-default-inverse-bg}{HTML}{000000}

    % commands and environments needed by pandoc snippets
    % extracted from the output of `pandoc -s`
    \providecommand{\tightlist}{%
      \setlength{\itemsep}{0pt}\setlength{\parskip}{0pt}}
    \DefineVerbatimEnvironment{Highlighting}{Verbatim}{commandchars=\\\{\}}
    % Add ',fontsize=\small' for more characters per line
    \newenvironment{Shaded}{}{}
    \newcommand{\KeywordTok}[1]{\textcolor[rgb]{0.00,0.44,0.13}{\textbf{{#1}}}}
    \newcommand{\DataTypeTok}[1]{\textcolor[rgb]{0.56,0.13,0.00}{{#1}}}
    \newcommand{\DecValTok}[1]{\textcolor[rgb]{0.25,0.63,0.44}{{#1}}}
    \newcommand{\BaseNTok}[1]{\textcolor[rgb]{0.25,0.63,0.44}{{#1}}}
    \newcommand{\FloatTok}[1]{\textcolor[rgb]{0.25,0.63,0.44}{{#1}}}
    \newcommand{\CharTok}[1]{\textcolor[rgb]{0.25,0.44,0.63}{{#1}}}
    \newcommand{\StringTok}[1]{\textcolor[rgb]{0.25,0.44,0.63}{{#1}}}
    \newcommand{\CommentTok}[1]{\textcolor[rgb]{0.38,0.63,0.69}{\textit{{#1}}}}
    \newcommand{\OtherTok}[1]{\textcolor[rgb]{0.00,0.44,0.13}{{#1}}}
    \newcommand{\AlertTok}[1]{\textcolor[rgb]{1.00,0.00,0.00}{\textbf{{#1}}}}
    \newcommand{\FunctionTok}[1]{\textcolor[rgb]{0.02,0.16,0.49}{{#1}}}
    \newcommand{\RegionMarkerTok}[1]{{#1}}
    \newcommand{\ErrorTok}[1]{\textcolor[rgb]{1.00,0.00,0.00}{\textbf{{#1}}}}
    \newcommand{\NormalTok}[1]{{#1}}
    
    % Additional commands for more recent versions of Pandoc
    \newcommand{\ConstantTok}[1]{\textcolor[rgb]{0.53,0.00,0.00}{{#1}}}
    \newcommand{\SpecialCharTok}[1]{\textcolor[rgb]{0.25,0.44,0.63}{{#1}}}
    \newcommand{\VerbatimStringTok}[1]{\textcolor[rgb]{0.25,0.44,0.63}{{#1}}}
    \newcommand{\SpecialStringTok}[1]{\textcolor[rgb]{0.73,0.40,0.53}{{#1}}}
    \newcommand{\ImportTok}[1]{{#1}}
    \newcommand{\DocumentationTok}[1]{\textcolor[rgb]{0.73,0.13,0.13}{\textit{{#1}}}}
    \newcommand{\AnnotationTok}[1]{\textcolor[rgb]{0.38,0.63,0.69}{\textbf{\textit{{#1}}}}}
    \newcommand{\CommentVarTok}[1]{\textcolor[rgb]{0.38,0.63,0.69}{\textbf{\textit{{#1}}}}}
    \newcommand{\VariableTok}[1]{\textcolor[rgb]{0.10,0.09,0.49}{{#1}}}
    \newcommand{\ControlFlowTok}[1]{\textcolor[rgb]{0.00,0.44,0.13}{\textbf{{#1}}}}
    \newcommand{\OperatorTok}[1]{\textcolor[rgb]{0.40,0.40,0.40}{{#1}}}
    \newcommand{\BuiltInTok}[1]{{#1}}
    \newcommand{\ExtensionTok}[1]{{#1}}
    \newcommand{\PreprocessorTok}[1]{\textcolor[rgb]{0.74,0.48,0.00}{{#1}}}
    \newcommand{\AttributeTok}[1]{\textcolor[rgb]{0.49,0.56,0.16}{{#1}}}
    \newcommand{\InformationTok}[1]{\textcolor[rgb]{0.38,0.63,0.69}{\textbf{\textit{{#1}}}}}
    \newcommand{\WarningTok}[1]{\textcolor[rgb]{0.38,0.63,0.69}{\textbf{\textit{{#1}}}}}
    
    
    % Define a nice break command that doesn't care if a line doesn't already
    % exist.
    \def\br{\hspace*{\fill} \\* }
    % Math Jax compatibility definitions
    \def\gt{>}
    \def\lt{<}
    \let\Oldtex\TeX
    \let\Oldlatex\LaTeX
    \renewcommand{\TeX}{\textrm{\Oldtex}}
    \renewcommand{\LaTeX}{\textrm{\Oldlatex}}
    % Document parameters
    % Document title
    \title{4-1---MiseAuPoint}
    
    
    
    
    
% Pygments definitions
\makeatletter
\def\PY@reset{\let\PY@it=\relax \let\PY@bf=\relax%
    \let\PY@ul=\relax \let\PY@tc=\relax%
    \let\PY@bc=\relax \let\PY@ff=\relax}
\def\PY@tok#1{\csname PY@tok@#1\endcsname}
\def\PY@toks#1+{\ifx\relax#1\empty\else%
    \PY@tok{#1}\expandafter\PY@toks\fi}
\def\PY@do#1{\PY@bc{\PY@tc{\PY@ul{%
    \PY@it{\PY@bf{\PY@ff{#1}}}}}}}
\def\PY#1#2{\PY@reset\PY@toks#1+\relax+\PY@do{#2}}

\expandafter\def\csname PY@tok@w\endcsname{\def\PY@tc##1{\textcolor[rgb]{0.73,0.73,0.73}{##1}}}
\expandafter\def\csname PY@tok@c\endcsname{\let\PY@it=\textit\def\PY@tc##1{\textcolor[rgb]{0.25,0.50,0.50}{##1}}}
\expandafter\def\csname PY@tok@cp\endcsname{\def\PY@tc##1{\textcolor[rgb]{0.74,0.48,0.00}{##1}}}
\expandafter\def\csname PY@tok@k\endcsname{\let\PY@bf=\textbf\def\PY@tc##1{\textcolor[rgb]{0.00,0.50,0.00}{##1}}}
\expandafter\def\csname PY@tok@kp\endcsname{\def\PY@tc##1{\textcolor[rgb]{0.00,0.50,0.00}{##1}}}
\expandafter\def\csname PY@tok@kt\endcsname{\def\PY@tc##1{\textcolor[rgb]{0.69,0.00,0.25}{##1}}}
\expandafter\def\csname PY@tok@o\endcsname{\def\PY@tc##1{\textcolor[rgb]{0.40,0.40,0.40}{##1}}}
\expandafter\def\csname PY@tok@ow\endcsname{\let\PY@bf=\textbf\def\PY@tc##1{\textcolor[rgb]{0.67,0.13,1.00}{##1}}}
\expandafter\def\csname PY@tok@nb\endcsname{\def\PY@tc##1{\textcolor[rgb]{0.00,0.50,0.00}{##1}}}
\expandafter\def\csname PY@tok@nf\endcsname{\def\PY@tc##1{\textcolor[rgb]{0.00,0.00,1.00}{##1}}}
\expandafter\def\csname PY@tok@nc\endcsname{\let\PY@bf=\textbf\def\PY@tc##1{\textcolor[rgb]{0.00,0.00,1.00}{##1}}}
\expandafter\def\csname PY@tok@nn\endcsname{\let\PY@bf=\textbf\def\PY@tc##1{\textcolor[rgb]{0.00,0.00,1.00}{##1}}}
\expandafter\def\csname PY@tok@ne\endcsname{\let\PY@bf=\textbf\def\PY@tc##1{\textcolor[rgb]{0.82,0.25,0.23}{##1}}}
\expandafter\def\csname PY@tok@nv\endcsname{\def\PY@tc##1{\textcolor[rgb]{0.10,0.09,0.49}{##1}}}
\expandafter\def\csname PY@tok@no\endcsname{\def\PY@tc##1{\textcolor[rgb]{0.53,0.00,0.00}{##1}}}
\expandafter\def\csname PY@tok@nl\endcsname{\def\PY@tc##1{\textcolor[rgb]{0.63,0.63,0.00}{##1}}}
\expandafter\def\csname PY@tok@ni\endcsname{\let\PY@bf=\textbf\def\PY@tc##1{\textcolor[rgb]{0.60,0.60,0.60}{##1}}}
\expandafter\def\csname PY@tok@na\endcsname{\def\PY@tc##1{\textcolor[rgb]{0.49,0.56,0.16}{##1}}}
\expandafter\def\csname PY@tok@nt\endcsname{\let\PY@bf=\textbf\def\PY@tc##1{\textcolor[rgb]{0.00,0.50,0.00}{##1}}}
\expandafter\def\csname PY@tok@nd\endcsname{\def\PY@tc##1{\textcolor[rgb]{0.67,0.13,1.00}{##1}}}
\expandafter\def\csname PY@tok@s\endcsname{\def\PY@tc##1{\textcolor[rgb]{0.73,0.13,0.13}{##1}}}
\expandafter\def\csname PY@tok@sd\endcsname{\let\PY@it=\textit\def\PY@tc##1{\textcolor[rgb]{0.73,0.13,0.13}{##1}}}
\expandafter\def\csname PY@tok@si\endcsname{\let\PY@bf=\textbf\def\PY@tc##1{\textcolor[rgb]{0.73,0.40,0.53}{##1}}}
\expandafter\def\csname PY@tok@se\endcsname{\let\PY@bf=\textbf\def\PY@tc##1{\textcolor[rgb]{0.73,0.40,0.13}{##1}}}
\expandafter\def\csname PY@tok@sr\endcsname{\def\PY@tc##1{\textcolor[rgb]{0.73,0.40,0.53}{##1}}}
\expandafter\def\csname PY@tok@ss\endcsname{\def\PY@tc##1{\textcolor[rgb]{0.10,0.09,0.49}{##1}}}
\expandafter\def\csname PY@tok@sx\endcsname{\def\PY@tc##1{\textcolor[rgb]{0.00,0.50,0.00}{##1}}}
\expandafter\def\csname PY@tok@m\endcsname{\def\PY@tc##1{\textcolor[rgb]{0.40,0.40,0.40}{##1}}}
\expandafter\def\csname PY@tok@gh\endcsname{\let\PY@bf=\textbf\def\PY@tc##1{\textcolor[rgb]{0.00,0.00,0.50}{##1}}}
\expandafter\def\csname PY@tok@gu\endcsname{\let\PY@bf=\textbf\def\PY@tc##1{\textcolor[rgb]{0.50,0.00,0.50}{##1}}}
\expandafter\def\csname PY@tok@gd\endcsname{\def\PY@tc##1{\textcolor[rgb]{0.63,0.00,0.00}{##1}}}
\expandafter\def\csname PY@tok@gi\endcsname{\def\PY@tc##1{\textcolor[rgb]{0.00,0.63,0.00}{##1}}}
\expandafter\def\csname PY@tok@gr\endcsname{\def\PY@tc##1{\textcolor[rgb]{1.00,0.00,0.00}{##1}}}
\expandafter\def\csname PY@tok@ge\endcsname{\let\PY@it=\textit}
\expandafter\def\csname PY@tok@gs\endcsname{\let\PY@bf=\textbf}
\expandafter\def\csname PY@tok@gp\endcsname{\let\PY@bf=\textbf\def\PY@tc##1{\textcolor[rgb]{0.00,0.00,0.50}{##1}}}
\expandafter\def\csname PY@tok@go\endcsname{\def\PY@tc##1{\textcolor[rgb]{0.53,0.53,0.53}{##1}}}
\expandafter\def\csname PY@tok@gt\endcsname{\def\PY@tc##1{\textcolor[rgb]{0.00,0.27,0.87}{##1}}}
\expandafter\def\csname PY@tok@err\endcsname{\def\PY@bc##1{\setlength{\fboxsep}{0pt}\fcolorbox[rgb]{1.00,0.00,0.00}{1,1,1}{\strut ##1}}}
\expandafter\def\csname PY@tok@kc\endcsname{\let\PY@bf=\textbf\def\PY@tc##1{\textcolor[rgb]{0.00,0.50,0.00}{##1}}}
\expandafter\def\csname PY@tok@kd\endcsname{\let\PY@bf=\textbf\def\PY@tc##1{\textcolor[rgb]{0.00,0.50,0.00}{##1}}}
\expandafter\def\csname PY@tok@kn\endcsname{\let\PY@bf=\textbf\def\PY@tc##1{\textcolor[rgb]{0.00,0.50,0.00}{##1}}}
\expandafter\def\csname PY@tok@kr\endcsname{\let\PY@bf=\textbf\def\PY@tc##1{\textcolor[rgb]{0.00,0.50,0.00}{##1}}}
\expandafter\def\csname PY@tok@bp\endcsname{\def\PY@tc##1{\textcolor[rgb]{0.00,0.50,0.00}{##1}}}
\expandafter\def\csname PY@tok@fm\endcsname{\def\PY@tc##1{\textcolor[rgb]{0.00,0.00,1.00}{##1}}}
\expandafter\def\csname PY@tok@vc\endcsname{\def\PY@tc##1{\textcolor[rgb]{0.10,0.09,0.49}{##1}}}
\expandafter\def\csname PY@tok@vg\endcsname{\def\PY@tc##1{\textcolor[rgb]{0.10,0.09,0.49}{##1}}}
\expandafter\def\csname PY@tok@vi\endcsname{\def\PY@tc##1{\textcolor[rgb]{0.10,0.09,0.49}{##1}}}
\expandafter\def\csname PY@tok@vm\endcsname{\def\PY@tc##1{\textcolor[rgb]{0.10,0.09,0.49}{##1}}}
\expandafter\def\csname PY@tok@sa\endcsname{\def\PY@tc##1{\textcolor[rgb]{0.73,0.13,0.13}{##1}}}
\expandafter\def\csname PY@tok@sb\endcsname{\def\PY@tc##1{\textcolor[rgb]{0.73,0.13,0.13}{##1}}}
\expandafter\def\csname PY@tok@sc\endcsname{\def\PY@tc##1{\textcolor[rgb]{0.73,0.13,0.13}{##1}}}
\expandafter\def\csname PY@tok@dl\endcsname{\def\PY@tc##1{\textcolor[rgb]{0.73,0.13,0.13}{##1}}}
\expandafter\def\csname PY@tok@s2\endcsname{\def\PY@tc##1{\textcolor[rgb]{0.73,0.13,0.13}{##1}}}
\expandafter\def\csname PY@tok@sh\endcsname{\def\PY@tc##1{\textcolor[rgb]{0.73,0.13,0.13}{##1}}}
\expandafter\def\csname PY@tok@s1\endcsname{\def\PY@tc##1{\textcolor[rgb]{0.73,0.13,0.13}{##1}}}
\expandafter\def\csname PY@tok@mb\endcsname{\def\PY@tc##1{\textcolor[rgb]{0.40,0.40,0.40}{##1}}}
\expandafter\def\csname PY@tok@mf\endcsname{\def\PY@tc##1{\textcolor[rgb]{0.40,0.40,0.40}{##1}}}
\expandafter\def\csname PY@tok@mh\endcsname{\def\PY@tc##1{\textcolor[rgb]{0.40,0.40,0.40}{##1}}}
\expandafter\def\csname PY@tok@mi\endcsname{\def\PY@tc##1{\textcolor[rgb]{0.40,0.40,0.40}{##1}}}
\expandafter\def\csname PY@tok@il\endcsname{\def\PY@tc##1{\textcolor[rgb]{0.40,0.40,0.40}{##1}}}
\expandafter\def\csname PY@tok@mo\endcsname{\def\PY@tc##1{\textcolor[rgb]{0.40,0.40,0.40}{##1}}}
\expandafter\def\csname PY@tok@ch\endcsname{\let\PY@it=\textit\def\PY@tc##1{\textcolor[rgb]{0.25,0.50,0.50}{##1}}}
\expandafter\def\csname PY@tok@cm\endcsname{\let\PY@it=\textit\def\PY@tc##1{\textcolor[rgb]{0.25,0.50,0.50}{##1}}}
\expandafter\def\csname PY@tok@cpf\endcsname{\let\PY@it=\textit\def\PY@tc##1{\textcolor[rgb]{0.25,0.50,0.50}{##1}}}
\expandafter\def\csname PY@tok@c1\endcsname{\let\PY@it=\textit\def\PY@tc##1{\textcolor[rgb]{0.25,0.50,0.50}{##1}}}
\expandafter\def\csname PY@tok@cs\endcsname{\let\PY@it=\textit\def\PY@tc##1{\textcolor[rgb]{0.25,0.50,0.50}{##1}}}

\def\PYZbs{\char`\\}
\def\PYZus{\char`\_}
\def\PYZob{\char`\{}
\def\PYZcb{\char`\}}
\def\PYZca{\char`\^}
\def\PYZam{\char`\&}
\def\PYZlt{\char`\<}
\def\PYZgt{\char`\>}
\def\PYZsh{\char`\#}
\def\PYZpc{\char`\%}
\def\PYZdl{\char`\$}
\def\PYZhy{\char`\-}
\def\PYZsq{\char`\'}
\def\PYZdq{\char`\"}
\def\PYZti{\char`\~}
% for compatibility with earlier versions
\def\PYZat{@}
\def\PYZlb{[}
\def\PYZrb{]}
\makeatother


    % For linebreaks inside Verbatim environment from package fancyvrb. 
    \makeatletter
        \newbox\Wrappedcontinuationbox 
        \newbox\Wrappedvisiblespacebox 
        \newcommand*\Wrappedvisiblespace {\textcolor{red}{\textvisiblespace}} 
        \newcommand*\Wrappedcontinuationsymbol {\textcolor{red}{\llap{\tiny$\m@th\hookrightarrow$}}} 
        \newcommand*\Wrappedcontinuationindent {3ex } 
        \newcommand*\Wrappedafterbreak {\kern\Wrappedcontinuationindent\copy\Wrappedcontinuationbox} 
        % Take advantage of the already applied Pygments mark-up to insert 
        % potential linebreaks for TeX processing. 
        %        {, <, #, %, $, ' and ": go to next line. 
        %        _, }, ^, &, >, - and ~: stay at end of broken line. 
        % Use of \textquotesingle for straight quote. 
        \newcommand*\Wrappedbreaksatspecials {% 
            \def\PYGZus{\discretionary{\char`\_}{\Wrappedafterbreak}{\char`\_}}% 
            \def\PYGZob{\discretionary{}{\Wrappedafterbreak\char`\{}{\char`\{}}% 
            \def\PYGZcb{\discretionary{\char`\}}{\Wrappedafterbreak}{\char`\}}}% 
            \def\PYGZca{\discretionary{\char`\^}{\Wrappedafterbreak}{\char`\^}}% 
            \def\PYGZam{\discretionary{\char`\&}{\Wrappedafterbreak}{\char`\&}}% 
            \def\PYGZlt{\discretionary{}{\Wrappedafterbreak\char`\<}{\char`\<}}% 
            \def\PYGZgt{\discretionary{\char`\>}{\Wrappedafterbreak}{\char`\>}}% 
            \def\PYGZsh{\discretionary{}{\Wrappedafterbreak\char`\#}{\char`\#}}% 
            \def\PYGZpc{\discretionary{}{\Wrappedafterbreak\char`\%}{\char`\%}}% 
            \def\PYGZdl{\discretionary{}{\Wrappedafterbreak\char`\$}{\char`\$}}% 
            \def\PYGZhy{\discretionary{\char`\-}{\Wrappedafterbreak}{\char`\-}}% 
            \def\PYGZsq{\discretionary{}{\Wrappedafterbreak\textquotesingle}{\textquotesingle}}% 
            \def\PYGZdq{\discretionary{}{\Wrappedafterbreak\char`\"}{\char`\"}}% 
            \def\PYGZti{\discretionary{\char`\~}{\Wrappedafterbreak}{\char`\~}}% 
        } 
        % Some characters . , ; ? ! / are not pygmentized. 
        % This macro makes them "active" and they will insert potential linebreaks 
        \newcommand*\Wrappedbreaksatpunct {% 
            \lccode`\~`\.\lowercase{\def~}{\discretionary{\hbox{\char`\.}}{\Wrappedafterbreak}{\hbox{\char`\.}}}% 
            \lccode`\~`\,\lowercase{\def~}{\discretionary{\hbox{\char`\,}}{\Wrappedafterbreak}{\hbox{\char`\,}}}% 
            \lccode`\~`\;\lowercase{\def~}{\discretionary{\hbox{\char`\;}}{\Wrappedafterbreak}{\hbox{\char`\;}}}% 
            \lccode`\~`\:\lowercase{\def~}{\discretionary{\hbox{\char`\:}}{\Wrappedafterbreak}{\hbox{\char`\:}}}% 
            \lccode`\~`\?\lowercase{\def~}{\discretionary{\hbox{\char`\?}}{\Wrappedafterbreak}{\hbox{\char`\?}}}% 
            \lccode`\~`\!\lowercase{\def~}{\discretionary{\hbox{\char`\!}}{\Wrappedafterbreak}{\hbox{\char`\!}}}% 
            \lccode`\~`\/\lowercase{\def~}{\discretionary{\hbox{\char`\/}}{\Wrappedafterbreak}{\hbox{\char`\/}}}% 
            \catcode`\.\active
            \catcode`\,\active 
            \catcode`\;\active
            \catcode`\:\active
            \catcode`\?\active
            \catcode`\!\active
            \catcode`\/\active 
            \lccode`\~`\~ 	
        }
    \makeatother

    \let\OriginalVerbatim=\Verbatim
    \makeatletter
    \renewcommand{\Verbatim}[1][1]{%
        %\parskip\z@skip
        \sbox\Wrappedcontinuationbox {\Wrappedcontinuationsymbol}%
        \sbox\Wrappedvisiblespacebox {\FV@SetupFont\Wrappedvisiblespace}%
        \def\FancyVerbFormatLine ##1{\hsize\linewidth
            \vtop{\raggedright\hyphenpenalty\z@\exhyphenpenalty\z@
                \doublehyphendemerits\z@\finalhyphendemerits\z@
                \strut ##1\strut}%
        }%
        % If the linebreak is at a space, the latter will be displayed as visible
        % space at end of first line, and a continuation symbol starts next line.
        % Stretch/shrink are however usually zero for typewriter font.
        \def\FV@Space {%
            \nobreak\hskip\z@ plus\fontdimen3\font minus\fontdimen4\font
            \discretionary{\copy\Wrappedvisiblespacebox}{\Wrappedafterbreak}
            {\kern\fontdimen2\font}%
        }%
        
        % Allow breaks at special characters using \PYG... macros.
        \Wrappedbreaksatspecials
        % Breaks at punctuation characters . , ; ? ! and / need catcode=\active 	
        \OriginalVerbatim[#1,codes*=\Wrappedbreaksatpunct]%
    }
    \makeatother

    % Exact colors from NB
    \definecolor{incolor}{HTML}{303F9F}
    \definecolor{outcolor}{HTML}{D84315}
    \definecolor{cellborder}{HTML}{CFCFCF}
    \definecolor{cellbackground}{HTML}{F7F7F7}
    
    % prompt
    \makeatletter
    \newcommand{\boxspacing}{\kern\kvtcb@left@rule\kern\kvtcb@boxsep}
    \makeatother
    \newcommand{\prompt}[4]{
        \ttfamily\llap{{\color{#2}[#3]:\hspace{3pt}#4}}\vspace{-\baselineskip}
    }
    

    
\setlength\headheight{30pt}
\setcounter{secnumdepth}{0} % Turns off numbering for sections

    % Prevent overflowing lines due to hard-to-break entities
    \sloppy 
    % Setup hyperref package
    \hypersetup{
      breaklinks=true,  % so long urls are correctly broken across lines
      colorlinks=true,
      urlcolor=urlcolor,
      linkcolor=linkcolor,
      citecolor=citecolor,
      }
    % Slightly bigger margins than the latex defaults
    \geometry{a4paper,tmargin=3cm,bmargin=2cm,lmargin=1cm,rmargin=1cm}\fancyhead[L]{Thème à définir}\fancyhead[L]{\adjustimage{height=1cm, valign=m}{/home/bouscadilla/Documents/Code/nbconvert/template/latex/pdf_solution/papier_eleve_ico_langage}\ttfamily\scshape Langage}\fancyhead[C]{\bfseries\MakeUppercase{4-1---MiseAuPoint}}\fancyhead[C]{\bfseries\MakeUppercase{4 --- Mise au point de programmes}}\fancyhead[R]{\monthyeardate\today}

    \fancyfoot[C]{\thepage}
    % #TODO ajouter les pages totales

    \pagestyle{fancy}
    


\begin{document}
    
    \title{4 --- Mise au point de programmes}
% \maketitle

    
    

    
    \hypertarget{chap.-4-mise-au-point-de-programmes-pa.dilla.fr1c}{%
\section{\texorpdfstring{Chap. 4 --- Mise au point de programmes
(\href{https://pa.dilla.fr/1c}{pa.dilla.fr/1c}
)}{Chap. 4 --- Mise au point de programmes (pa.dilla.fr/1c )}}\label{chap.-4-mise-au-point-de-programmes-pa.dilla.fr1c}}

    \hypertarget{types}{%
\subsection{4.1 --- Types}\label{types}}

    4.1.1 --- Les types en Python

    Chaque valeur manipulée par un programme Python est associée à un
\textbf{type}, qui caractérise la nature de cette valeur.
\begin{eleve}
    La fonction \texttt{type} permet d'obtenir le type de la valeur passée
en paramètre. \textbf{Utilise} cette fonction pour \textbf{déterminer}
le type de
\texttt{1,\ 3.14,\ True,\ "abc",\ None,\ (1,\ 2),\ {[}1,\ 2,\ 3{]},\ \{1,\ 2,\ 3\},\ \{\textquotesingle{}a\textquotesingle{}:\ 1,\ \textquotesingle{}b\textquotesingle{}:\ 2\}}.

\begin{enumerate}
\def\labelenumi{\arabic{enumi}.}
\setcounter{enumi}{1}
\tightlist
\item
  \textbf{Détermine} le type de \texttt{\{\}}, d'une fonction et d'une
  classe.
\end{enumerate}
        
        \end{eleve}\begin{reponse}
    \begin{longtable}[]{@{}lll@{}}
\toprule
\endhead
\textbf{valeur} & \textbf{type} & \textbf{description}\tabularnewline
\texttt{1} & \texttt{int} & nombres entiers\tabularnewline
\texttt{3.14} & \texttt{float} & nombres décimaux\tabularnewline
\texttt{True} & \texttt{bool} & booléens\tabularnewline
\texttt{"abc"} & \texttt{str} & chaînes de caractères\tabularnewline
\texttt{None} & \texttt{NoneType} & valeur indéfinie\tabularnewline
- & - & -\tabularnewline
\texttt{(1,\ 2)} & \texttt{tuple} & n-uplets\tabularnewline
\texttt{{[}1,\ 2,\ 3{]}} & \texttt{list} & tableaux\tabularnewline
\texttt{\{1,\ 2,\ 3\}} & \texttt{set} & ensembles\tabularnewline
\texttt{\{\textquotesingle{}a\textquotesingle{}:\ 1,\ \textquotesingle{}b\textquotesingle{}:\ 2\}}
& \texttt{dict} & dictionnaires\tabularnewline
\bottomrule
\end{longtable}

        \end{reponse}
    En Python, la gestion des types est qualifiée de \textbf{dynamique} :
c'est au moment de l'exécution du programme, lors de l'interprétation de
chaque opération de base, que l'interprète Python vérifie la concordance
entre les opérations et les types des valeurs utilisées.

    4.1.2 --- Annoter les variables et les fonctions

    Il est \textbf{indispensable} lors de la définition d'une fonction
d'avoir en tête les types attendus pour

\begin{itemize}
\tightlist
\item
  les paramètres et
\item
  l'éventuel type du résultat.
\end{itemize}

Pour la définition d'une interface, cette information est cruciale et
permet d'éviter autant que possible la mauvaise utilisation d'un module.
\begin{retenir}
    Python accepte l'annotation des \textbf{variables} et des
\textbf{fonctions}.

        \end{retenir}\begin{remarque}
    Ces annotations sont facultatives dans ce langage mais obligatoires dans
d'autres. Elles ont pour rôle :

\begin{itemize}
\tightlist
\item
  de documenter le code (utile pour toute relecture)
\item
  de permettre une vérification \emph{statique} (avant l'exécution) des
  types par des programmes externes.
\end{itemize}

        \end{remarque}\begin{exemple}
        {\scriptsize
    \begin{tcolorbox}[breakable, size=fbox, boxrule=1pt, pad at break*=1mm,colback=cellbackground, colframe=cellborder]
\prompt{In}{incolor}{ }{\boxspacing}
\begin{Verbatim}[commandchars=\\\{\}]
\PY{c+c1}{\PYZsh{} Annotation des variables}

\PY{n}{x}\PY{p}{:} \PY{n+nb}{int} \PY{o}{=} \PY{l+m+mi}{42}
\end{Verbatim}
\end{tcolorbox}
    }

        \end{exemple}\begin{exemple}
        {\scriptsize
    \begin{tcolorbox}[breakable, size=fbox, boxrule=1pt, pad at break*=1mm,colback=cellbackground, colframe=cellborder]
\prompt{In}{incolor}{ }{\boxspacing}
\begin{Verbatim}[commandchars=\\\{\}]
\PY{c+c1}{\PYZsh{} Annotation des fonctions}

\PY{k}{def} \PY{n+nf}{contient\PYZus{}doublon}\PY{p}{(}\PY{n}{t}\PY{p}{:} \PY{n+nb}{list}\PY{p}{)} \PY{o}{\PYZhy{}}\PY{o}{\PYZgt{}} \PY{n+nb}{bool}\PY{p}{:}
    \PY{c+c1}{\PYZsh{} annotation du paramètre : tableau (list)}
    \PY{c+c1}{\PYZsh{}         et de la sortie : valeur booléenne (bool)}
    \PY{k}{pass}

\PY{k}{def} \PY{n+nf}{cree} \PY{p}{(}\PY{p}{)} \PY{o}{\PYZhy{}}\PY{o}{\PYZgt{}} \PY{n+nb}{list} \PY{p}{:}
    \PY{c+c1}{\PYZsh{} annotation de la sortie : tableau (list)}
    \PY{k}{pass}

\PY{k}{def} \PY{n+nf}{contient}\PY{p}{(}\PY{n}{s}\PY{p}{:} \PY{n+nb}{list}\PY{p}{,} \PY{n}{x}\PY{p}{:} \PY{n+nb}{int}\PY{p}{)} \PY{o}{\PYZhy{}}\PY{o}{\PYZgt{}} \PY{n+nb}{bool}\PY{p}{:}
    \PY{c+c1}{\PYZsh{} annotation des paramètres : tableau (list)}
    \PY{c+c1}{\PYZsh{}                             nb entier (int)}
    \PY{c+c1}{\PYZsh{}           et de la sortie : valeur booléenne (bool)}
    \PY{k}{pass}

\PY{k}{def} \PY{n+nf}{ajoute}\PY{p}{(}\PY{n}{s}\PY{p}{:} \PY{n+nb}{list}\PY{p}{,} \PY{n}{x}\PY{p}{:} \PY{n+nb}{int}\PY{p}{)} \PY{o}{\PYZhy{}}\PY{o}{\PYZgt{}} \PY{k+kc}{None}\PY{p}{:}
    \PY{c+c1}{\PYZsh{} annotation des paramètres      : tableau (list)}
    \PY{c+c1}{\PYZsh{}                                  nb entier (int)}
    \PY{c+c1}{\PYZsh{} et explicitement aucune sortie : indéfinie (None)}
    \PY{k}{pass}
\end{Verbatim}
\end{tcolorbox}
    }

        \end{exemple}
    4.1.3 --- Types nommés et types paramétrés
\begin{remarque}
    En Python, les informations de types pour les \textbf{valeurs
structurées} (n-uplets, tableaux, dictionnaires, etc.) restent
\textbf{très superficielles}.

Un couple d'entier (comme \texttt{(1,\ 2)}) et un triplet mixte (comme
\texttt{(1,\ "abc",\ False)}) sont tous les deux le type \texttt{tuple}
alors qu'ils n'ont quand même rien à voir\ldots{}

        \end{remarque}
    Pour préciser les types de ces valeurs structurées, il faut utiliser le
module \texttt{typing}. Il définit de nouvelles versions des types de
base : les types \texttt{Tuple,\ List,\ Set,\ Dict}.

Ces nouvelles versions acceptent un ou plusieurs paramètres en fonctions
du ou des types de leurs composants.

\begin{longtable}[]{@{}ll@{}}
\toprule
\begin{minipage}[b]{0.47\columnwidth}\raggedright
\textbf{type}\strut
\end{minipage} & \begin{minipage}[b]{0.47\columnwidth}\raggedright
\textbf{description}\strut
\end{minipage}\tabularnewline
\midrule
\endhead
\begin{minipage}[t]{0.47\columnwidth}\raggedright
\texttt{Tuple{[}int,\ bool{]}}\strut
\end{minipage} & \begin{minipage}[t]{0.47\columnwidth}\raggedright
couple d'un entier et d'un booléen\strut
\end{minipage}\tabularnewline
\begin{minipage}[t]{0.47\columnwidth}\raggedright
\texttt{List{[}int{]}}\strut
\end{minipage} & \begin{minipage}[t]{0.47\columnwidth}\raggedright
tableau d'entiers\strut
\end{minipage}\tabularnewline
\begin{minipage}[t]{0.47\columnwidth}\raggedright
\texttt{Set{[}str{]}}\strut
\end{minipage} & \begin{minipage}[t]{0.47\columnwidth}\raggedright
ensemble de chaînes de caractères\strut
\end{minipage}\tabularnewline
\begin{minipage}[t]{0.47\columnwidth}\raggedright
\texttt{Dict{[}str,\ int{]}}\strut
\end{minipage} & \begin{minipage}[t]{0.47\columnwidth}\raggedright
dictionnaire dont les clés sont des chaînes de caractères et les valeurs
des entiers\strut
\end{minipage}\tabularnewline
\bottomrule
\end{longtable}
\begin{exemple}
        {\scriptsize
    \begin{tcolorbox}[breakable, size=fbox, boxrule=1pt, pad at break*=1mm,colback=cellbackground, colframe=cellborder]
\prompt{In}{incolor}{ }{\boxspacing}
\begin{Verbatim}[commandchars=\\\{\}]
\PY{k+kn}{from} \PY{n+nn}{typing} \PY{k+kn}{import} \PY{n}{List}

\PY{k}{def} \PY{n+nf}{cree} \PY{p}{(}\PY{p}{)} \PY{o}{\PYZhy{}}\PY{o}{\PYZgt{}} \PY{n}{List}\PY{p}{[}\PY{n+nb}{int}\PY{p}{]} \PY{p}{:}
    \PY{c+c1}{\PYZsh{} annotation de la sortie : tableau d\PYZsq{}entiers (List[int])}
    \PY{k}{pass}
\end{Verbatim}
\end{tcolorbox}
    }

        \end{exemple}
    \hypertarget{tester-un-programme}{%
\subsection{4.2 --- Tester un programme}\label{tester-un-programme}}

    4.2.1 --- Tester la correction d'une fonction

    Pour vérifier qu'une fonction fait bien ce qu'elle est sensée faire il
faut effectuer des tests.
\begin{eleve}
    \textbf{Implémenter} la classe \texttt{Intervalle} définissant
l'intervalle \texttt{a..b} (noté aussi \([a ; b]\)).

\begin{enumerate}
\def\labelenumi{\arabic{enumi}.}
\setcounter{enumi}{1}
\tightlist
\item
  \textbf{Ajouter} une méthode \texttt{est\_vide()} vérifiant si
  l'intervalle est vide (un intervalle tel que \(b \leq a\) est
  considéré comme vide).
\item
  \textbf{Vérifier} que la méthode \texttt{est\_vide()} est correcte.
\end{enumerate}
        
        \end{eleve}\begin{reponse}
    Définissons la classe \texttt{Intervalle} qui définit un intervalle
d'extrémité \texttt{self.a} et \texttt{self.b} :

\begin{Shaded}
\begin{Highlighting}[]
\KeywordTok{class}\NormalTok{ Intervalle:}
    \KeywordTok{def} \FunctionTok{\_\_init\_\_}\NormalTok{(}\VariableTok{self}\NormalTok{, debut, fin):}
        \CommentTok{"""Intervalle d\textquotesingle{}extrémité [a ; b]"""}
        \VariableTok{self}\NormalTok{.a }\OperatorTok{=}\NormalTok{ debut}
        \VariableTok{self}\NormalTok{.b }\OperatorTok{=}\NormalTok{ fin}
\NormalTok{    ...}
\end{Highlighting}
\end{Shaded}

et une méthode \texttt{est\_vide} qui renvoie une valeur booléenne
associée au prédicat \emph{l'intervalle est vide} :

\begin{Shaded}
\begin{Highlighting}[]
\NormalTok{    ...}
    \KeywordTok{def}\NormalTok{ est\_vide(}\VariableTok{self}\NormalTok{):}
        \CommentTok{"""Est ce que l\textquotesingle{}intervalle est vide?"""}
        \ControlFlowTok{return} \VariableTok{self}\NormalTok{.b }\OperatorTok{\textless{}} \VariableTok{self}\NormalTok{.a}
\end{Highlighting}
\end{Shaded}

Pour tester la fonction \texttt{est\_vide} on pourra vérifier que
l'exécution du programme ci-dessous affiche bien \texttt{False} puis
\texttt{True}.

\begin{Shaded}
\begin{Highlighting}[]
\CommentTok{\# premier test}
\NormalTok{mon\_inter }\OperatorTok{=}\NormalTok{ Intervalle(}\DecValTok{5}\NormalTok{, }\DecValTok{12}\NormalTok{)}
\BuiltInTok{print}\NormalTok{( mon\_inter.est\_vide() )}

\CommentTok{\# deuxième test (écrit en une ligne)}
\BuiltInTok{print}\NormalTok{( Intervalle(}\DecValTok{5}\NormalTok{,}\DecValTok{3}\NormalTok{).est\_vide() )}
\end{Highlighting}
\end{Shaded}

        \end{reponse}\begin{retenir}
    Le module \texttt{doctest} propose une façon pratique d'intégrer les
tests et les résultats attendus \textbf{directement dans la méthode} (ou
la fonction) concernée. Un appel à la fonction \texttt{testmod()}
effectue l'ensemble des tests et vérifie si le résultats escompté est
affiché. La synthèse des tests effectués est affichée dans l'interprète
Python.

Pour utiliser cet outil, il faut :

\begin{enumerate}
\def\labelenumi{\arabic{enumi}.}
\tightlist
\item
  importer la fonction \texttt{testmod} du module \texttt{doctest}
\item
  modifier la documentation des fonctions et méthodes
\item
  exécuter la fonction \texttt{testmod()}
\item
  étudier l'interprète pour vérifier la bonne exécution des tests.
\end{enumerate}

        \end{retenir}\begin{exemple}
    Ainsi pour la méthode \texttt{est\_vide()} de l'activité précédente, on
écrira :

\begin{Shaded}
\begin{Highlighting}[]
\ImportTok{from}\NormalTok{ doctest }\ImportTok{import}\NormalTok{ testmod}

\KeywordTok{class}\NormalTok{ Intervalle}
\NormalTok{    ...}
    \KeywordTok{def}\NormalTok{ est\_vide(}\VariableTok{self}\NormalTok{):}
        \CommentTok{"""Est ce que l\textquotesingle{}intervalle est vide?}
\CommentTok{        \textgreater{}\textgreater{}\textgreater{} mon\_inter = Intervalle(5,12)}
\CommentTok{        \textgreater{}\textgreater{}\textgreater{} mon\_inter.est\_vide()}
\CommentTok{        False}
\CommentTok{        \textgreater{}\textgreater{}\textgreater{} Intervalle(5, 3).est\_vide()}
\CommentTok{        True}
\CommentTok{        """}
        \ControlFlowTok{return} \VariableTok{self}\NormalTok{.b }\OperatorTok{\textless{}} \VariableTok{self}\NormalTok{.a}

\NormalTok{testmod()}
\end{Highlighting}
\end{Shaded}

On peut remarquer que :

\begin{itemize}
\tightlist
\item
  les tests sont écrits directement dans la documentation de la fonction
  (docstring)
\item
  les instructions à interpréter sont précédées de trois chevrons
  \texttt{\textgreater{}\textgreater{}\textgreater{}}
\item
  les résultats attendus écrits directement
\item
  il suffit d'écrire l'instruction \texttt{testmod()} pour lancer les
  tests.
\end{itemize}

        \end{exemple}
    4.2.1 --- Tester la correction d'un programme
\begin{eleve}
    \textbf{Écrire} une fonction \texttt{tri} qui trie un tableau d'entiers,
en place, par ordre croissant.

On cherche maintenant à tester la fonction \texttt{tri}.

\begin{enumerate}
\def\labelenumi{\arabic{enumi}.}
\setcounter{enumi}{1}
\tightlist
\item
  \textbf{Proposer} une fonction \texttt{test()} qui prend en argument
  un tableau \texttt{t}, appelle la fonction \texttt{tri} sur ce tableau
  puis vérifie que le tableau \texttt{t} est bien trié.
\end{enumerate}

\begin{itemize}
\tightlist
\item
  implémenter une fonction de test naïve
\item
  vérifier que le tableau avant tri et après tri contient les mêmes
  éléments, et pour chaque élément le même nombre d'occurrence.
\end{itemize}
        
        \end{eleve}
        {\scriptsize
    \begin{tcolorbox}[breakable, size=fbox, boxrule=1pt, pad at break*=1mm,colback=cellbackground, colframe=cellborder]
\prompt{In}{incolor}{ }{\boxspacing}
\begin{Verbatim}[commandchars=\\\{\}]
\PY{c+c1}{\PYZsh{} exemple de fonction de tri qui doit échouer :}
\PY{k+kn}{from} \PY{n+nn}{typing} \PY{k+kn}{import} \PY{n}{List}

\PY{k}{def} \PY{n+nf}{tri}\PY{p}{(}\PY{n}{t}\PY{p}{:} \PY{n}{List}\PY{p}{[}\PY{n+nb}{int}\PY{p}{]}\PY{p}{)} \PY{o}{\PYZhy{}}\PY{o}{\PYZgt{}} \PY{k+kc}{None}\PY{p}{:}
    \PY{l+s+sd}{\PYZdq{}\PYZdq{}\PYZdq{}Efface tout (et donc c\PYZsq{}est trié!)\PYZdq{}\PYZdq{}\PYZdq{}}
    \PY{n}{t}\PY{o}{.}\PY{n}{clear}\PY{p}{(}\PY{p}{)}


\PY{k}{def} \PY{n+nf}{tri}\PY{p}{(}\PY{n}{t}\PY{p}{:} \PY{n}{List}\PY{p}{[}\PY{n+nb}{int}\PY{p}{]}\PY{p}{)} \PY{o}{\PYZhy{}}\PY{o}{\PYZgt{}} \PY{k+kc}{None}\PY{p}{:}
    \PY{l+s+sd}{\PYZdq{}\PYZdq{}\PYZdq{}Supprime les doublons mais trie\PYZdq{}\PYZdq{}\PYZdq{}}
    \PY{n}{tab} \PY{o}{=} \PY{p}{[}\PY{p}{]}
    \PY{k}{for} \PY{n}{x} \PY{o+ow}{in} \PY{n}{t}\PY{p}{:}
        \PY{k}{if} \PY{n}{x} \PY{o+ow}{not} \PY{o+ow}{in} \PY{n}{tab}\PY{p}{:}
            \PY{n}{tab}\PY{o}{.}\PY{n}{append}\PY{p}{(}\PY{n}{x}\PY{p}{)}
    \PY{n}{tab}\PY{o}{.}\PY{n}{sort}\PY{p}{(}\PY{p}{)}
    \PY{n}{t}\PY{o}{.}\PY{n}{clear}\PY{p}{(}\PY{p}{)}
    \PY{k}{for} \PY{n}{x} \PY{o+ow}{in} \PY{n}{tab}\PY{p}{:}
        \PY{n}{t}\PY{o}{.}\PY{n}{append}\PY{p}{(}\PY{n}{x}\PY{p}{)}
\end{Verbatim}
\end{tcolorbox}
    }
\begin{reponse}
    

        \end{reponse}
        {\scriptsize
    \begin{tcolorbox}[breakable, size=fbox, boxrule=1pt, pad at break*=1mm,colback=cellbackground, colframe=cellborder]
\prompt{In}{incolor}{ }{\boxspacing}
\begin{Verbatim}[commandchars=\\\{\}]
\PY{k}{def} \PY{n+nf}{occurences}\PY{p}{(}\PY{n}{t}\PY{p}{)}\PY{p}{:}
    \PY{l+s+sd}{\PYZdq{}\PYZdq{}\PYZdq{}renvoie le dictionnaire des occurences de t}

\PY{l+s+sd}{    Args:}
\PY{l+s+sd}{        t (list): tableau en entrée}
\PY{l+s+sd}{    \PYZdq{}\PYZdq{}\PYZdq{}}
    \PY{n}{d} \PY{o}{=} \PY{p}{\PYZob{}}\PY{p}{\PYZcb{}}
    \PY{k}{for} \PY{n}{x} \PY{o+ow}{in} \PY{n}{t}\PY{p}{:}
        \PY{k}{if} \PY{n}{x} \PY{o+ow}{in} \PY{n}{d}\PY{p}{:}
            \PY{n}{d}\PY{p}{[}\PY{n}{x}\PY{p}{]} \PY{o}{+}\PY{o}{=} \PY{l+m+mi}{1}
        \PY{k}{else}\PY{p}{:}
            \PY{n}{d}\PY{p}{[}\PY{n}{x}\PY{p}{]}  \PY{o}{=} \PY{l+m+mi}{1}
    \PY{k}{return} \PY{n}{d}

\PY{k}{def} \PY{n+nf}{identiques}\PY{p}{(}\PY{n}{d1}\PY{p}{,} \PY{n}{d2}\PY{p}{)}\PY{p}{:}
    \PY{l+s+sd}{\PYZdq{}\PYZdq{}\PYZdq{}deux dictionnaires identiques}

\PY{l+s+sd}{    Args:}
\PY{l+s+sd}{        d1 (dict)}
\PY{l+s+sd}{        d2 (dict)}
\PY{l+s+sd}{    \PYZdq{}\PYZdq{}\PYZdq{}}
    \PY{k}{for} \PY{n}{x} \PY{o+ow}{in} \PY{n}{d1}\PY{p}{:}
        \PY{k}{assert} \PY{n}{x} \PY{o+ow}{in} \PY{n}{d2}
        \PY{k}{assert} \PY{n}{d1}\PY{p}{[}\PY{n}{x}\PY{p}{]} \PY{o}{==} \PY{n}{d2}\PY{p}{[}\PY{n}{x}\PY{p}{]}
    \PY{k}{for} \PY{n}{x} \PY{o+ow}{in} \PY{n}{d2}\PY{p}{:}
        \PY{k}{assert} \PY{n}{x} \PY{o+ow}{in} \PY{n}{d1}
        \PY{k}{assert} \PY{n}{d2}\PY{p}{[}\PY{n}{x}\PY{p}{]} \PY{o}{==} \PY{n}{d1}\PY{p}{[}\PY{n}{x}\PY{p}{]}

\PY{k}{def} \PY{n+nf}{test}\PY{p}{(}\PY{n}{t}\PY{p}{)}\PY{p}{:}
    \PY{l+s+sd}{\PYZdq{}\PYZdq{}\PYZdq{}teste la fonction tri sur le tableau t}

\PY{l+s+sd}{    Args:}
\PY{l+s+sd}{        t (list): tableau à tester}
\PY{l+s+sd}{    \PYZdq{}\PYZdq{}\PYZdq{}}
    \PY{n}{occ} \PY{o}{=} \PY{n}{occurences}\PY{p}{(}\PY{n}{t}\PY{p}{)}
    \PY{n}{tri}\PY{p}{(}\PY{n}{t}\PY{p}{)}
    \PY{k}{for} \PY{n}{i} \PY{o+ow}{in} \PY{n+nb}{range}\PY{p}{(}\PY{l+m+mi}{0}\PY{p}{,} \PY{n+nb}{len}\PY{p}{(}\PY{n}{t}\PY{p}{)} \PY{o}{\PYZhy{}} \PY{l+m+mi}{1}\PY{p}{)}\PY{p}{:}
        \PY{k}{assert} \PY{n}{t}\PY{p}{[}\PY{n}{i}\PY{p}{]} \PY{o}{\PYZlt{}}\PY{o}{=} \PY{n}{t}\PY{p}{[}\PY{n}{i}\PY{o}{+}\PY{l+m+mi}{1}\PY{p}{]}
    \PY{n}{identiques}\PY{p}{(}\PY{n}{occ}\PY{p}{,} \PY{n}{occurences}\PY{p}{(}\PY{n}{t}\PY{p}{)}\PY{p}{)}
\end{Verbatim}
\end{tcolorbox}
    }
\begin{eleve}
    Maintenant que la fonction test est correcte, on peut passer à des tests
un peu plus ambitieux.

\begin{enumerate}
\def\labelenumi{\arabic{enumi}.}
\tightlist
\item
  À l'aide de la fonction \texttt{randint} de la bibliothèque
  \texttt{random}, crée une fonction
  \texttt{tableau\_aleatoire(n:\ int,\ a:\ int,\ b:int)\ -\textgreater{}\ List{[}int{]}}
  qui renvoie un tableau de \texttt{n} éléments pris aléatoirement dans
  l'intervalle \texttt{a..b}.
\item
  Utilise la fonction précédente pour effectuer 100 tests effectués sur
  des tableaux de différentes tailles et dont les valeurs sont prises
  dans des intervalles variables.
\end{enumerate}
        
        \end{eleve}\begin{reponse}
    

        \end{reponse}
        {\scriptsize
    \begin{tcolorbox}[breakable, size=fbox, boxrule=1pt, pad at break*=1mm,colback=cellbackground, colframe=cellborder]
\prompt{In}{incolor}{ }{\boxspacing}
\begin{Verbatim}[commandchars=\\\{\}]
\PY{k+kn}{from} \PY{n+nn}{random} \PY{k+kn}{import} \PY{n}{randint}
\PY{k+kn}{from} \PY{n+nn}{typing} \PY{k+kn}{import} \PY{n}{List}


\PY{k}{def} \PY{n+nf}{tri}\PY{p}{(}\PY{n}{t}\PY{p}{)}\PY{p}{:}
    \PY{l+s+sd}{\PYZdq{}\PYZdq{}\PYZdq{}fonction de tri correcte\PYZdq{}\PYZdq{}\PYZdq{}}
    \PY{n}{t}\PY{o}{.}\PY{n}{sort}\PY{p}{(}\PY{p}{)}


\PY{k}{def} \PY{n+nf}{tableau\PYZus{}aleatoire}\PY{p}{(}\PY{n}{n}\PY{p}{:} \PY{n+nb}{int}\PY{p}{,} \PY{n}{a}\PY{p}{:} \PY{n+nb}{int}\PY{p}{,} \PY{n}{b}\PY{p}{:} \PY{n+nb}{int}\PY{p}{)} \PY{o}{\PYZhy{}}\PY{o}{\PYZgt{}} \PY{n}{List}\PY{p}{[}\PY{n+nb}{int}\PY{p}{]}\PY{p}{:}
    \PY{k}{return} \PY{p}{[}\PY{n}{randint}\PY{p}{(}\PY{n}{a}\PY{p}{,}\PY{n}{b}\PY{p}{)} \PY{k}{for} \PY{n}{\PYZus{}} \PY{o+ow}{in} \PY{n+nb}{range}\PY{p}{(}\PY{n}{n}\PY{p}{)}\PY{p}{]}


\PY{k}{for} \PY{n}{n} \PY{o+ow}{in} \PY{n+nb}{range}\PY{p}{(}\PY{l+m+mi}{100}\PY{p}{)}\PY{p}{:}
    \PY{c+c1}{\PYZsh{} [0,0,...,0]}
    \PY{n}{test}\PY{p}{(}\PY{n}{tableau\PYZus{}aleatoire}\PY{p}{(}\PY{n}{n}\PY{p}{,}\PY{l+m+mi}{0}\PY{p}{,}\PY{l+m+mi}{0}\PY{p}{)}\PY{p}{)}
    \PY{c+c1}{\PYZsh{} tableau avec bcp de doublons}
    \PY{n}{test}\PY{p}{(}\PY{n}{tableau\PYZus{}aleatoire}\PY{p}{(}\PY{n}{n}\PY{p}{,} \PY{o}{\PYZhy{}}\PY{n}{n}\PY{o}{/}\PY{o}{/}\PY{l+m+mi}{4}\PY{p}{,} \PY{n}{n}\PY{o}{/}\PY{o}{/}\PY{l+m+mi}{4}\PY{p}{)}\PY{p}{)}
    \PY{c+c1}{\PYZsh{} tableau de grande amplitude}
    \PY{n}{test}\PY{p}{(}\PY{n}{tableau\PYZus{}aleatoire}\PY{p}{(}\PY{n}{n}\PY{p}{,} \PY{o}{\PYZhy{}}\PY{l+m+mi}{10}\PY{o}{*}\PY{n}{n}\PY{p}{,} \PY{l+m+mi}{10}\PY{o}{*}\PY{l+m+mi}{10}\PY{p}{)}\PY{p}{)}
\end{Verbatim}
\end{tcolorbox}
    }

    4.2.2 --- Tester les performances
\begin{retenir}
    Après la correction, on souhaite le plus souvent vérifier les
\textbf{performances} de nos programmes.

La théorie permet de prédire les performances (ça s'appelle la
\textbf{complexité} et nous l'étudierons dans l'année ;).

        \end{retenir}
    Une façon simple et efficace est de mesurer le temps d'exécution d'un
programme. Pour cela, nous allons utiliser la fonction \texttt{time()}
de la bibliothèque \texttt{time} qui renvoie le nombre de secondes
écoulées depuis un instant de référence (1 janvier 1970 à minuit,
démarrage de l'ordinateur, etc.).

        {\scriptsize
    \begin{tcolorbox}[breakable, size=fbox, boxrule=1pt, pad at break*=1mm,colback=cellbackground, colframe=cellborder]
\prompt{In}{incolor}{ }{\boxspacing}
\begin{Verbatim}[commandchars=\\\{\}]
\PY{k+kn}{from} \PY{n+nn}{time} \PY{k+kn}{import} \PY{n}{time}

\PY{n+nb}{print}\PY{p}{(}\PY{n}{time}\PY{p}{(}\PY{p}{)}\PY{p}{)}
\PY{n+nb}{print}\PY{p}{(}\PY{n}{time}\PY{p}{(}\PY{p}{)}\PY{p}{)}
\end{Verbatim}
\end{tcolorbox}
    }
\begin{retenir}
    La valeur affichée ne nous intéresse pas, c'est la \textbf{différence}
entre deux valeurs qui nous indiquera la \textbf{durée} d'exécution !

        \end{retenir}\begin{exemple}
        {\scriptsize
    \begin{tcolorbox}[breakable, size=fbox, boxrule=1pt, pad at break*=1mm,colback=cellbackground, colframe=cellborder]
\prompt{In}{incolor}{ }{\boxspacing}
\begin{Verbatim}[commandchars=\\\{\}]
\PY{n}{tab} \PY{o}{=} \PY{n}{tableau\PYZus{}aleatoire}\PY{p}{(}\PY{l+m+mi}{10\PYZus{}000}\PY{p}{,} \PY{o}{\PYZhy{}}\PY{l+m+mi}{1\PYZus{}000}\PY{p}{,} \PY{l+m+mi}{1\PYZus{}000}\PY{p}{)}

\PY{n}{t0} \PY{o}{=} \PY{n}{time}\PY{p}{(}\PY{p}{)}
\PY{n}{tri}\PY{p}{(}\PY{n}{tab}\PY{p}{)}
\PY{n+nb}{print}\PY{p}{(}\PY{n}{time}\PY{p}{(}\PY{p}{)} \PY{o}{\PYZhy{}} \PY{n}{t0}\PY{p}{)}
\end{Verbatim}
\end{tcolorbox}
    }

        \end{exemple}\begin{retenir}
    Plutôt que de mesurer les performances d'un seul appel, il est
préférable d'essayer de faire varier les entrées, dans le but de relier
la taille de ces entrées avec la mesure du temps d'exécution.

        \end{retenir}\begin{exemple}
        {\scriptsize
    \begin{tcolorbox}[breakable, size=fbox, boxrule=1pt, pad at break*=1mm,colback=cellbackground, colframe=cellborder]
\prompt{In}{incolor}{ }{\boxspacing}
\begin{Verbatim}[commandchars=\\\{\}]
\PY{k}{for} \PY{n}{k} \PY{o+ow}{in} \PY{n+nb}{range}\PY{p}{(}\PY{l+m+mi}{10}\PY{p}{,} \PY{l+m+mi}{16}\PY{p}{)}\PY{p}{:}
    \PY{n}{n} \PY{o}{=} \PY{l+m+mi}{2} \PY{o}{*}\PY{o}{*} \PY{n}{k}
    \PY{n}{tab} \PY{o}{=} \PY{n}{tableau\PYZus{}aleatoire}\PY{p}{(}\PY{n}{n}\PY{p}{,} \PY{o}{\PYZhy{}}\PY{l+m+mi}{100}\PY{p}{,} \PY{l+m+mi}{100}\PY{p}{)}
    \PY{n}{t} \PY{o}{=} \PY{n}{time}\PY{p}{(}\PY{p}{)}
    \PY{n}{tri} \PY{p}{(}\PY{n}{tab}\PY{p}{)}
    \PY{n+nb}{print}\PY{p}{(}\PY{n}{n}\PY{p}{,} \PY{n}{time}\PY{p}{(}\PY{p}{)} \PY{o}{\PYZhy{}} \PY{n}{t}\PY{p}{)}
\end{Verbatim}
\end{tcolorbox}
    }

        \end{exemple}
    \hypertarget{invariants-de-structure}{%
\subsection{4.3 --- Invariants de
structure}\label{invariants-de-structure}}

    Il n'est pas rare que les attributs d'une classe satisfassent des
\textbf{invariants} (\emph{propriétés qui restent vraies tout au long de
l'exécution du programme}).
\begin{exemple}
    Voici quelques exemples possibles d'invariants de structure :

\begin{itemize}
\tightlist
\item
  un attribut représentant un mois de l'année a une valeur comprise
  entre 1 et 12 ;
\item
  un attribut contient un tableau d'entiers et représente le numéro de
  sécurité social. La taille du tableau doit être de 13 ;
\item
  un attribut contient une mesure d'angle qui doit être comprise entre 0
  et 360° ;
\item
  un attribut contient un tableau qui doit être trié en permanence ;
\item
  deux attributs \texttt{x} et \texttt{y} représentent une position sur
  une grille \(N \times N\) et ils doivent donc respecter les inégalités
  : \(0 \leq \texttt{x} < N\) et \(0 \leq \texttt{y} < N\)
\item
  etc.
\end{itemize}

        \end{exemple}\begin{remarque}
    \textbf{Concernant la programmation orientée objet.} Le principe
d'encapsulation de la programmation objet permet d'imaginer maintenir
ces invariants. Il suffit que le constructeur de la classe les
garantisse puis que les méthodes qui modifient les attributs
maintiennent ces invariants.

        \end{remarque}\begin{exemple}
    Programmation défensive pour le constructeur :

\begin{Shaded}
\begin{Highlighting}[]
\KeywordTok{class}\NormalTok{ C:}
    \KeywordTok{def} \FunctionTok{\_\_init\_\_}\NormalTok{(}\VariableTok{self}\NormalTok{, x, y):}
        \ControlFlowTok{if} \KeywordTok{not}\NormalTok{ (...invariant...):}
            \ControlFlowTok{raise} \PreprocessorTok{ValueError}\NormalTok{(}\StringTok{\textquotesingle{}...explication...\textquotesingle{}}\NormalTok{)}
        \VariableTok{self}\NormalTok{.x }\OperatorTok{=}\NormalTok{ x}
        \VariableTok{self}\NormalTok{.y }\OperatorTok{=}\NormalTok{ y}
\NormalTok{    ...}
\end{Highlighting}
\end{Shaded}

Programmation défensive pour les méthodes

\begin{Shaded}
\begin{Highlighting}[]
\NormalTok{    ...}
    \KeywordTok{def}\NormalTok{ deplace(}\VariableTok{self}\NormalTok{):}
        \ControlFlowTok{if}\NormalTok{ ...:}
            \VariableTok{self}\NormalTok{.x }\OperatorTok{+=} \DecValTok{1}
            \VariableTok{self}\NormalTok{.y }\OperatorTok{+=} \DecValTok{1}
        \ControlFlowTok{assert}\NormalTok{ ...invariant...}
\end{Highlighting}
\end{Shaded}

        \end{exemple}\begin{exemple}
    Lorsque la vérification d'un invariant commence à être complexe, on peut
déporter cette vérification dans une méthode spécifique.

\begin{Shaded}
\begin{Highlighting}[]
\KeywordTok{class}\NormalTok{ C:}
\NormalTok{    ...}

    \KeywordTok{def}\NormalTok{ valide(}\VariableTok{self}\NormalTok{):}
\NormalTok{        ...vérifie l}\StringTok{\textquotesingle{}invariant...}
\StringTok{        ...et lève une exception si besoin...}
\end{Highlighting}
\end{Shaded}

        \end{exemple}
    Afin d'aider à la mise au point des programmes, on peut annoter les
fonctions Python avec des \textbf{types}, qui décrivent la nature des
arguments et des résultats. Bien qu'il ne s'agisse là que d'une forme de
documentation supplémentaire, ignorée par l'interprète Python, des
outils externes permettent une \textbf{vérification statique} de ces
types, c'est-à-dire une vérification \textbf{avant} que le programme ne
soit exécuté. La mise au point des programmes passe également par une
phase de test rigoureuse, qui s'assure de la correction mais également
des performances. Le test de fonctions Python peut avantageusement se
contruire autour de l'instruction \texttt{assert}.


    % Add a bibliography block to the postdoc
    
    
    
\end{document}
