% template pour la conversion ipynb → pdf
% author : Pascal Padilla
% source : ???
%
% Deux types de modifications : cellule et page
%   * formatage des cellules : par exemple :
%     '..."metadata": {"tags": ["retenir"]},...'
%       * "cacher"
%       * "exo"
%       * "solution"
%       * "reponse"
%       * "proposition"
%       * "remarque"
%       * "exemple"
%       * "retenir"
%
%   * formatage de la page : theme + titre     
%     ...
%     "metadata": {
%       ...,
%       "latex_metadata": {
%       "theme": "machine",
%       "title": "Représentation des données"
%       },
%     Thèmes possibles : 
%       * "interface"
%       * "machine"
%       * "langage"
%       * "algo"
%       * "struct"
%       * "data"
%       * "reseausocial"
%       * "ds"
%       * "iot"
% 
%
\documentclass[a4paper,17pt]{extarticle}


    %unicode lualatex
\usepackage{fontspec}
\usepackage[sfdefault, condensed]{roboto} % police d'écriture plus moderne
\usepackage[french]{babel} % francisation
\usepackage[parfill]{parskip} %suppression indentation




\usepackage{fancyhdr}
\usepackage{multicol}

% figure non flotantes
\usepackage{float}
\let\origfigure\figure
\let\endorigfigure\endfigure
\renewenvironment{figure}[1][2] {
    \expandafter\origfigure\expandafter[H]
} {
    \endorigfigure
}

% mois/année
\usepackage{datetime}
\newdateformat{monthyeardate}{%
  \monthname[\THEMONTH] \THEYEAR}

% couleurs perso
\usepackage[table]{xcolor}
\definecolor{deepblue}{rgb}{0.3,0.3,0.8}
\definecolor{darkblue}{rgb}{0,0,0.3}
\definecolor{deepred}{rgb}{0.6,0,0}
\definecolor{iremred}{RGB}{204,35,50}
\definecolor{deepgreen}{rgb}{0,0.6,0}
\definecolor{backcolor}{rgb}{0.98,0.95,0.95}
\definecolor{grisClair}{rgb}{0.95,0.95,0.95}
\definecolor{orangeamu}{RGB}{250,178,11}
\definecolor{noiramu}{RGB}{35,31,32}
\definecolor{bleuamu}{RGB}{20,118,198}
\definecolor{bleuamudark}{RGB}{15,90,150}
\definecolor{cyanamu}{RGB}{77,198,244}


\usepackage{/home/bouscadilla/Documents/Code/nbconvert/template/latex/pdf_solution/xeboiboites}
%
% exemple
\newbreakabletheorem[
    small box style={fill=deepblue!90,draw=deepblue!15, rounded corners,line width=1pt},%
    big box style={fill=deepblue!5,draw=deepblue!15,thick,rounded corners,line width=1pt},%
    headfont={\color{white}\bfseries}
        ]{exemple}{Exemple}{}%{counterCo}
%
% remarque
\newbreakabletheorem[
    small box style={draw=ansi-green-intense!100,line width=2pt,fill=ansi-green-intense!0,rounded corners,decoration=penciline, decorate},%
	big box style={color=ansi-green-intense!90,fill=ansi-green-intense!10,thick,decoration={penciline},decorate},
    broken edges={draw=ansi-green-intense!90,thick,fill=orange!20!black!5, decoration={random steps, segment length=.5cm,amplitude=1.3mm},decorate},%
    other edges={decoration=penciline,decorate,thick},%
    headfont={\color{ansi-green-intense}\large\scshape\bfseries}
    ]{remarque}{Remarque}{}%{counterCa}
%
% formule (sans titre)
\newboxedequation[%
    big box style={fill=cyanamu!10,draw=cyanamu!100,thick,decoration=penciline,decorate}]%
    {form}
%
% Réponse
\newbreakabletheorem[
    small box style={fill=bleuamu!100, draw=bleuamu!60, line width=1pt,rounded corners,decorate},%
    big box style={fill=bleuamu!10,draw=bleuamu!30,thick,rounded corners,decorate},
    headfont={\color{white}\large\scshape\bfseries}
        ]{reponse}{Correction}{}
%

%
% À retenir
%\newbreakabletheorem[
%    small box style={fill=deepred!100, draw=deepred!80, line width=1pt,rounded corners,decorate},%
%    big box style={fill=deepred!10,draw=deepred!50,thick,rounded corners,decorate},
%    headfont={\color{white}\large\scshape\bfseries}
%        ]{retenir}{À retenir}{}
%
\newboxedequation[%
    big box style={fill=deepred!10,draw=deepred!0,thick,decoration=penciline,decorate}]%
    {retenir}



% astuce
\newspanning[
    image=/home/bouscadilla/Documents/Code/nbconvert/template/latex/pdf_solution/fig-idee,headfont=\bfseries,
    spanning style={very thick,decoration=penciline,decorate}
    ]{astuce}{Astuce}{}
%
% activité

\newcounter{counterCa}
\newbreakabletheorem[
    small box style={draw=orangeamu!100,line width=2pt,fill=orangeamu!100,rounded corners,decoration=penciline, decorate},%
	big box style={color=orangeamu!100,fill=orangeamu!5,thick,decoration={penciline},decorate},
    broken edges={draw=orangeamu!100,thick,fill=orangeamu!100, decoration={random steps, segment length=.5cm,amplitude=1.3mm},decorate},%
    other edges={decoration=penciline,decorate,thick},%
    headfont={\color{white}\large\scshape\bfseries}
    ]{activite}{\adjustimage{height=1cm, valign=m}{/home/bouscadilla/Documents/Code/nbconvert/template/latex/pdf_solution/papier_eleve_investigation.png}%
    Activité}{counterCa}
%   
%   environnement élève
%
\newenvironment{eleve}%
%{\begin{activite}\large\\} % écrire plus gros
{\begin{activite}\color{noiramu}\\[-0.5cm]}
{\end{activite}}

\newenvironment{formule}%
%{\begin{activite}\large\\} % écrire plus gros
{\begin{form}\color{bleuamu}}
{\end{form}}


\usepackage[breakable]{tcolorbox}
    \usepackage{parskip} % Stop auto-indenting (to mimic markdown behaviour)
    
    \usepackage{iftex}
    \ifPDFTeX
    	\usepackage[T1]{fontenc}
    	\usepackage{mathpazo}
    \else
    	\usepackage{fontspec}
    \fi

    % Basic figure setup, for now with no caption control since it's done
    % automatically by Pandoc (which extracts ![](path) syntax from Markdown).
    \usepackage{graphicx}
    % Maintain compatibility with old templates. Remove in nbconvert 6.0
    \let\Oldincludegraphics\includegraphics
    % Ensure that by default, figures have no caption (until we provide a
    % proper Figure object with a Caption API and a way to capture that
    % in the conversion process - todo).
    \usepackage{caption}
    \DeclareCaptionFormat{nocaption}{}
    \captionsetup{format=nocaption,aboveskip=0pt,belowskip=0pt}

    \usepackage[Export]{adjustbox} % Used to constrain images to a maximum size
    \adjustboxset{max size={0.9\linewidth}{0.9\paperheight}}
    \usepackage{float}
    \floatplacement{figure}{H} % forces figures to be placed at the correct location
    \usepackage{xcolor} % Allow colors to be defined
    \usepackage{enumerate} % Needed for markdown enumerations to work
    \usepackage{geometry} % Used to adjust the document margins
    \usepackage{amsmath} % Equations
    \usepackage{amssymb} % Equations
    \usepackage{textcomp} % defines textquotesingle
    % Hack from http://tex.stackexchange.com/a/47451/13684:
    \AtBeginDocument{%
        \def\PYZsq{\textquotesingle}% Upright quotes in Pygmentized code
    }
    \usepackage{upquote} % Upright quotes for verbatim code
    \usepackage{eurosym} % defines \euro
    \usepackage[mathletters]{ucs} % Extended unicode (utf-8) support
    \usepackage{fancyvrb} % verbatim replacement that allows latex

    % The hyperref package gives us a pdf with properly built
    % internal navigation ('pdf bookmarks' for the table of contents,
    % internal cross-reference links, web links for URLs, etc.)
    \usepackage{hyperref}
    % The default LaTeX title has an obnoxious amount of whitespace. By default,
    % titling removes some of it. It also provides customization options.
    \usepackage{titling}
    \usepackage{longtable} % longtable support required by pandoc >1.10
    \usepackage{booktabs}  % table support for pandoc > 1.12.2
    \usepackage[inline]{enumitem} % IRkernel/repr support (it uses the enumerate* environment)
    \usepackage[normalem]{ulem} % ulem is needed to support strikethroughs (\sout)
                                % normalem makes italics be italics, not underlines
    \usepackage{mathrsfs}
    

    
    % Colors for the hyperref package
    \definecolor{urlcolor}{rgb}{0,.145,.698}
    \definecolor{linkcolor}{rgb}{.71,0.21,0.01}
    \definecolor{citecolor}{rgb}{.12,.54,.11}

    % ANSI colors
    \definecolor{ansi-black}{HTML}{3E424D}
    \definecolor{ansi-black-intense}{HTML}{282C36}
    \definecolor{ansi-red}{HTML}{E75C58}
    \definecolor{ansi-red-intense}{HTML}{B22B31}
    \definecolor{ansi-green}{HTML}{00A250}
    \definecolor{ansi-green-intense}{HTML}{007427}
    \definecolor{ansi-yellow}{HTML}{DDB62B}
    \definecolor{ansi-yellow-intense}{HTML}{B27D12}
    \definecolor{ansi-blue}{HTML}{208FFB}
    \definecolor{ansi-blue-intense}{HTML}{0065CA}
    \definecolor{ansi-magenta}{HTML}{D160C4}
    \definecolor{ansi-magenta-intense}{HTML}{A03196}
    \definecolor{ansi-cyan}{HTML}{60C6C8}
    \definecolor{ansi-cyan-intense}{HTML}{258F8F}
    \definecolor{ansi-white}{HTML}{C5C1B4}
    \definecolor{ansi-white-intense}{HTML}{A1A6B2}
    \definecolor{ansi-default-inverse-fg}{HTML}{FFFFFF}
    \definecolor{ansi-default-inverse-bg}{HTML}{000000}

    % commands and environments needed by pandoc snippets
    % extracted from the output of `pandoc -s`
    \providecommand{\tightlist}{%
      \setlength{\itemsep}{0pt}\setlength{\parskip}{0pt}}
    \DefineVerbatimEnvironment{Highlighting}{Verbatim}{commandchars=\\\{\}}
    % Add ',fontsize=\small' for more characters per line
    \newenvironment{Shaded}{}{}
    \newcommand{\KeywordTok}[1]{\textcolor[rgb]{0.00,0.44,0.13}{\textbf{{#1}}}}
    \newcommand{\DataTypeTok}[1]{\textcolor[rgb]{0.56,0.13,0.00}{{#1}}}
    \newcommand{\DecValTok}[1]{\textcolor[rgb]{0.25,0.63,0.44}{{#1}}}
    \newcommand{\BaseNTok}[1]{\textcolor[rgb]{0.25,0.63,0.44}{{#1}}}
    \newcommand{\FloatTok}[1]{\textcolor[rgb]{0.25,0.63,0.44}{{#1}}}
    \newcommand{\CharTok}[1]{\textcolor[rgb]{0.25,0.44,0.63}{{#1}}}
    \newcommand{\StringTok}[1]{\textcolor[rgb]{0.25,0.44,0.63}{{#1}}}
    \newcommand{\CommentTok}[1]{\textcolor[rgb]{0.38,0.63,0.69}{\textit{{#1}}}}
    \newcommand{\OtherTok}[1]{\textcolor[rgb]{0.00,0.44,0.13}{{#1}}}
    \newcommand{\AlertTok}[1]{\textcolor[rgb]{1.00,0.00,0.00}{\textbf{{#1}}}}
    \newcommand{\FunctionTok}[1]{\textcolor[rgb]{0.02,0.16,0.49}{{#1}}}
    \newcommand{\RegionMarkerTok}[1]{{#1}}
    \newcommand{\ErrorTok}[1]{\textcolor[rgb]{1.00,0.00,0.00}{\textbf{{#1}}}}
    \newcommand{\NormalTok}[1]{{#1}}
    
    % Additional commands for more recent versions of Pandoc
    \newcommand{\ConstantTok}[1]{\textcolor[rgb]{0.53,0.00,0.00}{{#1}}}
    \newcommand{\SpecialCharTok}[1]{\textcolor[rgb]{0.25,0.44,0.63}{{#1}}}
    \newcommand{\VerbatimStringTok}[1]{\textcolor[rgb]{0.25,0.44,0.63}{{#1}}}
    \newcommand{\SpecialStringTok}[1]{\textcolor[rgb]{0.73,0.40,0.53}{{#1}}}
    \newcommand{\ImportTok}[1]{{#1}}
    \newcommand{\DocumentationTok}[1]{\textcolor[rgb]{0.73,0.13,0.13}{\textit{{#1}}}}
    \newcommand{\AnnotationTok}[1]{\textcolor[rgb]{0.38,0.63,0.69}{\textbf{\textit{{#1}}}}}
    \newcommand{\CommentVarTok}[1]{\textcolor[rgb]{0.38,0.63,0.69}{\textbf{\textit{{#1}}}}}
    \newcommand{\VariableTok}[1]{\textcolor[rgb]{0.10,0.09,0.49}{{#1}}}
    \newcommand{\ControlFlowTok}[1]{\textcolor[rgb]{0.00,0.44,0.13}{\textbf{{#1}}}}
    \newcommand{\OperatorTok}[1]{\textcolor[rgb]{0.40,0.40,0.40}{{#1}}}
    \newcommand{\BuiltInTok}[1]{{#1}}
    \newcommand{\ExtensionTok}[1]{{#1}}
    \newcommand{\PreprocessorTok}[1]{\textcolor[rgb]{0.74,0.48,0.00}{{#1}}}
    \newcommand{\AttributeTok}[1]{\textcolor[rgb]{0.49,0.56,0.16}{{#1}}}
    \newcommand{\InformationTok}[1]{\textcolor[rgb]{0.38,0.63,0.69}{\textbf{\textit{{#1}}}}}
    \newcommand{\WarningTok}[1]{\textcolor[rgb]{0.38,0.63,0.69}{\textbf{\textit{{#1}}}}}
    
    
    % Define a nice break command that doesn't care if a line doesn't already
    % exist.
    \def\br{\hspace*{\fill} \\* }
    % Math Jax compatibility definitions
    \def\gt{>}
    \def\lt{<}
    \let\Oldtex\TeX
    \let\Oldlatex\LaTeX
    \renewcommand{\TeX}{\textrm{\Oldtex}}
    \renewcommand{\LaTeX}{\textrm{\Oldlatex}}
    % Document parameters
    % Document title
    \title{corrige}
    
    
    
    
    
% Pygments definitions
\makeatletter
\def\PY@reset{\let\PY@it=\relax \let\PY@bf=\relax%
    \let\PY@ul=\relax \let\PY@tc=\relax%
    \let\PY@bc=\relax \let\PY@ff=\relax}
\def\PY@tok#1{\csname PY@tok@#1\endcsname}
\def\PY@toks#1+{\ifx\relax#1\empty\else%
    \PY@tok{#1}\expandafter\PY@toks\fi}
\def\PY@do#1{\PY@bc{\PY@tc{\PY@ul{%
    \PY@it{\PY@bf{\PY@ff{#1}}}}}}}
\def\PY#1#2{\PY@reset\PY@toks#1+\relax+\PY@do{#2}}

\expandafter\def\csname PY@tok@w\endcsname{\def\PY@tc##1{\textcolor[rgb]{0.73,0.73,0.73}{##1}}}
\expandafter\def\csname PY@tok@c\endcsname{\let\PY@it=\textit\def\PY@tc##1{\textcolor[rgb]{0.25,0.50,0.50}{##1}}}
\expandafter\def\csname PY@tok@cp\endcsname{\def\PY@tc##1{\textcolor[rgb]{0.74,0.48,0.00}{##1}}}
\expandafter\def\csname PY@tok@k\endcsname{\let\PY@bf=\textbf\def\PY@tc##1{\textcolor[rgb]{0.00,0.50,0.00}{##1}}}
\expandafter\def\csname PY@tok@kp\endcsname{\def\PY@tc##1{\textcolor[rgb]{0.00,0.50,0.00}{##1}}}
\expandafter\def\csname PY@tok@kt\endcsname{\def\PY@tc##1{\textcolor[rgb]{0.69,0.00,0.25}{##1}}}
\expandafter\def\csname PY@tok@o\endcsname{\def\PY@tc##1{\textcolor[rgb]{0.40,0.40,0.40}{##1}}}
\expandafter\def\csname PY@tok@ow\endcsname{\let\PY@bf=\textbf\def\PY@tc##1{\textcolor[rgb]{0.67,0.13,1.00}{##1}}}
\expandafter\def\csname PY@tok@nb\endcsname{\def\PY@tc##1{\textcolor[rgb]{0.00,0.50,0.00}{##1}}}
\expandafter\def\csname PY@tok@nf\endcsname{\def\PY@tc##1{\textcolor[rgb]{0.00,0.00,1.00}{##1}}}
\expandafter\def\csname PY@tok@nc\endcsname{\let\PY@bf=\textbf\def\PY@tc##1{\textcolor[rgb]{0.00,0.00,1.00}{##1}}}
\expandafter\def\csname PY@tok@nn\endcsname{\let\PY@bf=\textbf\def\PY@tc##1{\textcolor[rgb]{0.00,0.00,1.00}{##1}}}
\expandafter\def\csname PY@tok@ne\endcsname{\let\PY@bf=\textbf\def\PY@tc##1{\textcolor[rgb]{0.82,0.25,0.23}{##1}}}
\expandafter\def\csname PY@tok@nv\endcsname{\def\PY@tc##1{\textcolor[rgb]{0.10,0.09,0.49}{##1}}}
\expandafter\def\csname PY@tok@no\endcsname{\def\PY@tc##1{\textcolor[rgb]{0.53,0.00,0.00}{##1}}}
\expandafter\def\csname PY@tok@nl\endcsname{\def\PY@tc##1{\textcolor[rgb]{0.63,0.63,0.00}{##1}}}
\expandafter\def\csname PY@tok@ni\endcsname{\let\PY@bf=\textbf\def\PY@tc##1{\textcolor[rgb]{0.60,0.60,0.60}{##1}}}
\expandafter\def\csname PY@tok@na\endcsname{\def\PY@tc##1{\textcolor[rgb]{0.49,0.56,0.16}{##1}}}
\expandafter\def\csname PY@tok@nt\endcsname{\let\PY@bf=\textbf\def\PY@tc##1{\textcolor[rgb]{0.00,0.50,0.00}{##1}}}
\expandafter\def\csname PY@tok@nd\endcsname{\def\PY@tc##1{\textcolor[rgb]{0.67,0.13,1.00}{##1}}}
\expandafter\def\csname PY@tok@s\endcsname{\def\PY@tc##1{\textcolor[rgb]{0.73,0.13,0.13}{##1}}}
\expandafter\def\csname PY@tok@sd\endcsname{\let\PY@it=\textit\def\PY@tc##1{\textcolor[rgb]{0.73,0.13,0.13}{##1}}}
\expandafter\def\csname PY@tok@si\endcsname{\let\PY@bf=\textbf\def\PY@tc##1{\textcolor[rgb]{0.73,0.40,0.53}{##1}}}
\expandafter\def\csname PY@tok@se\endcsname{\let\PY@bf=\textbf\def\PY@tc##1{\textcolor[rgb]{0.73,0.40,0.13}{##1}}}
\expandafter\def\csname PY@tok@sr\endcsname{\def\PY@tc##1{\textcolor[rgb]{0.73,0.40,0.53}{##1}}}
\expandafter\def\csname PY@tok@ss\endcsname{\def\PY@tc##1{\textcolor[rgb]{0.10,0.09,0.49}{##1}}}
\expandafter\def\csname PY@tok@sx\endcsname{\def\PY@tc##1{\textcolor[rgb]{0.00,0.50,0.00}{##1}}}
\expandafter\def\csname PY@tok@m\endcsname{\def\PY@tc##1{\textcolor[rgb]{0.40,0.40,0.40}{##1}}}
\expandafter\def\csname PY@tok@gh\endcsname{\let\PY@bf=\textbf\def\PY@tc##1{\textcolor[rgb]{0.00,0.00,0.50}{##1}}}
\expandafter\def\csname PY@tok@gu\endcsname{\let\PY@bf=\textbf\def\PY@tc##1{\textcolor[rgb]{0.50,0.00,0.50}{##1}}}
\expandafter\def\csname PY@tok@gd\endcsname{\def\PY@tc##1{\textcolor[rgb]{0.63,0.00,0.00}{##1}}}
\expandafter\def\csname PY@tok@gi\endcsname{\def\PY@tc##1{\textcolor[rgb]{0.00,0.63,0.00}{##1}}}
\expandafter\def\csname PY@tok@gr\endcsname{\def\PY@tc##1{\textcolor[rgb]{1.00,0.00,0.00}{##1}}}
\expandafter\def\csname PY@tok@ge\endcsname{\let\PY@it=\textit}
\expandafter\def\csname PY@tok@gs\endcsname{\let\PY@bf=\textbf}
\expandafter\def\csname PY@tok@gp\endcsname{\let\PY@bf=\textbf\def\PY@tc##1{\textcolor[rgb]{0.00,0.00,0.50}{##1}}}
\expandafter\def\csname PY@tok@go\endcsname{\def\PY@tc##1{\textcolor[rgb]{0.53,0.53,0.53}{##1}}}
\expandafter\def\csname PY@tok@gt\endcsname{\def\PY@tc##1{\textcolor[rgb]{0.00,0.27,0.87}{##1}}}
\expandafter\def\csname PY@tok@err\endcsname{\def\PY@bc##1{\setlength{\fboxsep}{0pt}\fcolorbox[rgb]{1.00,0.00,0.00}{1,1,1}{\strut ##1}}}
\expandafter\def\csname PY@tok@kc\endcsname{\let\PY@bf=\textbf\def\PY@tc##1{\textcolor[rgb]{0.00,0.50,0.00}{##1}}}
\expandafter\def\csname PY@tok@kd\endcsname{\let\PY@bf=\textbf\def\PY@tc##1{\textcolor[rgb]{0.00,0.50,0.00}{##1}}}
\expandafter\def\csname PY@tok@kn\endcsname{\let\PY@bf=\textbf\def\PY@tc##1{\textcolor[rgb]{0.00,0.50,0.00}{##1}}}
\expandafter\def\csname PY@tok@kr\endcsname{\let\PY@bf=\textbf\def\PY@tc##1{\textcolor[rgb]{0.00,0.50,0.00}{##1}}}
\expandafter\def\csname PY@tok@bp\endcsname{\def\PY@tc##1{\textcolor[rgb]{0.00,0.50,0.00}{##1}}}
\expandafter\def\csname PY@tok@fm\endcsname{\def\PY@tc##1{\textcolor[rgb]{0.00,0.00,1.00}{##1}}}
\expandafter\def\csname PY@tok@vc\endcsname{\def\PY@tc##1{\textcolor[rgb]{0.10,0.09,0.49}{##1}}}
\expandafter\def\csname PY@tok@vg\endcsname{\def\PY@tc##1{\textcolor[rgb]{0.10,0.09,0.49}{##1}}}
\expandafter\def\csname PY@tok@vi\endcsname{\def\PY@tc##1{\textcolor[rgb]{0.10,0.09,0.49}{##1}}}
\expandafter\def\csname PY@tok@vm\endcsname{\def\PY@tc##1{\textcolor[rgb]{0.10,0.09,0.49}{##1}}}
\expandafter\def\csname PY@tok@sa\endcsname{\def\PY@tc##1{\textcolor[rgb]{0.73,0.13,0.13}{##1}}}
\expandafter\def\csname PY@tok@sb\endcsname{\def\PY@tc##1{\textcolor[rgb]{0.73,0.13,0.13}{##1}}}
\expandafter\def\csname PY@tok@sc\endcsname{\def\PY@tc##1{\textcolor[rgb]{0.73,0.13,0.13}{##1}}}
\expandafter\def\csname PY@tok@dl\endcsname{\def\PY@tc##1{\textcolor[rgb]{0.73,0.13,0.13}{##1}}}
\expandafter\def\csname PY@tok@s2\endcsname{\def\PY@tc##1{\textcolor[rgb]{0.73,0.13,0.13}{##1}}}
\expandafter\def\csname PY@tok@sh\endcsname{\def\PY@tc##1{\textcolor[rgb]{0.73,0.13,0.13}{##1}}}
\expandafter\def\csname PY@tok@s1\endcsname{\def\PY@tc##1{\textcolor[rgb]{0.73,0.13,0.13}{##1}}}
\expandafter\def\csname PY@tok@mb\endcsname{\def\PY@tc##1{\textcolor[rgb]{0.40,0.40,0.40}{##1}}}
\expandafter\def\csname PY@tok@mf\endcsname{\def\PY@tc##1{\textcolor[rgb]{0.40,0.40,0.40}{##1}}}
\expandafter\def\csname PY@tok@mh\endcsname{\def\PY@tc##1{\textcolor[rgb]{0.40,0.40,0.40}{##1}}}
\expandafter\def\csname PY@tok@mi\endcsname{\def\PY@tc##1{\textcolor[rgb]{0.40,0.40,0.40}{##1}}}
\expandafter\def\csname PY@tok@il\endcsname{\def\PY@tc##1{\textcolor[rgb]{0.40,0.40,0.40}{##1}}}
\expandafter\def\csname PY@tok@mo\endcsname{\def\PY@tc##1{\textcolor[rgb]{0.40,0.40,0.40}{##1}}}
\expandafter\def\csname PY@tok@ch\endcsname{\let\PY@it=\textit\def\PY@tc##1{\textcolor[rgb]{0.25,0.50,0.50}{##1}}}
\expandafter\def\csname PY@tok@cm\endcsname{\let\PY@it=\textit\def\PY@tc##1{\textcolor[rgb]{0.25,0.50,0.50}{##1}}}
\expandafter\def\csname PY@tok@cpf\endcsname{\let\PY@it=\textit\def\PY@tc##1{\textcolor[rgb]{0.25,0.50,0.50}{##1}}}
\expandafter\def\csname PY@tok@c1\endcsname{\let\PY@it=\textit\def\PY@tc##1{\textcolor[rgb]{0.25,0.50,0.50}{##1}}}
\expandafter\def\csname PY@tok@cs\endcsname{\let\PY@it=\textit\def\PY@tc##1{\textcolor[rgb]{0.25,0.50,0.50}{##1}}}

\def\PYZbs{\char`\\}
\def\PYZus{\char`\_}
\def\PYZob{\char`\{}
\def\PYZcb{\char`\}}
\def\PYZca{\char`\^}
\def\PYZam{\char`\&}
\def\PYZlt{\char`\<}
\def\PYZgt{\char`\>}
\def\PYZsh{\char`\#}
\def\PYZpc{\char`\%}
\def\PYZdl{\char`\$}
\def\PYZhy{\char`\-}
\def\PYZsq{\char`\'}
\def\PYZdq{\char`\"}
\def\PYZti{\char`\~}
% for compatibility with earlier versions
\def\PYZat{@}
\def\PYZlb{[}
\def\PYZrb{]}
\makeatother


    % For linebreaks inside Verbatim environment from package fancyvrb. 
    \makeatletter
        \newbox\Wrappedcontinuationbox 
        \newbox\Wrappedvisiblespacebox 
        \newcommand*\Wrappedvisiblespace {\textcolor{red}{\textvisiblespace}} 
        \newcommand*\Wrappedcontinuationsymbol {\textcolor{red}{\llap{\tiny$\m@th\hookrightarrow$}}} 
        \newcommand*\Wrappedcontinuationindent {3ex } 
        \newcommand*\Wrappedafterbreak {\kern\Wrappedcontinuationindent\copy\Wrappedcontinuationbox} 
        % Take advantage of the already applied Pygments mark-up to insert 
        % potential linebreaks for TeX processing. 
        %        {, <, #, %, $, ' and ": go to next line. 
        %        _, }, ^, &, >, - and ~: stay at end of broken line. 
        % Use of \textquotesingle for straight quote. 
        \newcommand*\Wrappedbreaksatspecials {% 
            \def\PYGZus{\discretionary{\char`\_}{\Wrappedafterbreak}{\char`\_}}% 
            \def\PYGZob{\discretionary{}{\Wrappedafterbreak\char`\{}{\char`\{}}% 
            \def\PYGZcb{\discretionary{\char`\}}{\Wrappedafterbreak}{\char`\}}}% 
            \def\PYGZca{\discretionary{\char`\^}{\Wrappedafterbreak}{\char`\^}}% 
            \def\PYGZam{\discretionary{\char`\&}{\Wrappedafterbreak}{\char`\&}}% 
            \def\PYGZlt{\discretionary{}{\Wrappedafterbreak\char`\<}{\char`\<}}% 
            \def\PYGZgt{\discretionary{\char`\>}{\Wrappedafterbreak}{\char`\>}}% 
            \def\PYGZsh{\discretionary{}{\Wrappedafterbreak\char`\#}{\char`\#}}% 
            \def\PYGZpc{\discretionary{}{\Wrappedafterbreak\char`\%}{\char`\%}}% 
            \def\PYGZdl{\discretionary{}{\Wrappedafterbreak\char`\$}{\char`\$}}% 
            \def\PYGZhy{\discretionary{\char`\-}{\Wrappedafterbreak}{\char`\-}}% 
            \def\PYGZsq{\discretionary{}{\Wrappedafterbreak\textquotesingle}{\textquotesingle}}% 
            \def\PYGZdq{\discretionary{}{\Wrappedafterbreak\char`\"}{\char`\"}}% 
            \def\PYGZti{\discretionary{\char`\~}{\Wrappedafterbreak}{\char`\~}}% 
        } 
        % Some characters . , ; ? ! / are not pygmentized. 
        % This macro makes them "active" and they will insert potential linebreaks 
        \newcommand*\Wrappedbreaksatpunct {% 
            \lccode`\~`\.\lowercase{\def~}{\discretionary{\hbox{\char`\.}}{\Wrappedafterbreak}{\hbox{\char`\.}}}% 
            \lccode`\~`\,\lowercase{\def~}{\discretionary{\hbox{\char`\,}}{\Wrappedafterbreak}{\hbox{\char`\,}}}% 
            \lccode`\~`\;\lowercase{\def~}{\discretionary{\hbox{\char`\;}}{\Wrappedafterbreak}{\hbox{\char`\;}}}% 
            \lccode`\~`\:\lowercase{\def~}{\discretionary{\hbox{\char`\:}}{\Wrappedafterbreak}{\hbox{\char`\:}}}% 
            \lccode`\~`\?\lowercase{\def~}{\discretionary{\hbox{\char`\?}}{\Wrappedafterbreak}{\hbox{\char`\?}}}% 
            \lccode`\~`\!\lowercase{\def~}{\discretionary{\hbox{\char`\!}}{\Wrappedafterbreak}{\hbox{\char`\!}}}% 
            \lccode`\~`\/\lowercase{\def~}{\discretionary{\hbox{\char`\/}}{\Wrappedafterbreak}{\hbox{\char`\/}}}% 
            \catcode`\.\active
            \catcode`\,\active 
            \catcode`\;\active
            \catcode`\:\active
            \catcode`\?\active
            \catcode`\!\active
            \catcode`\/\active 
            \lccode`\~`\~ 	
        }
    \makeatother

    \let\OriginalVerbatim=\Verbatim
    \makeatletter
    \renewcommand{\Verbatim}[1][1]{%
        %\parskip\z@skip
        \sbox\Wrappedcontinuationbox {\Wrappedcontinuationsymbol}%
        \sbox\Wrappedvisiblespacebox {\FV@SetupFont\Wrappedvisiblespace}%
        \def\FancyVerbFormatLine ##1{\hsize\linewidth
            \vtop{\raggedright\hyphenpenalty\z@\exhyphenpenalty\z@
                \doublehyphendemerits\z@\finalhyphendemerits\z@
                \strut ##1\strut}%
        }%
        % If the linebreak is at a space, the latter will be displayed as visible
        % space at end of first line, and a continuation symbol starts next line.
        % Stretch/shrink are however usually zero for typewriter font.
        \def\FV@Space {%
            \nobreak\hskip\z@ plus\fontdimen3\font minus\fontdimen4\font
            \discretionary{\copy\Wrappedvisiblespacebox}{\Wrappedafterbreak}
            {\kern\fontdimen2\font}%
        }%
        
        % Allow breaks at special characters using \PYG... macros.
        \Wrappedbreaksatspecials
        % Breaks at punctuation characters . , ; ? ! and / need catcode=\active 	
        \OriginalVerbatim[#1,codes*=\Wrappedbreaksatpunct]%
    }
    \makeatother

    % Exact colors from NB
    \definecolor{incolor}{HTML}{303F9F}
    \definecolor{outcolor}{HTML}{D84315}
    \definecolor{cellborder}{HTML}{CFCFCF}
    \definecolor{cellbackground}{HTML}{F7F7F7}
    
    % prompt
    \makeatletter
    \newcommand{\boxspacing}{\kern\kvtcb@left@rule\kern\kvtcb@boxsep}
    \makeatother
    \newcommand{\prompt}[4]{
        \ttfamily\llap{{\color{#2}[#3]:\hspace{3pt}#4}}\vspace{-\baselineskip}
    }
    

    
\setlength\headheight{30pt}
\setcounter{secnumdepth}{0} % Turns off numbering for sections

    % Prevent overflowing lines due to hard-to-break entities
    \sloppy 
    % Setup hyperref package
    \hypersetup{
      breaklinks=true,  % so long urls are correctly broken across lines
      colorlinks=true,
      urlcolor=urlcolor,
      linkcolor=linkcolor,
      citecolor=citecolor,
      }
    % Slightly bigger margins than the latex defaults
    \geometry{a4paper,tmargin=3cm,bmargin=2cm,lmargin=1cm,rmargin=1cm}\fancyhead[L]{Thème à définir}\fancyhead[C]{\bfseries\MakeUppercase{corrige}}\fancyhead[R]{\monthyeardate\today}

    \fancyfoot[C]{\thepage}
    % #TODO ajouter les pages totales

    \pagestyle{fancy}
    


\begin{document}
    
    \author{Pascal Padilla}\title{corrige}
% \maketitle

    
    

    
    \hypertarget{correction-bac-blanc}{%
\section{Correction Bac Blanc}\label{correction-bac-blanc}}

    \hypertarget{exercice-1}{%
\subsection{Exercice 1}\label{exercice-1}}

    1.a - La requête de l'énoncé affiche la liste de tous les ordinateurs et
affiche pour chacun sa marque et sa salle associée. C'est une relation
de deux attributs. Elle produit l'affichage de la table suivante :

\begin{longtable}[]{@{}ll@{}}
\toprule
salle & marque\_ordi\tabularnewline
\midrule
\endhead
012 & HP\tabularnewline
114 & Lenovo\tabularnewline
223 & Dell\tabularnewline
223 & Dell\tabularnewline
223 & Dell\tabularnewline
\bottomrule
\end{longtable}

    1.b - La requête de l'énoncé affiche la liste des noms et des salles de
tous les ordinateurs reliés à un vidéoprojecteur. Elle produit
l'affichage de la table suivante :

\begin{longtable}[]{@{}ll@{}}
\toprule
nom\_ordi & salle\tabularnewline
\midrule
\endhead
Gen-24 & 012\tabularnewline
Tech-62 & 114\tabularnewline
Gen-132 & 223\tabularnewline
\bottomrule
\end{longtable}

    2 - La requête donnant tous les attributs des ordinateurs correspondant
aux années supérieures ou égales à 2017 ordonnées par dates croissantes
est :

\begin{Shaded}
\begin{Highlighting}[]
\KeywordTok{SELECT} \OperatorTok{*} \KeywordTok{FROM}\NormalTok{ Ordinateur}
\KeywordTok{WHERE}\NormalTok{ anne }\OperatorTok{\textgreater{}=} \DecValTok{2017}
\KeywordTok{ORDER} \KeywordTok{BY}\NormalTok{ annee }\KeywordTok{ASC}\NormalTok{;}
\end{Highlighting}
\end{Shaded}

    3.a - Pour des raisons de contrainte d'intégrité, l'attribut
\texttt{salle} ne peut pas être une clé primaire. En effet, la clé
primaire de chaque élément de la relation \texttt{Ordinateur} doit être
\textbf{unique} ce qui n'est pas le cas de l'attribut proposé.

    3.b - En respectant les notations de l'énoncé, la relation
\texttt{Imprimante} se définit :

\[
\begin{array}{ll}
\texttt{Imprimante(}\\
    \texttt{\qquad\underline{nom\_imprimante: String},} \\
    \texttt{\qquad\underline{nom\_ordi: String},} \\
    \texttt{\qquad marque\_imp: String,}\\
    \texttt{\qquad modele\_imp: String,}\\
    \texttt{\qquad salle: String)}
\end{array}
\]

\texttt{nom\_ordi} est une clé étrangère de la relation
\texttt{Imprimante} car c'est une clé primaire de la relation
\texttt{Ordinateur}.

    4.a - Pour insérer le vidéoprojecteur de l'énoncé en salle 315 il faudra
écrire la requête :

```sql INSERT INTO Videoprojecteur(salle, marque\_video, modele\_video,
tni) VALUES (`315', `NEC', `ME402X', false);

    4.b - La requête nécessite une jointure :

```sql SELECT o.salle, o.nom\_ordi, v.marque\_video FROM Ordinateur AS o
JOIN Videoprojecteur AS v ON o.salle = v.salle WHERE v.tni = false ;

    \hypertarget{exercice-2}{%
\subsection{Exercice 2}\label{exercice-2}}

    1.a - L'abre possède 7 noeuds. Il a donc une taille égale à 7.

    1.b - La hauteur de l'arbre est de 4. C'est le nombre de nœuds du plus
grand chemin entre la racine et ses feuilles.

    2 - L'arbre suivant possède les mêmes valeurs que celui de l'énoncé mais
est bien construit :

\begin{verbatim}
     10
   /    \
  5     15
 / \   /  \
4   8 12  20
\end{verbatim}

    3 - Les instructions de l'énoncé produisent l'arbre suivant :

\begin{verbatim}
         10
      /      \
     /        \
    8          20
  /            /
 4            15
/ \          /  \
   5        12
  / \      / \
\end{verbatim}

    4 - La méthode \texttt{hauteur} de la classe \texttt{Noeud} renvoie la
hauteur d'un noeud en partant de la racine. Ainsi une feuille a une
hauteur de 1, un noeud relié à une feuille a une hauteur de 2 et ainsi
de suite.

Pour implémenter la méthode hauteur de la classe arbre, il suffit de
renvoyer la hauteur du nœud racine. Ce qui donne :

\begin{Shaded}
\begin{Highlighting}[]
\KeywordTok{class}\NormalTok{ Arbre:}

\NormalTok{    ...}

    \KeywordTok{def}\NormalTok{ hauteur(}\VariableTok{self}\NormalTok{):}
        \ControlFlowTok{return} \VariableTok{self}\NormalTok{.racine.hauteur()}
\end{Highlighting}
\end{Shaded}

    5 - La méthode \texttt{taille} de la classe \texttt{Noeud} renvoie la
taille du sous-arbre de racine le noeud. Elle peut être implémentée de
la façon suivante :

\begin{Shaded}
\begin{Highlighting}[]
\KeywordTok{class}\NormalTok{ Noeud:}

\NormalTok{    ...}

    \KeywordTok{def}\NormalTok{ taille(}\VariableTok{self}\NormalTok{):}
        \CommentTok{\# cas de base}
        \CommentTok{\# le noeud est une feuille}
        \ControlFlowTok{if} \VariableTok{self}\NormalTok{.gauche }\OperatorTok{==} \VariableTok{None} \KeywordTok{and} \VariableTok{self}\NormalTok{.droit }\OperatorTok{==} \VariableTok{None}\NormalTok{:}
            \ControlFlowTok{return} \DecValTok{1}
        
        \ControlFlowTok{if} \VariableTok{self}\NormalTok{.gauche }\OperatorTok{==} \VariableTok{None}\NormalTok{:}
\NormalTok{            t\_gauche }\OperatorTok{=} \DecValTok{0}
        \ControlFlowTok{else}\NormalTok{:}
\NormalTok{            t\_gauche }\OperatorTok{=} \VariableTok{self}\NormalTok{.gauche.taille()}
        
        \ControlFlowTok{if} \VariableTok{self}\NormalTok{.droit }\OperatorTok{==} \VariableTok{None}\NormalTok{:}
\NormalTok{            t\_droit }\OperatorTok{=} \DecValTok{0}
        \ControlFlowTok{else}\NormalTok{:}
\NormalTok{            t\_droit }\OperatorTok{=} \VariableTok{self}\NormalTok{.droit.taille()}

        \ControlFlowTok{return} \DecValTok{1} \OperatorTok{+}\NormalTok{ t\_gauche }\OperatorTok{+}\NormalTok{ t\_droit}
\end{Highlighting}
\end{Shaded}

On peut maintenant implémenter la méthode \texttt{taille} pour la classe
\texttt{Arbre} en remarquant que la taille d'un arbre est égal à la
taille de sa racine !

\begin{Shaded}
\begin{Highlighting}[]
\KeywordTok{class}\NormalTok{ Arbre:}
    
\NormalTok{    ...}

    \KeywordTok{def}\NormalTok{ taille(}\VariableTok{self}\NormalTok{):}
        \ControlFlowTok{return} \VariableTok{self}\NormalTok{.racine.taille()}
\end{Highlighting}
\end{Shaded}

    6.a - Un arbre binaire de recherche est \emph{bien construit} s'il n'est
pas possible de le \emph{réduire} à un arbre de hauteur \(h-1\) car
sinon, la propriété \emph{bien construit} ne serait pas vérifiée. Donc
un tel arbre doit avoir une taille supérieure aux arbres de hauteur
\(h-1\), c'est-à dire que sa taille doit être supérieure à
\(2^{h-1} - 1\).

On a donc l'encadrement suivant pour un arbre bien construit :

\(2 ^{h-1} - 1 < t \leq 2^h - 1\).

La deuxième partie de l'inégalité est vraie pour tout ABR, mais la
première partie est caractéristique des ABR bien construit.

    6.b - Implémentons la méthode \texttt{bien\_construit} qui s'appuie sur
une telle propriété :

```python class Arbre:

\begin{verbatim}
...

def bien_contruit(self):
    t = self.taille()
    h = self.hauteur()

    if  t > 2 ** (h-1) - 1:
        return True
    else:
        return False
\end{verbatim}

    \hypertarget{exercice-3}{%
\subsection{Exercice 3}\label{exercice-3}}

    1.a - Il faut implémenter une fonction renvoyant la somme des éléments
d'un tableau donné en argument.

        {\scriptsize
    \begin{tcolorbox}[breakable, size=fbox, boxrule=1pt, pad at break*=1mm,colback=cellbackground, colframe=cellborder]
\prompt{In}{incolor}{2}{\boxspacing}
\begin{Verbatim}[commandchars=\\\{\}]
\PY{k}{def} \PY{n+nf}{total\PYZus{}hors\PYZus{}reduction}\PY{p}{(}\PY{n}{tab}\PY{p}{)}\PY{p}{:}
    \PY{n}{total} \PY{o}{=} \PY{l+m+mi}{0}
    \PY{n}{n} \PY{o}{=} \PY{n+nb}{len}\PY{p}{(}\PY{n}{tab}\PY{p}{)}
    \PY{k}{for} \PY{n}{i} \PY{o+ow}{in} \PY{n+nb}{range}\PY{p}{(}\PY{n}{n}\PY{p}{)}\PY{p}{:}
        \PY{n}{total} \PY{o}{=} \PY{n}{total} \PY{o}{+} \PY{n}{tab}\PY{p}{[}\PY{n}{i}\PY{p}{]}
    \PY{k}{return} \PY{n}{total}


\PY{n}{tab} \PY{o}{=} \PY{p}{[}\PY{l+m+mf}{30.5}\PY{p}{,} \PY{l+m+mf}{15.0}\PY{p}{,} \PY{l+m+mf}{6.0}\PY{p}{,} \PY{l+m+mf}{20.0}\PY{p}{,} \PY{l+m+mf}{5.0}\PY{p}{,} \PY{l+m+mf}{35.0}\PY{p}{,} \PY{l+m+mf}{10.5}\PY{p}{]}
\PY{n+nb}{print}\PY{p}{(}\PY{n}{total\PYZus{}hors\PYZus{}reduction}\PY{p}{(}\PY{n}{tab}\PY{p}{)}\PY{p}{)}
\end{Verbatim}
\end{tcolorbox}
    }

    \begin{Verbatim}[commandchars=\\\{\}]
122.0
    \end{Verbatim}

    1.b - Voici la version complète de la fonction donnée dans l'énoncé :

        {\scriptsize
    \begin{tcolorbox}[breakable, size=fbox, boxrule=1pt, pad at break*=1mm,colback=cellbackground, colframe=cellborder]
\prompt{In}{incolor}{4}{\boxspacing}
\begin{Verbatim}[commandchars=\\\{\}]
\PY{k}{def} \PY{n+nf}{offre\PYZus{}bienvenue}\PY{p}{(}\PY{n}{tab}\PY{p}{:} \PY{n+nb}{list}\PY{p}{)} \PY{o}{\PYZhy{}}\PY{o}{\PYZgt{}} \PY{n+nb}{float}\PY{p}{:}
    \PY{n}{somme} \PY{o}{=} \PY{l+m+mi}{0}
    \PY{n}{longueur} \PY{o}{=} \PY{n+nb}{len}\PY{p}{(}\PY{n}{tab}\PY{p}{)}
    \PY{k}{if} \PY{n}{longueur} \PY{o}{\PYZgt{}} \PY{l+m+mi}{0}\PY{p}{:}
        \PY{n}{somme} \PY{o}{=} \PY{n}{tab}\PY{p}{[}\PY{l+m+mi}{0}\PY{p}{]} \PY{o}{*} \PY{l+m+mf}{0.8}
    \PY{k}{if} \PY{n}{longueur} \PY{o}{\PYZgt{}} \PY{l+m+mi}{1}\PY{p}{:}
        \PY{n}{somme} \PY{o}{=} \PY{n}{somme} \PY{o}{+} \PY{n}{tab}\PY{p}{[}\PY{l+m+mi}{1}\PY{p}{]} \PY{o}{*} \PY{l+m+mf}{0.7}
    \PY{k}{if} \PY{n}{longueur} \PY{o}{\PYZgt{}} \PY{l+m+mi}{2}\PY{p}{:}
        \PY{k}{for} \PY{n}{i} \PY{o+ow}{in} \PY{n+nb}{range}\PY{p}{(}\PY{l+m+mi}{2}\PY{p}{,} \PY{n}{longueur}\PY{p}{)}\PY{p}{:}
            \PY{n}{somme} \PY{o}{=} \PY{n}{somme} \PY{o}{+} \PY{n}{tab}\PY{p}{[}\PY{n}{i}\PY{p}{]}
    \PY{k}{return} \PY{n}{somme}


\PY{n+nb}{print}\PY{p}{(}\PY{n}{offre\PYZus{}bienvenue}\PY{p}{(}\PY{n}{tab}\PY{p}{)}\PY{p}{)}
\end{Verbatim}
\end{tcolorbox}
    }

    \begin{Verbatim}[commandchars=\\\{\}]
111.4
    \end{Verbatim}

    2 - Pour implémenter la fonction demandée, il faut différencier tous les
cas possibles :

        {\scriptsize
    \begin{tcolorbox}[breakable, size=fbox, boxrule=1pt, pad at break*=1mm,colback=cellbackground, colframe=cellborder]
\prompt{In}{incolor}{7}{\boxspacing}
\begin{Verbatim}[commandchars=\\\{\}]
\PY{k}{def} \PY{n+nf}{prix\PYZus{}solde}\PY{p}{(}\PY{n}{tab}\PY{p}{)}\PY{p}{:}
    \PY{n}{longueur} \PY{o}{=} \PY{n+nb}{len}\PY{p}{(}\PY{n}{tab}\PY{p}{)}
    \PY{n}{total} \PY{o}{=} \PY{n}{total\PYZus{}hors\PYZus{}reduction}\PY{p}{(}\PY{n}{tab}\PY{p}{)}
    \PY{k}{if}   \PY{n}{longueur} \PY{o}{\PYZgt{}}\PY{o}{=} \PY{l+m+mi}{5}\PY{p}{:}
        \PY{k}{return} \PY{n}{total} \PY{o}{*} \PY{l+m+mf}{0.5}
    \PY{k}{elif} \PY{n}{longueur} \PY{o}{==} \PY{l+m+mi}{4}\PY{p}{:}
        \PY{k}{return} \PY{n}{total} \PY{o}{*} \PY{l+m+mf}{0.6}
    \PY{k}{elif} \PY{n}{longueur} \PY{o}{==} \PY{l+m+mi}{3}\PY{p}{:}
        \PY{k}{return} \PY{n}{total} \PY{o}{*} \PY{l+m+mf}{0.7}
    \PY{k}{elif} \PY{n}{longueur} \PY{o}{==} \PY{l+m+mi}{2}\PY{p}{:}
        \PY{k}{return} \PY{n}{total} \PY{o}{*} \PY{l+m+mf}{0.8}
    \PY{k}{elif} \PY{n}{longueur} \PY{o}{==} \PY{l+m+mi}{1}\PY{p}{:}
        \PY{k}{return} \PY{n}{total} \PY{o}{*} \PY{l+m+mf}{0.9}
    \PY{k}{else}\PY{p}{:}
        \PY{k}{return} \PY{l+m+mi}{0}


\PY{n+nb}{print}\PY{p}{(}\PY{n}{prix\PYZus{}solde}\PY{p}{(}\PY{n}{tab}\PY{p}{)}\PY{p}{)}
\end{Verbatim}
\end{tcolorbox}
    }

    \begin{Verbatim}[commandchars=\\\{\}]
61.0
    \end{Verbatim}

    3.a - Implémentation de la fonction qui renvoie la valeur minimale d'un
tableau :

        {\scriptsize
    \begin{tcolorbox}[breakable, size=fbox, boxrule=1pt, pad at break*=1mm,colback=cellbackground, colframe=cellborder]
\prompt{In}{incolor}{4}{\boxspacing}
\begin{Verbatim}[commandchars=\\\{\}]
\PY{k}{def} \PY{n+nf}{minimum}\PY{p}{(}\PY{n}{tab}\PY{p}{)}\PY{p}{:}
    \PY{n}{mini} \PY{o}{=} \PY{n}{tab}\PY{p}{[}\PY{l+m+mi}{0}\PY{p}{]}
    \PY{n}{longueur} \PY{o}{=} \PY{n+nb}{len}\PY{p}{(}\PY{n}{tab}\PY{p}{)}
    \PY{k}{for}  \PY{n}{i} \PY{o+ow}{in} \PY{n+nb}{range}\PY{p}{(}\PY{l+m+mi}{1}\PY{p}{,} \PY{n}{longueur}\PY{p}{)}\PY{p}{:}
        \PY{k}{if} \PY{n}{tab}\PY{p}{[}\PY{n}{i}\PY{p}{]} \PY{o}{\PYZlt{}} \PY{n}{mini}\PY{p}{:}
            \PY{n}{mini} \PY{o}{=} \PY{n}{tab}\PY{p}{[}\PY{n}{i}\PY{p}{]}
    \PY{k}{return} \PY{n}{mini}


\PY{n+nb}{print}\PY{p}{(}\PY{n}{minimum}\PY{p}{(}\PY{n}{tab}\PY{p}{)}\PY{p}{)}
\end{Verbatim}
\end{tcolorbox}
    }

    \begin{Verbatim}[commandchars=\\\{\}]
5.0
    \end{Verbatim}

    3.b - Utilisons la fonction \texttt{minimum()} créée à la question
précédente :

        {\scriptsize
    \begin{tcolorbox}[breakable, size=fbox, boxrule=1pt, pad at break*=1mm,colback=cellbackground, colframe=cellborder]
\prompt{In}{incolor}{5}{\boxspacing}
\begin{Verbatim}[commandchars=\\\{\}]
\PY{k}{def} \PY{n+nf}{offre\PYZus{}bon\PYZus{}client}\PY{p}{(}\PY{n}{tab}\PY{p}{)}\PY{p}{:}
    \PY{n}{total} \PY{o}{=} \PY{n}{total\PYZus{}hors\PYZus{}reduction}\PY{p}{(}\PY{n}{tab}\PY{p}{)}
    \PY{n}{longueur} \PY{o}{=} \PY{n+nb}{len}\PY{p}{(}\PY{n}{tab}\PY{p}{)}
    \PY{k}{if} \PY{n}{longueur} \PY{o}{\PYZgt{}} \PY{l+m+mi}{1}\PY{p}{:}
        \PY{n}{mini} \PY{o}{=} \PY{n}{minimum}\PY{p}{(}\PY{n}{tab}\PY{p}{)}
        \PY{n}{total} \PY{o}{=} \PY{n}{total} \PY{o}{\PYZhy{}} \PY{n}{mini}
    \PY{k}{return} \PY{n}{total}


\PY{n+nb}{print}\PY{p}{(}\PY{n}{offre\PYZus{}bon\PYZus{}client}\PY{p}{(}\PY{n}{tab}\PY{p}{)}\PY{p}{)}
\end{Verbatim}
\end{tcolorbox}
    }

    \begin{Verbatim}[commandchars=\\\{\}]
117.0
    \end{Verbatim}

    4.a - Si on permute les articles à \texttt{6.0} et \texttt{20.0}, alors
on obtient le tableau
\texttt{{[}30.5,\ 15.0,\ 20.0,\ 6.0,\ 5.0,\ 35.0,\ 10.5{]}}. Pour ce
tableau, les articles à \texttt{15.0} et \texttt{5.0} sont offerts. Le
prix après promotion est donc différent de 111 euros.

    4.b - Pour avoir un prix le plus bas possible, je propose (par exemple)
le panier suivant :

\texttt{{[}35.0,\ 30.5,\ 20.0,\ 6.0,\ 10.5,\ 15.0,\ 5.0{]}}

Le total de remise s'élève à \(20,0 + 6,0 = 26,0\).

    4.c - Pour minimiser le coût il faut maximiser la remise. Ainsi, pour un
tableau donné, il faut arriver à mettre les articles les plus chers en
remise. Une méthode systématique pour arriver à cela est d'ordonner les
articles par \textbf{ordre décroissant}.

    \hypertarget{exercice-4}{%
\subsection{Exercice 4}\label{exercice-4}}

    1 - \textbf{réponse souvent incomplète sur les copies}

le couple \((1,3)\) est une inversion pour le tableau
\texttt{{[}4,\ 8,\ 3,\ 7{]}} car il respecte des deux propriétés :

\begin{itemize}
\tightlist
\item
  \(1 < 3\)
\item
  \(\texttt{tab[1]} = 8 > \texttt{tab[3]} = 7\)
\end{itemize}

    2 - le couple \((2, 3)\) n'est pas une inversion car :

\begin{itemize}
\tightlist
\item
  \(2 < 3\)
\item
  mais \(\texttt{tab[2]} = 3 > \texttt{tab[3]} = 7\) n'est pas vérifiée
\end{itemize}

    A.1.a et A.1.b - La fonction \texttt{fonction1(tab,\ i)} compte le
nombre d'inversion \((\texttt{i}, j)\) que contient le tableau
\texttt{tab} en partant du rang \texttt{i} fixé.

\begin{itemize}
\tightlist
\item
  \texttt{fonction1({[}1,\ 5,\ 3,\ 7{]},\ 0)} compte le nombre
  d'inversion de rang 0. Il n'y en a aucune donc la fonction renvoie
  \texttt{0}. En effet :

  \begin{itemize}
  \tightlist
  \item
    \(1<5\) donc le couple \((0, 1)\) n'est pas une inversion
  \item
    \(1<3\) donc le couple \((0, 2)\) n'est pas une inversion
  \item
    \(1<7\) donc le couple \((0, 3)\) n'est pas une inversion
  \end{itemize}
\item
  \texttt{fonction1({[}1,\ 5,\ 3,\ 7{]},\ 1)} compte le nombre
  d'inversion de rang 1. Il y en a une seule car \(5>3\) pour le couple
  \$(1, 2). La fonction renvoie donc \texttt{1}.
\item
  \texttt{fonction1({[}1,\ 5,\ 2,\ 6,\ 4{]},\ 1)} compte le nombre
  d'inversion de rang 1. Il y en a deux car \(5>2\) pour le couple \$(1,
  2) et \(5>4\) pour le couple \((1, 4)\). La fonction renvoie donc
  \texttt{2}.
\end{itemize}

        {\scriptsize
    \begin{tcolorbox}[breakable, size=fbox, boxrule=1pt, pad at break*=1mm,colback=cellbackground, colframe=cellborder]
\prompt{In}{incolor}{14}{\boxspacing}
\begin{Verbatim}[commandchars=\\\{\}]
\PY{k}{def} \PY{n+nf}{fonction1}\PY{p}{(}\PY{n}{tab}\PY{p}{,} \PY{n}{i}\PY{p}{)}\PY{p}{:}
    \PY{n}{nb\PYZus{}elem} \PY{o}{=} \PY{n+nb}{len}\PY{p}{(}\PY{n}{tab}\PY{p}{)}
    \PY{n}{cpt} \PY{o}{=} \PY{l+m+mi}{0}
    \PY{k}{for} \PY{n}{j} \PY{o+ow}{in} \PY{n+nb}{range}\PY{p}{(}\PY{n}{i}\PY{o}{+}\PY{l+m+mi}{1}\PY{p}{,} \PY{n}{nb\PYZus{}elem}\PY{p}{)}\PY{p}{:}
        \PY{k}{if} \PY{n}{tab}\PY{p}{[}\PY{n}{j}\PY{p}{]} \PY{o}{\PYZlt{}} \PY{n}{tab}\PY{p}{[}\PY{n}{i}\PY{p}{]}\PY{p}{:}
            \PY{n}{cpt} \PY{o}{+}\PY{o}{=} \PY{l+m+mi}{1}
    \PY{k}{return} \PY{n}{cpt}

\PY{n+nb}{print}\PY{p}{(}\PY{n}{fonction1}\PY{p}{(}\PY{p}{[}\PY{l+m+mi}{1}\PY{p}{,} \PY{l+m+mi}{5}\PY{p}{,} \PY{l+m+mi}{3}\PY{p}{,} \PY{l+m+mi}{7}\PY{p}{]}\PY{p}{,} \PY{l+m+mi}{0}\PY{p}{)}\PY{p}{)}
\PY{n+nb}{print}\PY{p}{(}\PY{n}{fonction1}\PY{p}{(}\PY{p}{[}\PY{l+m+mi}{1}\PY{p}{,} \PY{l+m+mi}{5}\PY{p}{,} \PY{l+m+mi}{3}\PY{p}{,} \PY{l+m+mi}{7}\PY{p}{]}\PY{p}{,} \PY{l+m+mi}{1}\PY{p}{)}\PY{p}{)}
\PY{n+nb}{print}\PY{p}{(}\PY{n}{fonction1}\PY{p}{(}\PY{p}{[}\PY{l+m+mi}{1}\PY{p}{,} \PY{l+m+mi}{5}\PY{p}{,} \PY{l+m+mi}{2}\PY{p}{,} \PY{l+m+mi}{6}\PY{p}{,} \PY{l+m+mi}{4}\PY{p}{]}\PY{p}{,} \PY{l+m+mi}{1}\PY{p}{)}\PY{p}{)}
\end{Verbatim}
\end{tcolorbox}
    }

    \begin{Verbatim}[commandchars=\\\{\}]
0
1
2
    \end{Verbatim}

    A.2 - Pour compter le nombre total d'inversions, on va accumuler toutes
les inversions de rang 0, puis toutes celles de rang 1, et ainsi de
suite jusqu'à l'avant dernière case case du tableau.

        {\scriptsize
    \begin{tcolorbox}[breakable, size=fbox, boxrule=1pt, pad at break*=1mm,colback=cellbackground, colframe=cellborder]
\prompt{In}{incolor}{20}{\boxspacing}
\begin{Verbatim}[commandchars=\\\{\}]
\PY{k}{def} \PY{n+nf}{nombre\PYZus{}inversions}\PY{p}{(}\PY{n}{tab}\PY{p}{)}\PY{p}{:}
    \PY{n}{total} \PY{o}{=} \PY{l+m+mi}{0}
    \PY{n}{longueur} \PY{o}{=} \PY{n+nb}{len}\PY{p}{(}\PY{n}{tab}\PY{p}{)}
    \PY{k}{for} \PY{n}{i} \PY{o+ow}{in} \PY{n+nb}{range}\PY{p}{(}\PY{n}{longueur}\PY{o}{\PYZhy{}}\PY{l+m+mi}{1}\PY{p}{)}\PY{p}{:}
        \PY{n}{total} \PY{o}{=} \PY{n}{total} \PY{o}{+} \PY{n}{fonction1}\PY{p}{(}\PY{n}{tab}\PY{p}{,} \PY{n}{i}\PY{p}{)}
    \PY{k}{return} \PY{n}{total}

\PY{n+nb}{print} \PY{p}{(}\PY{n}{nombre\PYZus{}inversions}\PY{p}{(}\PY{p}{[}\PY{l+m+mi}{1}\PY{p}{,} \PY{l+m+mi}{5}\PY{p}{,} \PY{l+m+mi}{7}\PY{p}{]}\PY{p}{)}\PY{p}{)}
\PY{n+nb}{print} \PY{p}{(}\PY{n}{nombre\PYZus{}inversions}\PY{p}{(}\PY{p}{[}\PY{l+m+mi}{1}\PY{p}{,} \PY{l+m+mi}{6}\PY{p}{,} \PY{l+m+mi}{2}\PY{p}{,} \PY{l+m+mi}{7}\PY{p}{,} \PY{l+m+mi}{3}\PY{p}{]}\PY{p}{)}\PY{p}{)}
\PY{n+nb}{print} \PY{p}{(}\PY{n}{nombre\PYZus{}inversions}\PY{p}{(}\PY{p}{[}\PY{l+m+mi}{7}\PY{p}{,} \PY{l+m+mi}{6}\PY{p}{,} \PY{l+m+mi}{5}\PY{p}{,} \PY{l+m+mi}{3}\PY{p}{]}\PY{p}{)}\PY{p}{)}
\end{Verbatim}
\end{tcolorbox}
    }

    \begin{Verbatim}[commandchars=\\\{\}]
0
3
6
    \end{Verbatim}

    A.3 - Soit \texttt{n} la longueur du tableau, la fonction
\texttt{nombre\_inversion} effectue \texttt{n} boucles.

Chaque boucle effectue \texttt{n} appels à \texttt{fonction1}, qui elle
même effectue \texttt{i} tours de boucles.

La fonction \texttt{nombre\_inversions} effectue donc
\(n + (n-1) + (n-2) + \ldots + 2 + 1 = \frac{n(n+1)}{2}\) tests.

L'ordre de grandeur de la complexité en temps est donc \(n^2\) avec
\(n\) la longueur du tableau.

C'est une complexité quadratique.

    B.1 - Un algorithme de tri qui a une complexité meilleure que
quadratique est le tri \textbf{fusion} ou le tri \textbf{rapide}. Ces
deux ont des complexité quasi-linéaire en \(n\log(n)\).

    B.2 -

        {\scriptsize
    \begin{tcolorbox}[breakable, size=fbox, boxrule=1pt, pad at break*=1mm,colback=cellbackground, colframe=cellborder]
\prompt{In}{incolor}{7}{\boxspacing}
\begin{Verbatim}[commandchars=\\\{\}]
\PY{k}{def} \PY{n+nf}{moitie\PYZus{}gauche}\PY{p}{(}\PY{n}{tab}\PY{p}{)}\PY{p}{:}
    \PY{k}{if} \PY{n}{tab} \PY{o}{==} \PY{p}{[}\PY{p}{]}\PY{p}{:}
        \PY{k}{return} \PY{p}{[}\PY{p}{]}

    \PY{n}{n} \PY{o}{=} \PY{n+nb}{len}\PY{p}{(}\PY{n}{tab}\PY{p}{)}
    \PY{n}{milieu} \PY{o}{=} \PY{p}{(}\PY{n}{n}\PY{o}{\PYZhy{}}\PY{l+m+mi}{1}\PY{p}{)} \PY{o}{/}\PY{o}{/} \PY{l+m+mi}{2}

    \PY{n}{m\PYZus{}gauche} \PY{o}{=} \PY{p}{[}\PY{k+kc}{None}\PY{p}{]} \PY{o}{*} \PY{p}{(}\PY{n}{milieu}\PY{o}{+}\PY{l+m+mi}{1}\PY{p}{)}
    \PY{k}{for} \PY{n}{i} \PY{o+ow}{in} \PY{n+nb}{range}\PY{p}{(}\PY{n}{milieu} \PY{o}{+} \PY{l+m+mi}{1}\PY{p}{)}\PY{p}{:}
        \PY{n}{m\PYZus{}gauche}\PY{p}{[}\PY{n}{i}\PY{p}{]} \PY{o}{=} \PY{n}{tab}\PY{p}{[}\PY{n}{i}\PY{p}{]}
    
    \PY{k}{return} \PY{n}{m\PYZus{}gauche}
\end{Verbatim}
\end{tcolorbox}
    }

    B.3 - Implémentation :

\begin{Shaded}
\begin{Highlighting}[]
\KeywordTok{def}\NormalTok{ nb\_inversions\_rec(tab):}
\NormalTok{    tab\_g }\OperatorTok{=}\NormalTok{ moitie\_gauche(tab)}
\NormalTok{    nb\_inv\_gauche }\OperatorTok{=}\NormalTok{ nb\_inversions\_rec(tab\_g)}
    
\NormalTok{    tab\_d  }\OperatorTok{=}\NormalTok{ moitie\_droit(tab)}
\NormalTok{    nb\_inv\_droit }\OperatorTok{=}\NormalTok{ nb\_inversions\_rec(tab\_d)}

\NormalTok{    tab\_g\_trie }\OperatorTok{=}\NormalTok{ tri(tab\_g)}
\NormalTok{    tab\_d\_trie }\OperatorTok{=}\NormalTok{ tri(tab\_d)}
\NormalTok{    nb\_inv\_tries }\OperatorTok{=}\NormalTok{ nb\_inv\_tab(tab\_g\_trie, tab\_d\_trie)}

\NormalTok{    nb\_inv }\OperatorTok{=}\NormalTok{ nb\_inv\_gauche }\OperatorTok{+}\NormalTok{ nb\_inv\_droit }\OperatorTok{+}\NormalTok{ nb\_inv\_tries}
    \ControlFlowTok{return}\NormalTok{ nb\_inv}
\end{Highlighting}
\end{Shaded}

    \hypertarget{exercice-5}{%
\subsection{Exercice 5}\label{exercice-5}}

    1.a - À la fin de l'exécution, la file \(F\) est vide et la pile \(P\)
contient le contenu de \(F\) initial inversé :

\begin{itemize}
\tightlist
\item
  \(F : \text{enfilement}\rightarrow \empty \rightarrow \text{défilement}\)
\item
  \(P : \text{empilement/dépilement}\leftrightarrow \text{"rouge" "vert" "jaune" "rouge" "jaune"}\)
\end{itemize}

    1.b - Pour déterminer la taille, nous allons vider la file originale
dans une file temporaire, initialement vide. À chaque défilement, nous
incrémentons le compteur permettant un dénombrement.

Ensuite, pour remettre en état la file originale \(F\), on défile la
file temporaire dans la \(F\) qui était devenue vide.

        {\scriptsize
    \begin{tcolorbox}[breakable, size=fbox, boxrule=1pt, pad at break*=1mm,colback=cellbackground, colframe=cellborder]
\prompt{In}{incolor}{10}{\boxspacing}
\begin{Verbatim}[commandchars=\\\{\}]
\PY{k}{def} \PY{n+nf}{taille\PYZus{}file}\PY{p}{(}\PY{n}{F}\PY{p}{)}\PY{p}{:}
    \PY{n}{taille} \PY{o}{=} \PY{l+m+mi}{0}

    \PY{n}{F\PYZus{}temp} \PY{o}{=} \PY{n}{creer\PYZus{}file\PYZus{}vide}\PY{p}{(}\PY{p}{)}
    \PY{k}{while} \PY{o+ow}{not} \PY{n}{est\PYZus{}vide}\PY{p}{(}\PY{n}{F}\PY{p}{)}\PY{p}{:}
        \PY{n}{taille} \PY{o}{=} \PY{n}{taille} \PY{o}{+} \PY{l+m+mi}{1}
        \PY{n}{enfiler}\PY{p}{(}\PY{n}{F\PYZus{}temp}\PY{p}{,} \PY{n}{defiler}\PY{p}{(}\PY{n}{F}\PY{p}{)}\PY{p}{)}
    
    \PY{k}{while} \PY{o+ow}{not} \PY{n}{est\PYZus{}vide}\PY{p}{(}\PY{n}{F\PYZus{}temp}\PY{p}{)}\PY{p}{:}
        \PY{n}{enfiler}\PY{p}{(}\PY{n}{F}\PY{p}{,} \PY{n}{defiler}\PY{p}{(}\PY{n}{F\PYZus{}temp}\PY{p}{)}\PY{p}{)}
    
    \PY{k}{return} \PY{n}{taille}
\end{Verbatim}
\end{tcolorbox}
    }

    2 - L'idée de l'algorithme est la suivante :

\begin{enumerate}
\def\labelenumi{\arabic{enumi}.}
\tightlist
\item
  vider la file \(F\) dans une pile temporaire (comme le fait la
  question 1.a)
\item
  retourner/inverser la pile temporaire en la dépilant dans une deuxième
  pile
\item
  renvoyer la deuxième pile qui contient les mêmes éléments que \(F\),
  dans le bon ordre.
\end{enumerate}

        {\scriptsize
    \begin{tcolorbox}[breakable, size=fbox, boxrule=1pt, pad at break*=1mm,colback=cellbackground, colframe=cellborder]
\prompt{In}{incolor}{13}{\boxspacing}
\begin{Verbatim}[commandchars=\\\{\}]
\PY{k}{def} \PY{n+nf}{former\PYZus{}pile}\PY{p}{(}\PY{n}{F}\PY{p}{)}\PY{p}{:}
    \PY{c+c1}{\PYZsh{} pile temporaire qui contiendra les valeurs de F}
    \PY{c+c1}{\PYZsh{} mais dans l\PYZsq{}ordre inversé}
    \PY{n}{P\PYZus{}temp} \PY{o}{=} \PY{n}{creer\PYZus{}pile\PYZus{}vide}\PY{p}{(}\PY{p}{)}
    \PY{k}{while} \PY{o+ow}{not} \PY{n}{est\PYZus{}vide}\PY{p}{(}\PY{n}{F}\PY{p}{)}\PY{p}{:}
        \PY{n}{empiler}\PY{p}{(}\PY{n}{P\PYZus{}temp}\PY{p}{,} \PY{n}{defiler}\PY{p}{(}\PY{n}{F}\PY{p}{)}\PY{p}{)}
    
    \PY{c+c1}{\PYZsh{} retourner/inverser la pile temporaire}
    \PY{n}{P} \PY{o}{=} \PY{n}{creer\PYZus{}pile\PYZus{}vide}\PY{p}{(}\PY{p}{)}
    \PY{k}{while} \PY{o+ow}{not} \PY{n}{est\PYZus{}vide}\PY{p}{(}\PY{n}{P\PYZus{}temp}\PY{p}{)}\PY{p}{:}
        \PY{n}{empiler}\PY{p}{(}\PY{n}{P}\PY{p}{,} \PY{n}{depiler}\PY{p}{(}\PY{n}{P\PYZus{}temp}\PY{p}{)}\PY{p}{)}
    
    \PY{k}{return} \PY{n}{P}      
\end{Verbatim}
\end{tcolorbox}
    }

    Voici une deuxième implémentation qui va remettre en état la file \(F\)
(au lieu de la laisser vide). L'algorithme est le suivant :

\begin{itemize}
\tightlist
\item
  défiler \(F\) et mettre chaque élément dans une pile temporaire et
  dans une file temporaire
\item
  remettre en état la file \(F\) en parcourant/vidant la file temporaire
\item
  dépiler la pile temporaire dans la pile finale afin de remettre les
  éléments dans le bon ordre
\item
  renvoyer la pile finale
\end{itemize}

        {\scriptsize
    \begin{tcolorbox}[breakable, size=fbox, boxrule=1pt, pad at break*=1mm,colback=cellbackground, colframe=cellborder]
\prompt{In}{incolor}{14}{\boxspacing}
\begin{Verbatim}[commandchars=\\\{\}]
\PY{k}{def} \PY{n+nf}{former\PYZus{}pile}\PY{p}{(}\PY{n}{F}\PY{p}{)}\PY{p}{:}
    \PY{n}{F\PYZus{}temp} \PY{o}{=} \PY{n}{creer\PYZus{}file\PYZus{}vide}\PY{p}{(}\PY{p}{)}
    \PY{n}{P\PYZus{}temp} \PY{o}{=} \PY{n}{creer\PYZus{}pile\PYZus{}vide}\PY{p}{(}\PY{p}{)}

    \PY{k}{while} \PY{o+ow}{not} \PY{n}{est\PYZus{}vide}\PY{p}{(}\PY{n}{F}\PY{p}{)}\PY{p}{:}
        \PY{n}{element} \PY{o}{=} \PY{n}{defiler}\PY{p}{(}\PY{n}{F}\PY{p}{)}
        \PY{n}{enfiler}\PY{p}{(}\PY{n}{F\PYZus{}temp}\PY{p}{,} \PY{n}{element}\PY{p}{)}
        \PY{n}{empiler}\PY{p}{(}\PY{n}{P\PYZus{}temp}\PY{p}{,} \PY{n}{element}\PY{p}{)}
    
    \PY{c+c1}{\PYZsh{} remise en état de F}
    \PY{k}{while} \PY{o+ow}{not} \PY{n}{est\PYZus{}vide}\PY{p}{(}\PY{n}{F\PYZus{}temp}\PY{p}{)}\PY{p}{:}
        \PY{n}{enfiler}\PY{p}{(}\PY{n}{F}\PY{p}{,} \PY{n}{defiler}\PY{p}{(}\PY{n}{F\PYZus{}temp}\PY{p}{)}\PY{p}{)}
    
    \PY{c+c1}{\PYZsh{} inversion de la pile P\PYZus{}temp}
    \PY{c+c1}{\PYZsh{} dans la pile P à renvoyer}
    \PY{n}{P} \PY{o}{=} \PY{n}{creer\PYZus{}pile\PYZus{}vide}\PY{p}{(}\PY{p}{)}
    \PY{k}{while} \PY{o+ow}{not} \PY{n}{est\PYZus{}vide}\PY{p}{(}\PY{n}{P\PYZus{}temp}\PY{p}{)}\PY{p}{:}
        \PY{n}{empiler}\PY{p}{(}\PY{n}{P}\PY{p}{,} \PY{n}{depiler}\PY{p}{(}\PY{n}{P\PYZus{}temp}\PY{p}{)}\PY{p}{)}
    
    \PY{k}{return} \PY{n}{P}
\end{Verbatim}
\end{tcolorbox}
    }

    3 - L'algorithme proposé ressemble beaucoup à celui de
\texttt{taille\_fle}. Cette fois-ci, on ajoute 1 au total seulement si
l'élément défilé est égal à l'élément passé en argument de la fonction.

        {\scriptsize
    \begin{tcolorbox}[breakable, size=fbox, boxrule=1pt, pad at break*=1mm,colback=cellbackground, colframe=cellborder]
\prompt{In}{incolor}{15}{\boxspacing}
\begin{Verbatim}[commandchars=\\\{\}]
\PY{k}{def} \PY{n+nf}{nb\PYZus{}elements}\PY{p}{(}\PY{n}{F}\PY{p}{,} \PY{n}{elt}\PY{p}{)}\PY{p}{:}
    \PY{n}{total} \PY{o}{=} \PY{l+m+mi}{0}

    \PY{n}{F\PYZus{}temp} \PY{o}{=} \PY{n}{creer\PYZus{}file\PYZus{}vide}\PY{p}{(}\PY{p}{)}
    \PY{k}{while} \PY{o+ow}{not}\PY{p}{(}\PY{n}{est\PYZus{}vide}\PY{p}{(}\PY{n}{F}\PY{p}{)}\PY{p}{)}\PY{p}{:}
        \PY{n}{elt\PYZus{}courant} \PY{o}{=} \PY{n}{defiler}\PY{p}{(}\PY{n}{F}\PY{p}{)}
        \PY{k}{if} \PY{n}{elt\PYZus{}courant} \PY{o}{==} \PY{n}{elt}\PY{p}{:}
            \PY{n}{total} \PY{o}{=} \PY{n}{total} \PY{o}{+} \PY{l+m+mi}{1}
        \PY{n}{enfiler}\PY{p}{(}\PY{n}{F\PYZus{}temp}\PY{p}{,} \PY{n}{elt\PYZus{}courant}\PY{p}{)}
    
    \PY{c+c1}{\PYZsh{} remise en état de F}
    \PY{k}{while} \PY{o+ow}{not}\PY{p}{(}\PY{n}{est\PYZus{}vide}\PY{p}{(}\PY{n}{F\PYZus{}temp}\PY{p}{)}\PY{p}{)}\PY{p}{:}
        \PY{n}{enfiler}\PY{p}{(}\PY{n}{F}\PY{p}{,} \PY{n}{defiler}\PY{p}{(}\PY{n}{F\PYZus{}temp}\PY{p}{)}\PY{p}{)}

    \PY{k}{return} \PY{n}{total}
\end{Verbatim}
\end{tcolorbox}
    }

    4 - L'implémentation proposée est la suivante :

\begin{itemize}
\tightlist
\item
  compter le nombre d'élément \texttt{"rouge"} de la file. Si ce nombre
  est plus grand que \texttt{nb\_rouge} passé en argument, alors le
  contenu n'est pas correct et la fonction s'arrête en renvoyant
  \texttt{False}.
\item
  on fait de même avec les éléments \texttt{"vert"}, puis avec les
  éléments \texttt{"jaune"}
\item
  si la fonction arrive à passer les trois tests précédents, alors c'est
  qu'elle est correcte et elle s'arrête en renvoyant \texttt{True}.
\end{itemize}

        {\scriptsize
    \begin{tcolorbox}[breakable, size=fbox, boxrule=1pt, pad at break*=1mm,colback=cellbackground, colframe=cellborder]
\prompt{In}{incolor}{11}{\boxspacing}
\begin{Verbatim}[commandchars=\\\{\}]
\PY{k}{def} \PY{n+nf}{verifier\PYZus{}contenu}\PY{p}{(}\PY{n}{F}\PY{p}{,} \PY{n}{nb\PYZus{}rouge}\PY{p}{,} \PY{n}{nb\PYZus{}vert}\PY{p}{,} \PY{n}{nb\PYZus{}jaune}\PY{p}{)}\PY{p}{:}
    \PY{n}{rouge\PYZus{}reel} \PY{o}{=} \PY{n}{nb\PYZus{}elements}\PY{p}{(}\PY{n}{F}\PY{p}{,} \PY{l+s+s2}{\PYZdq{}}\PY{l+s+s2}{rouge}\PY{l+s+s2}{\PYZdq{}}\PY{p}{)}
    \PY{k}{if} \PY{n}{rouge\PYZus{}reel} \PY{o}{\PYZgt{}} \PY{n}{nb\PYZus{}rouge}\PY{p}{:}
        \PY{k}{return} \PY{k+kc}{False}
    
    \PY{n}{vert\PYZus{}reel} \PY{o}{=} \PY{n}{nb\PYZus{}elements}\PY{p}{(}\PY{n}{F}\PY{p}{,} \PY{l+s+s2}{\PYZdq{}}\PY{l+s+s2}{vert}\PY{l+s+s2}{\PYZdq{}}\PY{p}{)}
    \PY{k}{if} \PY{n}{vert\PYZus{}reel} \PY{o}{\PYZgt{}} \PY{n}{nb\PYZus{}vert}\PY{p}{:}
        \PY{k}{return} \PY{k+kc}{False}
    
    \PY{n}{jaune\PYZus{}reel} \PY{o}{=} \PY{n}{nb\PYZus{}elements}\PY{p}{(}\PY{n}{F}\PY{p}{,} \PY{l+s+s2}{\PYZdq{}}\PY{l+s+s2}{jaune}\PY{l+s+s2}{\PYZdq{}}\PY{p}{)}
    \PY{k}{if} \PY{n}{jaune\PYZus{}reel} \PY{o}{\PYZgt{}} \PY{n}{nb\PYZus{}jaune}\PY{p}{:}
        \PY{k}{return} \PY{k+kc}{False}
    
    \PY{k}{return} \PY{k+kc}{True}
\end{Verbatim}
\end{tcolorbox}
    }


    % Add a bibliography block to the postdoc
    
    
    
\end{document}
